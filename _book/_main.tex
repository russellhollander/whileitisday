% Options for packages loaded elsewhere
\PassOptionsToPackage{unicode}{hyperref}
\PassOptionsToPackage{hyphens}{url}
%
\documentclass[
]{book}
\usepackage{amsmath,amssymb}
\usepackage{iftex}
\ifPDFTeX
  \usepackage[T1]{fontenc}
  \usepackage[utf8]{inputenc}
  \usepackage{textcomp} % provide euro and other symbols
\else % if luatex or xetex
  \usepackage{unicode-math} % this also loads fontspec
  \defaultfontfeatures{Scale=MatchLowercase}
  \defaultfontfeatures[\rmfamily]{Ligatures=TeX,Scale=1}
\fi
\usepackage{lmodern}
\ifPDFTeX\else
  % xetex/luatex font selection
\fi
% Use upquote if available, for straight quotes in verbatim environments
\IfFileExists{upquote.sty}{\usepackage{upquote}}{}
\IfFileExists{microtype.sty}{% use microtype if available
  \usepackage[]{microtype}
  \UseMicrotypeSet[protrusion]{basicmath} % disable protrusion for tt fonts
}{}
\makeatletter
\@ifundefined{KOMAClassName}{% if non-KOMA class
  \IfFileExists{parskip.sty}{%
    \usepackage{parskip}
  }{% else
    \setlength{\parindent}{0pt}
    \setlength{\parskip}{6pt plus 2pt minus 1pt}}
}{% if KOMA class
  \KOMAoptions{parskip=half}}
\makeatother
\usepackage{xcolor}
\usepackage{longtable,booktabs,array}
\usepackage{calc} % for calculating minipage widths
% Correct order of tables after \paragraph or \subparagraph
\usepackage{etoolbox}
\makeatletter
\patchcmd\longtable{\par}{\if@noskipsec\mbox{}\fi\par}{}{}
\makeatother
% Allow footnotes in longtable head/foot
\IfFileExists{footnotehyper.sty}{\usepackage{footnotehyper}}{\usepackage{footnote}}
\makesavenoteenv{longtable}
\usepackage{graphicx}
\makeatletter
\def\maxwidth{\ifdim\Gin@nat@width>\linewidth\linewidth\else\Gin@nat@width\fi}
\def\maxheight{\ifdim\Gin@nat@height>\textheight\textheight\else\Gin@nat@height\fi}
\makeatother
% Scale images if necessary, so that they will not overflow the page
% margins by default, and it is still possible to overwrite the defaults
% using explicit options in \includegraphics[width, height, ...]{}
\setkeys{Gin}{width=\maxwidth,height=\maxheight,keepaspectratio}
% Set default figure placement to htbp
\makeatletter
\def\fps@figure{htbp}
\makeatother
\setlength{\emergencystretch}{3em} % prevent overfull lines
\providecommand{\tightlist}{%
  \setlength{\itemsep}{0pt}\setlength{\parskip}{0pt}}
\setcounter{secnumdepth}{5}
\ifLuaTeX
  \usepackage{selnolig}  % disable illegal ligatures
\fi
\IfFileExists{bookmark.sty}{\usepackage{bookmark}}{\usepackage{hyperref}}
\IfFileExists{xurl.sty}{\usepackage{xurl}}{} % add URL line breaks if available
\urlstyle{same}
\hypersetup{
  pdftitle={While It Is Day!},
  pdfauthor={Paul E. Kretzmann},
  hidelinks,
  pdfcreator={LaTeX via pandoc}}

\title{While It Is Day!}
\author{Paul E. Kretzmann}
\date{1926}

\begin{document}
\maketitle

{
\setcounter{tocdepth}{1}
\tableofcontents
}
\hypertarget{while-it-is-day.-john-9-4}{%
\chapter*{While It Is Day. John 9, 4}\label{while-it-is-day.-john-9-4}}

\begin{verbatim}
Let us work while it is day,
Let us labor while we may!
Soon will come the night of eath
When we yield our final breath.
Let us work for Chirst, the Lord,
Bear aloft His mighty Word!

Let us work without restraint,
Never weary, lax or faint;
Firm in battling selfishness,
Loyal in true faithfulness,
Place our strength at His command,
Do His work throughout the land!

Let us work while yet we may!
Soon will come that glorious day
When our labor here si done
When the precious prize is won,
When we rest and take our ease
In the homes of endless peace!
\end{verbatim}

(\emph{Knowing and Doing}, p.82.)

\hypertarget{foreword}{%
\chapter*{Foreword}\label{foreword}}

In these days of spiritual and moral decay the question is often asked, ``What's wrong with the church?'' And this question, while often presented by those who have never given the Church a fair trial, is, in a general way, warranted; for there is no institution in the entire world that can do aso much good and has such high and holy responsibilities as the Church. Where else shall the world turn, which is so torn by doubt and fear? Where else, fi not to the Church with its saving Gospel, the only message of hope? Yes, the world is sick, and the Church has the only remedy that can cure its ills.

And in these days, when the Battle between Christianity and the world is becoming harder and harder and the line of demarcation separating the two is becoming more and more indistinct; when worldliness is making such terrible inroads in the Church; when love is becoming cold and indifference and laxity are paralyzing Christian activities; when, as the Savior predicted, there are all the indications that the world's destruction is near, -- in this our evil time, what does the Church itself need more than a deeper appreciation for the Gospel and a reconsecration to the spreading of this Gospel? The need in our Church is not, primarily, more men and means, nor more churches and members, but congregations filled with a greater soul-winning spirit, members with a deeper passion for blood-bought souls, members, we repeat, who realize that an actually overwhelming responsibility rests upon them and that their sacred ans supreme duty in this world is to testify for Christ by word and deed and thus to help save souls.

While we have heard it stated again and again -- and it is also brought out in this volume -- that two-thirds of the world, or nearly a thousand million souls, do not konw Jesus, the Savior, we fail to work as determinedly and self-sacrificially as we should to bring our fellow-sinners the tidings of salvation. The great commission, ``Go ye into all the world and preach the Gospel to every creature,'' often remains unfulfilled. yet all our church- and school-work has this one object in view, this goal -- to help save souls.

Many and ever more soul-winners -- that is the need of the hour! When our members have caught the soul-saving spirit, and when their hearts are set on fire to save men, greater advancements will be made in our mission-fields here and zbroad. There will be a larger outpouring of gifts, a self-sacrificing stewardship. Our members will not be satisfied with giving merely part of their time, spare change, and to make half-hearted efforts, but they will give the best of their time, the best of their money, and put their whole heart and soul into the great work. And there will be a solution to so many discouraging church problems. The Church will grow inwardly and outwardly as never before.

Is it necessary to point out that our young people, the future leaders of the Church, should be in the very front ranks of the Church's missionary crusade? Indeed, young and old should consider it the highest privilege to be colaborers with God in the work of salvation of men through Jesus Christ. But our youth must lead. In this spirit the leaders of the Walther League at the Deteroit Convention proposed the so-called systematic mission endeavor. This is merely a united and organized effort of the young people, under the direction of their pastors, to lead others to the Savior and His Church. The plan was most whole-heartedly endorsed by succeeding conventions. The endeavor, which is gaining in favor, has already wrought untold blessings for the Church in deepengin the spirituality of the young people.

Realizing that the young people, while taking an active part, can only assist the pastors and congregations, the scope of \emph{While It Is Day!} was widened to be of direct use to all congregations of the entire Church and for all individual members, young and old. It covers all practical phases; and where detailed information is not give, there are enough suggestions to help any judicious leader. \emph{While It Is Day!} is also to serve as a text-book, and it is suggested that the various studies be taken up systematically in regular classes during six-to-eight-week periods. The spirit of a united mission endeavor is well expressed in the title \emph{While It Is Day!} There is no time to be lost. Every hour and every minute men are perishing without the saving Gospel. ``The night cometh when no man can work''; so let us labor ``while it is day.''

P.G. Prokopy

\hypertarget{go-ye-matt.-28-19}{%
\chapter{Go Ye! Matt. 28, 19}\label{go-ye-matt.-28-19}}

\hypertarget{the-divine-commission.}{%
\section*{The Divine Commission.}\label{the-divine-commission.}}
\addcontentsline{toc}{section}{The Divine Commission.}

It is God's perogative to have people come to Him, and His invitation is extended to all men in a serious, efficacious call.

``Come unto Me, all ye that labor and are heavy laden, and I will give you rest!'' Matt. 11, 28.

``Look unto Me and be ye saved, all the ends of the earth.'' Is. 45, 22.

``Come, for all things are now ready.'' Luke 14, 17.

``And the Spirit and the bride say, Come! And let him that heareth say, Come! And let him that is athirst come. And whosoever will, let him take the water of life freely.'' Rev.~22, 17.

But to make known this glorious intention and invitation to men \emph{God has commissioned His children}, the believers, throughout the world. They are to be His representatives; they are to be His messengers, His ambassadors; they are to be His agents in making known His call of salvation, in inviting men to the feast of His love and grace.

Even in Old Testament times this was true. To His Zion, to the members of His Church under the Old Dispensation, the Lord calls out: --

``O Zion, that bringest good tidings, get thee up into a high mountain! O Jerusalem, that bringest good tidings, lift up thy voice with strength; lift it up, be not afraid; say unto the cities of Judah, Behold your God!'' Is. 40, 9.

But still more direct, still more unmistakable and powerful, is the Lord's commission to the New Testament Church and to all its members: --

``GO YE THEREFORE AND TEACH ALL NATIONS; BAPTIZING THEM IN THE NAME OF THE FATHER AND OF THE SON AND OF THE HOLY GHOST; TEACHING THEM TO OBSERVE ALL THINGS WHATSOEVER I HAVE COMMANDED YOU.'' Matt. 28, 19. 20.

``Go ye!'' He says.

``Bring My sons from far and My daughters from the ends of the earth!'' Is. 43, 6.

``Ye shall be witnesses unto Me both in Jerusalem, and in all Judea, and in Samaria, and unto the uttermost part of the earth.'' Acts 1, 8.

It is not enough that we build churches and chapels and have free pews for all who desire to come; it is not enough that we erect bulletin-boards and announcement-boards in front of our churches, at street intersections, and in other public places; it is not enough that we publish parish-papers and pulpit programs; it is not enough that we advertise in the newspapers on Saturdays and upon all special occasions. All this is good and laudable; all this means carrying out a part of the work which is ours to do; all this may reach souls that are in need of the message of salvation; all this may bring some into the fold.

``GO!'' means \emph{personal work}, if possible work in person, by direct personal contact; it means seeing that work is done and carrying out the work in person, if possible, or attending to its performance in person, if it must be done through others.

``Go ye also into the vineyard!'' Matt 20, 4. 7. That is the Lord's Commission. It might be done by proxy, of course; but where does a person do his own work, which is entrusted to him, by proxy, unless it be, perhaps, under his own direct, personal supervision?

And there is more to be considered. The Bridegroom, Christ, is sending His friends to win the bride, the believers, whom He wants with Him in the enjoyment of the eternal bliss of heaven. St.~Paul says of his own work in winning souls for Christ: --

``I have espoused you to one Husband that I may present you as a chaste virgin to Christ.'' 2 Cor. 11, 2. And John the Baptist testified: --

``The Friend of the Bridegroom which standeth and heareth Him, rejoiceth greatly because of the Bridegroom's voice. This my joy therefore is fulfilled.'' John 3, 29. John, like Paul, did his work for Christ in person; he gained souls by personal work. It is the way we ought to choose as we have opportunity and in accordance with the method of working set forth by God in His Holy Word.

``Go YE!'' says Christ. The commission is not confined to the apostles. They, indeed, are the teachers of the whole world until the end of time. Through their word others are to believe on Jesus Christ, the Savior of the world. John 17, 20.

``Go YE!'' is not addressed to trained pastors and missionaries only, although to them is committed the task of public preaching. It was a little servant girl, a slave whose name is not even mentioned in the Bible, who called the attention of Naaman's wife to the prophet of Jehovah in Israel. 2 Kings 5, 3. It was the untrained fisherman, Andrew, who told his brother Simon about the Messiah, John 1, 41, and whose example was followed by Philip in speaking to Nathanel. John 1, 45.

``Go ye INTO ALL THE WORLD!'' is the Savior's commission. The world is big and wide, and there are still more than twice as many without the knowledge of Christ as there are, even nominally, within the pale of the Church. INto many of the heathen lands the messengers of the Gospel have gone. Into some of the heathen lands some of our own messengers are gone -- alas! into all too few. There never was a truer answer given than that turned in my a young Christian, the question being: How many missionaries have we in India and China? He said, with wonderful frankness: ``NOT ENOUGH!''

Yes, NOT ENOUGH! We have not reached nearly all lands of the world. We have made but a feeble, an all too feeble, beginning.

Somehow we do not seem to realize the misery of the untold millions that are ``without God in the world.'' Eph 2, 12. Perhaps if they were living across the street from us and we had the picture of their misery and their idolatry and their vileness before our eyes every day, our hearts would be stirred to a greater effort in their behalf, and we might accomplish more in our foreign mission endeavor.

~~We smoothly speak of teeming masses, of millions still in darkest night;\\
\hspace*{0.333em}\hspace*{0.333em}We glibly pray that those in blindness be given spiritual sight;\\
\hspace*{0.333em}\hspace*{0.333em}We prate about our mission duty and of the missionary need:\\
\hspace*{0.333em}\hspace*{0.333em}But what would you do, and what would I do,\\
\hspace*{0.333em}\hspace*{0.333em}IF CHINA WERE ACROSS THE STREET?

~~We say that we are interested when now and then we hear a talk\\
\hspace*{0.333em}\hspace*{0.333em}Of how the heathen hosts are living and in the fiercest horrors walk,\\
\hspace*{0.333em}\hspace*{0.333em}How they are kept the truth from learning, their leaders empty husks them feed:\\
\hspace*{0.333em}\hspace*{0.333em}But what would you do, and what would I do,\\
\hspace*{0.333em}\hspace*{0.333em}IF INDIA WERE ACROSS THE STREET?

~~We feel that we have done our duty when we just sometimes give a mite,\\
\hspace*{0.333em}\hspace*{0.333em}When from the riches of our treasures we now and then deal out a bite;\\
\hspace*{0.333em}\hspace*{0.333em}We spend our billions for vain baubles, for luxuries we do not need:\\
\hspace*{0.333em}\hspace*{0.333em}But what would you do, and what would I do,\\
\hspace*{0.333em}\hspace*{0.333em}WITH AFRICA ACROSS THE STREET?

~~* * *

~~Oh, may the love of Christ constrain us to see our mission duty through\\
\hspace*{0.333em}\hspace*{0.333em}That we be filled with burning fervor, that less we talk and more we do\\
\hspace*{0.333em}\hspace*{0.333em}That we no longer speak of burdens, but lift the misery untold\\
\hspace*{0.333em}\hspace*{0.333em}THAT ALL OUR LIFE BE SPENT IN BRINGING MORE SOULS INTO THE SAVIOR'S FOLD

It is absolutely necessary that we get this better viewpoint, that we begin to think of our missionary duty in terms of direct contact, that we visualize the spiritual needs of those who are still children of wrath without being conscious of that fact.

``INTO ALL THE WORLD!'' Not only the foreign countries, but also the home field, the country in which we live!

~~If you cannot cross the ocean\\
\hspace*{0.333em}\hspace*{0.333em}And the heathen lands explore,\\
\hspace*{0.333em}\hspace*{0.333em}You can find the heathen nearer,\\
\hspace*{0.333em}\hspace*{0.333em}You can help them at your door!

Ah, yes; millions of them, in the very midst of Christianity and civilization!

Do you know whether your nearest neighbors are members of a Christian Church? Have you ever inquired whether the people across the street know anything about the Savior and the way of salvation?

Have you ever considered that thousands of us who are sitting at the full tables of God's riches in Christ Jesus as we have them in our dear Lutheran Church have done little or nothing to bring the Gospel of God's mercy to men and women and children in our own neighborhood?

~~Shall we, whose souls are lighted\\
\hspace*{0.333em}\hspace*{0.333em}With wisdom from on high,\\
\hspace*{0.333em}\hspace*{0.333em}Shall we to men benighted\\
\hspace*{0.333em}\hspace*{0.333em}The lamp of life deny?

Shall we do so by failing to make an earnest effort to reach them by \emph{going} and inviting them to partake in the riches earned also for them by Christ's atoning work?

``PREACH THE GOSPEL!'' That is the means committed to us for the winning of blood-bought souls, the means by which the grace of God is to be brought to the attention of men and to be made alive in their hearts.

Not the so-called Gospel of social service, of which we hear so much in our days; not the message of present-worldliness, with its cry of: Save the people of this world! It is true that the highest forms of social blessings have come to men with Christianity, and that pracitcally every real advance in the world in the last nineteen centuries proceeded from Christianity or is connected with the Christian religion. But that is the effect of the soul-changing power of the Gospel of Christ, a power which is so great that it influences not only those who actually confess Christianity and live in accordance with its high ideals, but that it exerts a purging and a beautifying impulse also on others who come in contact with its monuments.

The Gospel which we are to bring to men is a power of God unto salvation because it is the message of the free grace and mercy of God in Christ Jesus, the one and only Savior of mankind, who in our stead and for our redemption came down from heaven, was incarnate of the Holy Ghost by the Virgin Mary, and gave His life as a ransom for mankind when He died on the cross. It is the message of the forgiveness of sins for the skae of Jesus that is the essence of the gospel. And it is the Gospel which we are bending our efforts to bring to all men everywhere.

``TO EVERY CREATURE,'' ``TO ALL NATIONS!'' To every member of this lost and condemned mankind! To rich and poor, to old and young, to the socially prominent and to the outcast of human society, to the capitalist and to the workingman -- to \emph{all} men this message is to be brought.

``God will have \emph{all} men to be saved and to come unto the knowledge of the truth.'' 1 Tim. 2, 4.

Having made disciples of men wherever we find them, in palace and in hovel, and having brought them to Christ by the Sacrament of Baptism, we are to extend our initial work by teaching them to observe all things whatsoever He has commanded us. There is no end, no limit, to the possibilities of our evangelistic work on this side of the grave. The more we work, the greater are the possibilities and the greater the opportunities for service.

Do you know what place such soul-winning has in the eyes of God?

The whole machinery of redemption was set in motion by Him because of it. Even in the Old Testament He says, time and again, that He is the Savior of His people.

``God, their Savior, which had done great things in Egypt.'' Ps. 106, 21.

``I am the Lord, thy God, the Holy One of Israel, thy Savior.'' Is 43, 4.

``There is no God else beside Me, a just God and a Savior; there is none beside Me.'' Is. 45, 21.

``All Flesh shall know that I, the Lord, am thy Savior and thy Redeemer, the Mighty One of Jacob.'' Is. 49, 26.

``Thou shalt know that I, the Lord, am thy Savior and thy Redeemer.'' Is. 60, 16.

``For He said, Surely they are My people, children that will not lie: so He was their Savior.'' Is. 63, 8.

And has not the New Testament fully borne out the promise and the prophecy of the Old? Is not the thought of the salvation of makind on the basis of the love and mercy of God the central theme in every book given to men in the New Dispensation?

``My spirit hath rejoiced in God, my Savior.'' Luke 1, 47.

``Paul, an apostle of Jesus Christ by the commandment of God, our Savior, and Lord Jesus Christ, which is our Hope.'' 1 Tim. 1, 1.

``We trust in the living God, who is the Savior of all men.'' 1 Tim. 4, 10.

``God hath in due times manifested His Word through preaching, which is committed unto me according to the commandment of God, our Savior.'' Titus 1, 3.

``That they may adorn the doctrine of God, our Savior, in all things.'' Titus 2, 10.

``To the only wise God, our Savior, be glory and majesty, dominion and power, both now and Forever!'' Jude 25.

Do you want further evidence of the Importance of soul-winning as God sees it? Not only does His own name indicate His desire for the salvation of men, but He also states it in words of unmistakable emphasis.

``Have I any pleasure at all that the wicked should die? saith the Lord God, and not that he should return from his ways and live?'' Ezek. 18, 23.

``God will have all men to be saved, and to come unto the knowledge of truth.'' 1 Tim. 2, 4.

``The Lord is not willing that any should perish, but that all should come to repentance.'' 2 Pet. 3, 9.

Do you still need more information to convince you of the interest that God takes in saving men from their sins? Can there be a greater proof than that He sent His only-begotten Son, Jesus Christ, to accomplish the salvation of all mankind?

``We have seen and do testify that the Father sent the Son to be the Savior of the world.'' 1 John 4, 14.

``When the fulness of the time was come, God sent forth His Son, made of a woman, made under the Law, to redeem them that were under the Law.'' Gal. 4, 4.

``God so loved the world that He gave His only-begotten Son, that whosoever believeth in Him should not perish, but have everlasting life.'' John 3, 16.

Is it a wonder, with such facts from the Bible before them, that some of the foremost workers for Christ felt constrained to go and measure up, in some degree, to the expectatino of God?

It was Carey who declared that we must ``expect great things from God,'' and that we must ``attempt great things for God,'' and who in the strength of Is. 54, 2. 3 set out for India.

It was Allen Gardiner who gave up all prospects of becoming wealthy in order to go to the darkest part of South America, Tierra del Fuego, and to lay down his life with the words before his eyes: ``My soul, wait thou only upon God; for my expectation is from Him.'' Ps. 62, 58.

It was David Livingstone who refused to abandon his task after Stanley had found him, but resolutely sent the younger man home with the precious records of work already accomplished, while he turned back to finish alone his great undertaking.

It was Mary Slessor, of Calabar, who left the very coast which served as a station of communication with far-away England and went into the interior to each more of those people, who reverently called her ``Ma,'' the Gospel of salvation.

It was Theodore Fliedner who did not shrink from a discharged female conflict, but with this woman as the first inmate of his improvised home began the work which resulted in the revival of the female diaconate.

It was John Geddie who went to Aneityum, laboring there with such success that the native Christians themselves said of him, ``When he landed in 1848, there were no Christians here; when he left, in 1872, there were no heathen.''

It was Brainerd who declared, ``I cared not where or how I lived or what hardships I went through, so that I could but gain souls for Christ.''

It was Gregory who stated, ``OF all the sacrifices there is none in the sight of Almighty God equal to zeal for souls.''

~~``Go ye unto ev'ry nation!''\\
\hspace*{0.333em}\hspace*{0.333em}Is the Savior's great command;\\
\hspace*{0.333em}\hspace*{0.333em}``Preach he Gospel of salvation\\
\hspace*{0.333em}\hspace*{0.333em}To all men in ev'ry land;\\
\hspace*{0.333em}\hspace*{0.333em}Teach them all the glorious message\\
\hspace*{0.333em}\hspace*{0.333em}That I died to end all strife\\
\hspace*{0.333em}\hspace*{0.333em}And that death might be the passage\\
\hspace*{0.333em}\hspace*{0.333em}To the blissful endless life.''

~~`Tis by Jesus' love and merit\\
\hspace*{0.333em}\hspace*{0.333em}All men are at peace with God,\\
\hspace*{0.333em}\hspace*{0.333em}Reassured by His free Spirit,\\
\hspace*{0.333em}\hspace*{0.333em}Saved from all their guilty load.\\
\hspace*{0.333em}\hspace*{0.333em}He who trusts in Christ his Savior,\\
\hspace*{0.333em}\hspace*{0.333em}Who for all men did atone,\\
\hspace*{0.333em}\hspace*{0.333em}Will receive the Father's favor,\\
\hspace*{0.333em}\hspace*{0.333em}Will be saved by grace alone.

~~To the nations most enlightened\\
\hspace*{0.333em}\hspace*{0.333em}With this world's progressive lore,\\
\hspace*{0.333em}\hspace*{0.333em}And to those whose souls are frightened,\\
\hspace*{0.333em}\hspace*{0.333em}Bound by superstitious lore;\\
\hspace*{0.333em}\hspace*{0.333em}Those whose god is this world's mammon\\
\hspace*{0.333em}\hspace*{0.333em}And those deep in poverty,\\
\hspace*{0.333em}\hspace*{0.333em}To the rich and the street gamin,\\
\hspace*{0.333em}\hspace*{0.333em}Comes the call to make them free.

~~Let us shout it full of gladness\\
\hspace*{0.333em}\hspace*{0.333em}Wheresoever men we find;\\
\hspace*{0.333em}\hspace*{0.333em}Let us drive away all sadness,\\
\hspace*{0.333em}\hspace*{0.333em}Grief of heart and care of mine;\\
\hspace*{0.333em}\hspace*{0.333em}Let us tell the wondrous story\\
\hspace*{0.333em}\hspace*{0.333em}Of the marvel of God's love,\\
\hspace*{0.333em}\hspace*{0.333em}Let us magnify His Glory\\
\hspace*{0.333em}\hspace*{0.333em}Till the hardest hearts we move;

~~Till all men of ev'ry station\\
\hspace*{0.333em}\hspace*{0.333em}Rich and poor and young and old;\\
\hspace*{0.333em}\hspace*{0.333em}Till all men of ev'ry nation\\
\hspace*{0.333em}\hspace*{0.333em}May be brought into the fold;\\
\hspace*{0.333em}\hspace*{0.333em}Till the Savior's robe of beauty\\
\hspace*{0.333em}\hspace*{0.333em}Covers ev'ry guilty stain;\\
\hspace*{0.333em}\hspace*{0.333em}Till they know their highest duty\\
\hspace*{0.333em}\hspace*{0.333em}Everlasting life to gain.

Would you know the motive which prompts such a response in the hearts of Christians everywhere? -- You will find more on this point in the next chapters.

\hypertarget{i-delight-to-do-thy-will-is.-40-8}{%
\chapter{I Delight to Do Thy Will! Is. 40, 8}\label{i-delight-to-do-thy-will-is.-40-8}}

\hypertarget{the-obligation-of-love.}{%
\section*{The obligation of Love.}\label{the-obligation-of-love.}}
\addcontentsline{toc}{section}{The obligation of Love.}

The divine commission is not an arbitrary command; it is not a legal precept issued by God by virtue of His majesty and power. It is, as a matter of fact, addressed to Christians and would have no meaning for anyone else. Only he can understand this commission and properly act upon it in whose heart the Holy Ghost has already wrought a knowledge of the salvation brought by Christ and revealed in His Word. It is a heart of this kind that is actuated by the obligation of love resting upon it.

And how can it be otherwise, since the Christina continually has before his eyes the wonderful picture of Christ and the manner in which He carried out and satisfied the obligation of love resting upon Him by virtue of His own choice?

For what was the guiding principle of His life and work?

``THEN SAID I, LO, I COME; IN THE VOLUME OF THE BOOK IT IS WRITTEN OF ME. I DELIGHT TO DO THY WILL, O MY GOD.'' Ps. 40, 8.

These are words of the Messiah, as the writer to the Hebrews, chap.~10, 5-7, shows. The Son of God had from eternity taken part in the counsel of God pertaining to fallen mankind, and He had declared His willingness to work the redemption, which none but He could accomplish. This attitude is evident throughout our Savior's life.

``Wist ye not that I \emph{must} be about My Father's business?'' was the half-reproachful question which He addressed to His parents when He was taken to the festival of the Passover at the age of twelve years. Luke 2, 49.

``I \emph{must} walk to-day and to-morrow and the day following.'' Luke 13, 33.

``I \emph{must} work the works of Him that sent Me while it is day; the night cometh when no man can work.'' John 9, 4.

``From that time forth began Jesus to show unto His disciples how that He \emph{must} go unto Jerusalem, and suffer many things of the elders and chief priests and scribes, and be killed, and be raised again the third day.'' Matt. 16, 21.

``For I say unto you that this that is written \emph{must} yet be accomplished in Me, And He was reckoned among the transgressors.'' Luke 22, 37.

``Thinkest thou that I cannot now pray to My Father, and He shall presently give Me more than twleve legions of angels? But how then shall Scriptures be fulfilled that it \emph{must} be?'' Matt. 26, 53-54.

``Remember how He spake unto you when he was yet in Galilee, saying, The Son of Man \emph{must} be delivered into the hands of sinful men and be crucified and the third day rise again.'' Luke 24, 7.

``\emph{Ought} not Christ to havve suffered these things and to enter into His glory?'' Luke 24, 26.

``And He said unto them, Thus it is written, and thus \emph{it behooved Christ} to suffer and to rise from the dead the third day.'' Luke 24, 46.

Thus we find it all the way through the life of Christ, -- the ``must'' of the divine obligation resting upon Him. He has placed Himself at the disposal of God, and in line with His own eternal will, which is at all times in perfect agreement with that of the Father, John 5, 19, He carried out the plan of redemption.

What the German hymn-writer Paul Gerhardt has the Savior say is true: --

~~~Yea, Father, yea most willingly\\
\hspace*{0.333em}\hspace*{0.333em}\hspace*{0.333em}I'll bear what Thou commandest;\\
\hspace*{0.333em}\hspace*{0.333em}\hspace*{0.333em}My will conforms to Thy decree,\\
\hspace*{0.333em}\hspace*{0.333em}\hspace*{0.333em}I do what Thou demandest.--\\
\hspace*{0.333em}\hspace*{0.333em}\hspace*{0.333em}O wondrous Love, what has Thou done!\\
\hspace*{0.333em}\hspace*{0.333em}\hspace*{0.333em}The Father offers up his Son,\\
\hspace*{0.333em}\hspace*{0.333em}\hspace*{0.333em}The Son, content, descendeth!\\
\hspace*{0.333em}\hspace*{0.333em}\hspace*{0.333em}O Love, O Love, how strong art Thou!\\
\hspace*{0.333em}\hspace*{0.333em}\hspace*{0.333em}In shroud and grave Thou lay'st Him low\\
\hspace*{0.333em}\hspace*{0.333em}\hspace*{0.333em}Whose word the mountains rendeth!

Where would we and all mankind be if the Savior had wavered in His divine determination, if He had faltered and shrunk at sight of the cross on which His tortured body was to be suspended?! What an immeasurable burden of gratitude is laid upon us by virtue of His unflinching persistence in the obedience prompted by His redemptive love!

Are you looking for still further evidence regarding the position which soul-winning has in the mind of Christ, the one and only Savior of mankind? Consider the place it has in His life and work. Remember that His very name indicates the purpose of His life and work; for Jesus means ``Redeemer, Savior.'' Matt. 1, 21.

It is the name given to our Lord throughout the New Testament; it is used by the inspired writers with an evident feeling of exultation. The very angel of the Lord speaks it with a hushed reverence when he announces the birth of the Lord: --

``Unto you is born this day in the city of David a Savior, which is Christ the Lord.'' Luke 2, 11.

It is found in the joyful testimony of the Samaritans of Sychar: --

``Now we believe, not because of thy saving; for we have heard Him ourselves and know that this is indeed the Christ, the Savior of the world.'' John 4, 42.

And think of the numerous other passages in which the name is blazoned as on a banner to be borne before the eyes of the believers, to make them realize ever more fully the unspeakable gift of God! Read them for yourself: Acts 5, 31; 13, 23; Phil. 3, 20; 2 Tim. 1, 10; Titus 1, 4; 2, 13; 3, 6; 2 Pet. 1, 11; 2, 20; 3, 2. 18; 1 John 4, 14.

What the name of Jesus indicates, what the angel's explanation proclaims, that is emphasized in Christ's earthly mission. No one has said it better, no one could express it more definitely than the Lord Himself when He says: --

``The Son of Man is come to save that which was lost.'' Matt. 18, 11. And again: --

``The Son of Man is come to seek and to save that which was lost.'' Luke 19, 10.

This is also the clear statement of that ``Gospel in a nutshell,'' given in Christ's own words: --

``God so loved the world that He gave His only-begotten Son, that whosoever believeth in Him should not perish, but have everlasting life.'' John 3, 16.

``These things I say that ye might be saved.'' John 5, 34.

``I am the Door; by Me, if any man enter in, He shall be saved.'' John 10, 9.

``This is the will of Him that sent Me, that every one which seeth the Son and believeth on Him may have everlasting life, and I will raise him up at the last day.'' John 6, 40.

In the very performance of His miracles our Lord's chief gift was that of the forgiveness of sins with its assurance of salvation. To the man sick of the palsy He gave, first of all, that wonderful certainty: --

``Son, be of good cheer; thy sins be forgiven thee.'' Matt. 9, 2; Luke 5, 20.

And when the great sinner knelt at His feet in the house of the Pharisee, the most outstanding gift of Christ is that which He Himself indicates: --

``Wherefore I say unto thee, Her sins, which are many, are forgiven.'' Luke 7, 47.

That this winning of souls for the kingdom of God was the object of Christ in all His preaching, in all His work, is obvious from the general tone and tendency of all His acts and all of His precepts. He tells the former demoniac to preach the kingdom of God. He summarizes His own invitation in the words: ``Go out quickly into the streets and lanes, highways and hedges, and compel them to come in.'' As Dr.~Pierson says: ``The command is one which is incarnated in His whole life and is suggested or implied in the very idea of discipleship: `Follow Me, and I will make you fishers of men.'\,''

Do we need further evidence to convince us that the obligation of love was the guiding principle of the Savior's life and that the importance of soul-winning in His work is the outstanding feature of the entire Gospel? If nothing else will impress us, we cannot deny the witness of His death upon the cross. He Himself says of it: --

``And I, if I be lifted up from the earth, will draw all men unto Me.'' John 12, 32.

Read the account of the gospels, the description of the Savior's crucifixion and of His death on Calvary. Cp. Luke 23, 32-43.

The matter is most beautifully put by St.~Paul when he writes: --

``The life which I now live in the flesh I live by the faith of the Son of God, who loved me and gave Himself for me.'' Gal. 2, 20.

``Who gave Himself for us that He might redeem us from all iniquity and purify unto Himself a peculiar people.'' Titus 2, 14.

Truly, it is a remarkable topic, and one which should duly impress us with the unbounded glory of the love of Jesus in His vicarious redemption and with the fulness of the love which could cause the great Son of God to humble Himself for our sakes.

\begin{center}\rule{0.5\linewidth}{0.5pt}\end{center}

But now comes the test for every one of us. As St.~Paul puts it: --

``Let this mind be in you which was also in Christ Jesus, who, being in the form of God, thought it not robbery to be equal with God, but made Himself of no reputation and took upon Him the form of a servant and was made in the likeness of men; and, being found in fashion as a man, He humbled Himself and became obedient unto death, even the death of the cross.'' Phil. 2, 5-8.

The mind of Christ was that according to which He felt the obligation laid upon Him by His Father's love and His own; it was the mind which caused Him to be the great Servant of mankind in order to show them the way of salvation. Jesus Himself calls our attention to this phase of His work: --

``Whosoever will be great among you, let him be your minister; and whosoever will be chief among you, let him be your servant; even as the Son of Man came, not to be ministered unto, but to minister and give His life a ransom for many.'' Matt. 20, 26-28.

The obligation of love which rested upon Jesus has passed onto us, who bear His name and are filled with His spirit. The wonderful union which has been established between Christ and us by virtue of the faith that lives in us has given us some of His power. Since Christ has made his abode in us, together with the Father and the Holy Ghost, we are in a position to bear much fruit of the kind which He inspires and loves. We are now, as St.~Paul writes, His workmanship, created in Christ Jesus unto good works, which God hath before ordained that we should wlak in them. Eph. 2, 10.

In accordance with these facts there is one great motto which Christians love to keep before their eyes at all times, namely: --

``The love of Christ constraineth us.'' 2 Cor. 5, 14.

Obviously this is not the constraint of the Law and of fear; for ``perfect love casteth out fear.'' It is the urgency and the power of the love which we have received in Christ, as an outflow of the divine power in Christ, and it is the zeal which now impels us forward for love of Christ, in appreciation of the boundless mercy which we have received.

Is it necessary to emphasize this point any further? Is the obligation of love brought to our attention to-day and with reference to the situation as we have it before our eyes in the world? Have we a responsibility which we ought to feel with at least a small fraction of the fervor and zeal shown to us by Christ?

Oh, the need of the world for the love which we alone can bring to men by virtue of the Gospel entrusted to us is still immensely, overwhelmingly great. It is not only that men are without Christ, in a kind of a neutral situation, but it is that millions of them are living in open and shameful opposition to Him, children of wrath and heirs of eternal damnation.

Here are some of the facts as they are accessible to us to-day with regard to the WORLD WITHOUT CHRIST!

According to the latest available statistics the population of India is 320,000,000. Now, if we figure all the Protestant societies that are now working in that country of teeming millions (and that includes not a few whose Christianity is of the very liberal kind, not much better than the religion of the heathen themselves), we have far fewer than a million baptized Christians (849,500). Even if we count all those who are members of the Roman Catholic and of the Syrian churches, we have barely five million Christians! Barely one and one half percent of the total population -- and the gains that are being made are so heart-breakingly small! Does our obligation extend to India?

The situation in Southeastern Asia, including Assam, Burma, Siam, and the Malay Peninsula, French Indo-China, that is, all countries east of India and south of China, is as follows. The population, all told, is somewhat over 53,000,000. In this great mass there are fewer than 100,000 Christians, and some sections may be said to be altogether unoccupied as yet. Not even one-fifth of one per cent. won for Christ!

Next comes the immense country of China, with its more than 3,200,000 square miles and its population of 440,000,000. Do you know that here, ALL TOLD, the number fo communicant Christians has not yet reached the 400,000 mark, although 174 societies are now at work? The fraction is so infinitesimally small that one hesitates to write it. Entire provinces are still without so much as one messenger of salvation!

Japan's population exceeds 60,000,000, and we have read so much about Christian leaders in the island empire that we have probably oerestimated the number of Christians. As a matter of fact, the latest statistics give the number of communicant members of all Protestant missions as not quite 200,000. Again a number which is quite disheartening in its smallness!

As we go over to Korea, which has had intercourse with the Western World for a matter of only a few years, we find a population of 17,000,000 under Japanese rule. Although there are many factors in this country which have been found favorable to mission-work, yet the number of Protestant Christians is below 100,000, or not yet one half of one per cent.

As we next look at the Near East, comprising Egypt, Asia Minor (with Armenia and Kurdistan), Syria, Palestine, Arabia, Mesopotamis, and Persia, the situation is still more depressing. The total population of this section of the world is estimated at almost 55,000,000. We have here the location of the cradle of the human race, the site of the world's greatest ancient empires, the land of the Bible and of the Savior. We still have remnants of the Armenian Church, nominally Christian, in Armenia, there are many sects of the Greek Orthodox Church and one or two of the Roman Catholic Church in this section, not to speak of the Coptic Church in Egypt; but the number of Christians is at best very small, and the number of Protestants is as yet below 20,000.

Next we consider Africa, the ``Dark Continent.'' Its native population is estimated, with some degree of probability, as reaching 150,000,000. In this entire number there are only three million Protestant Christians, ans possibly seven million more, who are nominally members of the Abyssinian, Coptic, and Roman Catholic churches. Again the discrepancy is so great that it is appalling.

Latin America includes Central and South America, with a total of 85,000,000. Til now hardly more than a beginning has been made in bringing the Gospel to this mixed population; for, although almost all the countries concerned are nominally Roman Catholic, yet the number of professed Christians amounts to only a very small percentage of the total, since the workers, all told, amount to barely 2,500. A moment's reflection will show the utter inadequacy of the present missionary occupation.

Therea re a few spot sin Oceania, or in the islands of the Pacific, which offer a distinct relief. We are here dealing with Malaysia, Melanesia, Micronesia, and Polynesia, whose combined population is more than 60,000,000. A few islands are entirely Christianized, but over ninety-five per cent. of the territory is still without the Gospel-message, some sections having not even been touched.

And what shall we say of the unoccupied fields in many parts of the world, which stand as a constant challene before the eyes of Christianity? Is it the ``regions beyond'' that offer the most serious problems at the present time, because circumstances have here combined to keep out the name and the Word of Christ. There is the heart of Asia, with Mongolia, Chinese Turkistan, Tibet, Afghanistan, and Baluchistan; there is the interior of Africa, with almost fifty pagan tribes; there is the heart of South America, many parts of which are not even explored.

That is the challenge to Christianity, that is our obligation of love!

Nor have we as yet mentioned the field wich is both a problem and the most emphatic challenge, at our very doors. Even if we count all those who are only nominally members of Christian churches in our country, we have

\begin{center} BETWEEN 60 AND 65 PER CENT. OF OUR TOTAL POPULATION NOT WITHIN THE CHURCH \end{center}

Think of it: some 65,000,000 of our fellow-citizens in this country have not yet accepted the Gospel of Jesus Christ unto their salvation, and, stranger still, many of these have not even heard of their Savior. In the midst of a so-called Christian civilization, people have never been approached with a view of making them acquainted with the great truths which will bring redemption also to them, the justification which is ready for them in the perfect atonement of Jesus Christ.

And the matter is of unusual interest to

\begin{center} US LUTHERANS! \end{center}

Due partly to the need of gathering those who applied to us for spiritual care during the great immigration from Lutheran and semi-Lutheran countries, partly to the unfortunate language question, we have not yet reached out to our fellow-citizens as opportunity offered. And what is more, a conservative estimate tells us that

\begin{center} ABOUT TEN MILLION PEOPLE OF LUTHERAN EXTRACTION IN THIS COUNTRY ARE NOT CONNECTED WITH THE LUTHERAN CHURCH! \end{center}

So many reasons have been advanced for this condition. But, whatever the reason, these souls are a constant challenge to us, they present to us the

\begin{center} OPPORTUNITY AND THE OBLIGATION OF LOVE! \end{center}

It is because personal work has been so largely neglected in our midst that the deficit in souls is so great against our Church. There can be no question concerning the fact that, in addition to the public proclamation of the Gospel, words for christ to the individual are most effective in the winning of sould. A kind, but earnest word to a negligent churchgoer of our own confirmation class, a tactful invitation to a neighbor, a letter confessing Christ in a frank manner -- these are the things that count with the individual and often serve as entering-wedges for the Word of salvation.

``It is the man-to-man work that tells. And because it is this work that is most effective, this is the work that is best to do. Even though it is less attractive work, as we look at it, and seems to others less important to be done, we must admit that the results are worth considering. As John B Gough said of the one loving word of Joel Stratton that won him: `My friend, it may be a small matter for you to speak the one word for Christ that wins a needy soul, -- a \emph{small matter to you}, but it is \emph{everything to him}.' It is forgetting this truth that causes personal work to be neglected.'' (\emph{Trumbull}.)

It was the greatest missionary of all times that said, as he summarized the devotion of a lifetime in one sentence: --

``I AM A DEBTOR both to the Greeks and to the barbarians; both to the wise and to the unwise.'' Rom. 1, 14.

If we realize the obligation of love resting upon us,

\begin{center} WE ARE DEBTORS! \end{center}

\hypertarget{workers-together-with-him-2-cor.-6-1}{%
\chapter{Workers Together with Him! 2 Cor. 6, 1}\label{workers-together-with-him-2-cor.-6-1}}

\hypertarget{the-biblical-precept-and-example.}{%
\section*{The Biblical Precept and Example.}\label{the-biblical-precept-and-example.}}
\addcontentsline{toc}{section}{The Biblical Precept and Example.}

It is a wonderful name: ``WORKERS TOGETHER WITH GOD!'' -- a name of rich content, a name which bestows a world of honor upon us.

The very expression ``together with us'' is full of significance and power. It reminds us of so many other gifts and blessings of God, especially of those which were so richly imparted to us in Christ Jesus.

We are \emph{heirs together with Christ}, as St.~Paul so beautifully states: --

``The Spirit itself beareth witness with our spirit that we are the children of God; and if children, then heirs; heirs of God and \emph{joint heirs with christ}.'' Rom. 8, 16. 17.

And it is particularly comforting to us, who are descendants of heathen, that St.~Paul writes: --

``That the gentiles should be fellow-heirs and of the same body and partakers of His promise in Christ by the Gospel.'' Eph. 3, 6.

We are \emph{partakers together of the life in Christ}, which will have its culmination in the enjoyment of the glory of heaven. The apostle states: --

``It is a faithful saying: For if we be dead with Him, we shall also be live with Him; if we suffer, we shall also reign with Him.'' 2 Tim. 2, 11. 12.

The honor which has thereby been bestowed on the human race can hardly be estimated highly enough, for it is one point of evidence showing the greatness of God's mercy toward us.

The very inspired writers marvel at some of the facts connected with the history of man's redemption. In the mystery of the incarnation, for instance, one might well wonder why the Lord idd not appear in the form of an angel to bring redemption to men. But we are told: --

``Verily He took not on Him the nature of angels, but He took on Him the seed of Abraham.'' Heb. 2, 16.

Well may we sing in the glorious Christmastide:

~~Th' eternal Father's only son\\
\hspace*{0.333em}\hspace*{0.333em}For a manger leaves His throne;\\
\hspace*{0.333em}\hspace*{0.333em}Disguised in our poor flesh and blood\\
\hspace*{0.333em}\hspace*{0.333em}Is now the everlasting Good.

The mystery of the incarnation of our Lord, as the first step in the perfected redemption, is so great that the ``angels desire to look into'' the marvelous facts connected therewith. 1 Pet. 1, 12.

It is true, moreover, that the Lord uses the holy angels as His messengers. Thus we find that the angel Gabriel was at various times sent to Daniel, particularly to strengthen and comfort him on account of the visions which were given to him. The same angel was sent also to Mary and to Zacharias.

But angels are not honored with the name of WORKERS TOGETHER WITH GOD. While an angel brought the news of the birth of the Savior to the shepherds on the fields of Bethlehem, and while it was a chorus of angels that first sang an anthem of praise in glorifying God for this holy birth, it is true, nevertheless, that angels were not entrusted with the divine commission, but this \emph{distinctino was given to human beings}.

NOT ANGELS, BUT MEN are chosen by God to preach the Gospel to every creature; upon MEN is placed the obligation of love. Those whose brother the Savior became by His sacred incarnation are to make known to all members of the human family the news of the redemption wrought by their Brother.

With the consciousness of this distinction, of this unequaled honor, we can understand the precepts of the Lord. For it is not only in the divine commission itself that He speaks to us concerning the need of bringing the message of salvation to others, but also in many other passages, whose import and significance should be considered by us with the most assiduous attention.

Even in the Old Testament we find the Lord calling out to us in an excess of jubilation: --

``Say among the heathen that the Lord reigneth; the world also shall be established that it shall not be moved. He shall judge the people righteously.'' Ps. 96, 10.

And we may well consider, in this connection, passages like Ps. 117, 1; Is. 34, 1; Jer. 4, 2.

But it is in the New Testament that this feature of bringing to others the assurance of the recemption gained by Christ is particularly prominent. Who could forget the words addressed by Christ to the healed and grateful demoniac: --

``Return to thine own house and show how great things God hath done unto thee''? Luke 8, 39.

In this case it required no second urging, for we are told that ``he went his way and published throughout the whole city how great things Jesus had done unto him.''

Can we afford to do less with the fulness of God's spiritual blessings resting upon us?

The words of St.~Paul to the Galatians are well known, but they will bear repetition: --

``Let us not be weary in well-doing; for in due season we shall reap, if we faint not. As we have therefore opportunity, let us do good unto all men, especially unto them who are of the household of faith.'' Gal. 6, 9. 10.

Can we pass on greater blessings than the forgiveness of sins, peace with God, the happiness of a good conscience, which are ours in the Gospel?

\hypertarget{zealously-affected-in-a-good-thing-gal.-4-18}{%
\chapter{Zealously Affected in a Good Thing! Gal. 4, 18}\label{zealously-affected-in-a-good-thing-gal.-4-18}}

\hypertarget{qualifications-of-the-workers.}{%
\section*{Qualifications of the Workers.}\label{qualifications-of-the-workers.}}
\addcontentsline{toc}{section}{Qualifications of the Workers.}

The Scripture-passage which we have at the head of the present chapter is peculiarly appropriate to our discussion. The Apostle Paul was not in sympathy with an attitude which is always ready to receive, the plea being that faith must be disassociated from works. It is true that saving faith in its essence is the receiving of the grace of God in the Gospel.

But saving faith is, nevertheless, a living faith. It is a light which not only receives fuel, but wich also shines. The apostle fittingly calls it ``faith \emph{which worketh by love},'' Gal. 5, 6, that is, a faith which is active in love, which shows itself in works of love.

\hypertarget{the-time-is-short.-1-cor.-7-29}{%
\chapter{The Time is Short. 1 Cor. 7, 29}\label{the-time-is-short.-1-cor.-7-29}}

\hypertarget{salesmanship-for-the-lord.}{%
\section*{Salesmanship for the Lord.}\label{salesmanship-for-the-lord.}}
\addcontentsline{toc}{section}{Salesmanship for the Lord.}

The qualifications which we considered in the last chapter make for a strong Christian personality. The possession of any one of them or of a group of them is a fine asset to the soul-winner. But he must not be satisfied with having gained some measure of ability along one line. It is necessary for him to build up for higher efficiency. Let us group some of the positive qualities which come into consideration in working for the Lord.

Let us list some of the qualifications of the intellect, of the sensibilities, of the will, and of the spirit, or heart, as they must be highly developed and as they must be kept in the highest possible state of efficiency.

\hypertarget{because-of-his-importunity.-luke-11-8}{%
\chapter{Because of His Importunity. Luke 11, 8}\label{because-of-his-importunity.-luke-11-8}}

\hypertarget{the-need-and-power-of-prayer.}{%
\section*{The Need and Power of Prayer.}\label{the-need-and-power-of-prayer.}}
\addcontentsline{toc}{section}{The Need and Power of Prayer.}

One of the most significant, illuminating, and stimulating facts about Jesus is that which tells us that the Savior made \emph{prayer a habit}. This is all the more remarkable if we consider that this habit on the part of the Lord is reported in such a matter-of-fact way, without the slightest indication of a false enthusiasm in the incidents or in their recital, that the impressin of the account is thereby heightened. We simply find a wonderful intimacy and fellowship existing between Jesus and His heavenly Father, which found its expression in the act of prayer, not merely as a devotional exercise, but as a form of communication by and through which He derived the support and the strength which He needed for His work.

In the very first months of His public ministry in Galilee, shortly after He had made Capernaum His headquarters, Jesus,

\hypertarget{i-am-persuaded.-rom.-8-38}{%
\chapter{I Am Persuaded. Rom. 8, 38}\label{i-am-persuaded.-rom.-8-38}}

\hypertarget{having-the-courage-of-ones-convictions.}{%
\section*{Having the Courage of One's Convictions.}\label{having-the-courage-of-ones-convictions.}}
\addcontentsline{toc}{section}{Having the Courage of One's Convictions.}

The discussion of the present chapter is essential for the purpose of this study. It speaks of a most important part of the personal equipment of the worker for Christ. Without the qualification which it imples much of the testifying for Chirist is of an indifferent, mechanical kind, without the force that, in itself, carries the certainty of conviction.

Luther was wont to refer to a man who was not at all times ready to stand up for his convictions as a ``soft-stepper.'' He did not mean to question the sincerity of any one, but he felt that some of the fundamental principles of the truth were occasionally sacrificed on the altar of what men would like to describe as tact, but which often has its roots in a timidity not at all in keeping with the high ideals held out by the Word of God.

\hypertarget{by-all-means-save-some-1-cor-9-22.}{%
\chapter{By All Means Save Some! 1 Cor, 9, 22.}\label{by-all-means-save-some-1-cor-9-22.}}

\hypertarget{meeting-the-unchurched.}{%
\section*{Meeting the Unchurched.}\label{meeting-the-unchurched.}}
\addcontentsline{toc}{section}{Meeting the Unchurched.}

BETWEEN SIXTY AND SIXTY-FIVE PERCENT. OF OUR TOTAL POPULATION HAS NO CHURCH AFFILIATION!

Read that sentence again, for we shall consider it once more in connection with the object of the present chapter. It is a fact which, somehow, must sink into our consciousness by degrees. It means that an average of six or seven out of every ten persons whom we see on the street, whom we meet on our travels, whith whom we deal in a business way, ARE NOT EVEN NOMINALLY CHRISTIANS!

\hypertarget{patient-toward-all-men.-1-thess.-5-14.}{%
\chapter{Patient Toward All Men. 1 Thess. 5, 14.}\label{patient-toward-all-men.-1-thess.-5-14.}}

\hypertarget{meeting-objections-of-the-wrongly-informed.}{%
\section*{Meeting Objections of the Wrongly Informed.}\label{meeting-objections-of-the-wrongly-informed.}}
\addcontentsline{toc}{section}{Meeting Objections of the Wrongly Informed.}

As WORKERS TOGETHER WITH GOD much of our work will naturally concern the unchurched; in fact, this is the only part of our work in which we can be aggressive, in which we can and should take the initiative.

Lutheran soul-winners are not proselyters. In all our work we follow the admonition of the apostle:

``LET NONE of you suffer\ldots as a BUSY-BODY in other men's matters.'' 1 Pet. 4, 15.

\hypertarget{i-will-seek-that-which-was-lost.-ezek.-34-16.}{%
\chapter{I Will Seek That Which was Lost. Ezek. 34, 16.}\label{i-will-seek-that-which-was-lost.-ezek.-34-16.}}

\hypertarget{canvassing.}{%
\section*{Canvassing.}\label{canvassing.}}
\addcontentsline{toc}{section}{Canvassing.}

We have now come to the point where the practical execution of the plan is the primary consideration. It is understood, of course, that the Lutheran soul-winner does not confine his efforts to any one day in the year, that he is not satisfied with one particular occasion for doing the greatest good. Our aim is to do good to all men, to try to interest them in their soul's salvation at all times, to keep the possibilities of the message of redemption in view whenever occasion offers

At the same time, experience has shown that great, united, systematic mission endeavors are productive of much good. It is self-evident, in the case of Lutherans, that the emotional element must not become too prominent. Informatino concerning mission-work, concerning the will of God pertaining to our sanctification, a thorough knowledge of the needs of men and of the way to help them in their spiritual need is essential to our work. Emotionalism alone is like a straw-fire, which quickly burns out and therefore is without lasting effects. The fire which we aim to kindle by our missionary endeavors is intended to set fire to heart and conscience, to mind and soul, for we want men to accept the message of the recemption of their souls through the atonement wrought by Christ.

\hypertarget{let-us-not-be-weary.-gal.-6-9}{%
\chapter{Let Us Not Be Weary. Gal. 6, 9}\label{let-us-not-be-weary.-gal.-6-9}}

\hypertarget{follow-up-work.}{%
\section*{Follow-Up Work.}\label{follow-up-work.}}
\addcontentsline{toc}{section}{Follow-Up Work.}

There is a reason for selecting the heading of this chapterin just that form: --

LET US NOT BE WEARY!

\hypertarget{feed-my-lambs-john-21-15}{%
\chapter{Feed My Lambs! John 21, 15}\label{feed-my-lambs-john-21-15}}

\hypertarget{founding-and-conducting-a-sunday-school.}{%
\section*{Founding and Conducting a Sunday-School.}\label{founding-and-conducting-a-sunday-school.}}
\addcontentsline{toc}{section}{Founding and Conducting a Sunday-School.}

The subject of this chapter is very closely connected with that of all personal endeavor in mission work. It links up with the historical fact that the laymen of the early Church were actively engaged in spreading the Gospel, as we learned in Chapter 3, and that the Lord expects all Christians, whether pastors or laymen, to take a direct, personal, active interest in the spread of His kingdom.

The matter is brought home to us even by a consideration of some historical facts in church history. The Methodist Church did not come into existence until a century after the establishment of the first Lutheran congregation in America. And yet, this denomination has more than twice as many members in our country as all Lutheran bodies put together. The Baptists began work in our country in 1636, or about the same time that Lutheran preaching was established on the Delaware. Yet the Baptists, too, are much stronger than the combined Lutheran bodies of our country.

\end{document}
