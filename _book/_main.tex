% Options for packages loaded elsewhere
\PassOptionsToPackage{unicode}{hyperref}
\PassOptionsToPackage{hyphens}{url}
%
\documentclass[
]{book}
\usepackage{amsmath,amssymb}
\usepackage{iftex}
\ifPDFTeX
  \usepackage[T1]{fontenc}
  \usepackage[utf8]{inputenc}
  \usepackage{textcomp} % provide euro and other symbols
\else % if luatex or xetex
  \usepackage{unicode-math} % this also loads fontspec
  \defaultfontfeatures{Scale=MatchLowercase}
  \defaultfontfeatures[\rmfamily]{Ligatures=TeX,Scale=1}
\fi
\usepackage{lmodern}
\ifPDFTeX\else
  % xetex/luatex font selection
\fi
% Use upquote if available, for straight quotes in verbatim environments
\IfFileExists{upquote.sty}{\usepackage{upquote}}{}
\IfFileExists{microtype.sty}{% use microtype if available
  \usepackage[]{microtype}
  \UseMicrotypeSet[protrusion]{basicmath} % disable protrusion for tt fonts
}{}
\makeatletter
\@ifundefined{KOMAClassName}{% if non-KOMA class
  \IfFileExists{parskip.sty}{%
    \usepackage{parskip}
  }{% else
    \setlength{\parindent}{0pt}
    \setlength{\parskip}{6pt plus 2pt minus 1pt}}
}{% if KOMA class
  \KOMAoptions{parskip=half}}
\makeatother
\usepackage{xcolor}
\usepackage{longtable,booktabs,array}
\usepackage{calc} % for calculating minipage widths
% Correct order of tables after \paragraph or \subparagraph
\usepackage{etoolbox}
\makeatletter
\patchcmd\longtable{\par}{\if@noskipsec\mbox{}\fi\par}{}{}
\makeatother
% Allow footnotes in longtable head/foot
\IfFileExists{footnotehyper.sty}{\usepackage{footnotehyper}}{\usepackage{footnote}}
\makesavenoteenv{longtable}
\usepackage{graphicx}
\makeatletter
\def\maxwidth{\ifdim\Gin@nat@width>\linewidth\linewidth\else\Gin@nat@width\fi}
\def\maxheight{\ifdim\Gin@nat@height>\textheight\textheight\else\Gin@nat@height\fi}
\makeatother
% Scale images if necessary, so that they will not overflow the page
% margins by default, and it is still possible to overwrite the defaults
% using explicit options in \includegraphics[width, height, ...]{}
\setkeys{Gin}{width=\maxwidth,height=\maxheight,keepaspectratio}
% Set default figure placement to htbp
\makeatletter
\def\fps@figure{htbp}
\makeatother
\setlength{\emergencystretch}{3em} % prevent overfull lines
\providecommand{\tightlist}{%
  \setlength{\itemsep}{0pt}\setlength{\parskip}{0pt}}
\setcounter{secnumdepth}{5}
\usepackage{gentium}
\ifLuaTeX
  \usepackage{selnolig}  % disable illegal ligatures
\fi
\IfFileExists{bookmark.sty}{\usepackage{bookmark}}{\usepackage{hyperref}}
\IfFileExists{xurl.sty}{\usepackage{xurl}}{} % add URL line breaks if available
\urlstyle{same}
\hypersetup{
  pdftitle={While It Is Day!},
  pdfauthor={Paul E. Kretzmann},
  hidelinks,
  pdfcreator={LaTeX via pandoc}}

\title{While It Is Day!}
\usepackage{etoolbox}
\makeatletter
\providecommand{\subtitle}[1]{% add subtitle to \maketitle
  \apptocmd{\@title}{\par {\large #1 \par}}{}{}
}
\makeatother
\subtitle{A Manual for Soul-Winners}
\author{Paul E. Kretzmann}
\date{1926}

\begin{document}
\maketitle

{
\setcounter{tocdepth}{1}
\tableofcontents
}
\setlength\parindent{1em}

\chapter*{While It Is Day. John 9, 4}\label{while-it-is-day.-john-9-4}

~~Let us work while it is day,\\
\hspace*{0.333em}\hspace*{0.333em}Let us labor while we may!\\
\hspace*{0.333em}\hspace*{0.333em}Soon will come the night of eath\\
\hspace*{0.333em}\hspace*{0.333em}When we yield our final breath.\\
\hspace*{0.333em}\hspace*{0.333em}Let us work for Chirst, the Lord,\\
\hspace*{0.333em}\hspace*{0.333em}Bear aloft His mighty Word!\\
\strut \\
\hspace*{0.333em}\hspace*{0.333em}Let us work without restraint,\\
\hspace*{0.333em}\hspace*{0.333em}Never weary, lax or faint;\\
\hspace*{0.333em}\hspace*{0.333em}Firm in battling selfishness,\\
\hspace*{0.333em}\hspace*{0.333em}Loyal in true faithfulness,\\
\hspace*{0.333em}\hspace*{0.333em}Place our strength at His command,\\
\hspace*{0.333em}\hspace*{0.333em}Do His work throughout the land!\\
\strut \\
\hspace*{0.333em}\hspace*{0.333em}Let us work while yet we may!\\
\hspace*{0.333em}\hspace*{0.333em}Soon will come that glorious day\\
\hspace*{0.333em}\hspace*{0.333em}When our labor here si done\\
\hspace*{0.333em}\hspace*{0.333em}When the precious prize is won,\\
\hspace*{0.333em}\hspace*{0.333em}When we rest and take our ease\\
\hspace*{0.333em}\hspace*{0.333em}In the homes of endless peace!\\
\strut \\
\hspace*{0.333em}\hspace*{0.333em}(\emph{Knowing and Doing}, p.82.)

\chapter*{Publisher's Note}\label{publishers-note}
\addcontentsline{toc}{chapter}{Publisher's Note}

I was already familiar with Paul Kretzmann's \emph{Popular Commentary} when I first heard about this soul-winning manual on an episode of the \emph{A Word Fitly Spoken} podcast with Rev.~Willie Grills and Rev.~Zelwyn Heide. As a convert to Lutheranism I wondered what a Lutheran soul-winning manual might be like, and set off to find a copy. In the end I could only find a single copy available from any source, in this case a seller on Amazon, whose copy had at some point been in the Emmaus Bible College Library.

I enjoyed the book and wanted to ensure that others who wanted to read it would have an easier time getting a copy for themselves, so I decided to republish the book. This also gave me a practical application for some technical skills I had been developing. I transcribed the text using the Vim text edior, formatted everything with R Markdown and some \LaTeX, and compiled everything with the Bookdown R package, all on computers running Arch and Fedora Linux. I spent 8 months working on this project here and there in my spare time.

Some of the information in the book, like population figures and country names, is outdated due to the 1926 original publication date. Many of the suggested methods and procedures could (and likely would) be enhanced by technologies developed in the subsequent years for any soul-winners seeking to implement Kretzmann's suggestions.

I hope you, the reader, find this work both enjoyable and inspirational, and try your hand at ``Salesmanship for the Lord.''

\hfill\break
Russell Hollander\\
Trimble, MO\\
2023

\chapter*{Foreword}\label{foreword}
\addcontentsline{toc}{chapter}{Foreword}

In these days of spiritual and moral decay the question is often asked, ``What's wrong with the church?'' And this question, while often presented by those who have never given the Church a fair trial, is, in a general way, warranted; for there is no institution in the entire world that can do so much good and has such high and holy responsibilities as the Church. Where else shall the world turn, which is so torn by doubt and fear? Where else, if not to the Church with its saving Gospel, the only message of hope? Yes, the world is sick, and the Church has the only remedy that can cure its ills.

And in these days, when the Battle between Christianity and the world is becoming harder and harder and the line of demarcation separating the two is becoming more and more indistinct; when worldliness is making such terrible inroads in the Church; when love is becoming cold and indifference and laxity are paralyzing Christian activities; when, as the Savior predicted, there are all the indications that the world's destruction is near, -- in this our evil time, what does the Church itself need more than a deeper appreciation for the Gospel and a reconsecration to the spreading of this Gospel? The need in our Church is not, primarily, more men and means, nor more churches and members, but congregations filled with a greater soul-winning spirit, members with a deeper passion for blood-bought souls, members, we repeat, who realize that an actually overwhelming responsibility rests upon them and that their sacred and supreme duty in this world is to testify for Christ by word and deed and thus to help save souls.

While we have heard it stated again and again -- and it is also brought out in this volume -- that two-thirds of the world, or nearly a thousand million souls, do not know Jesus, the Savior, we fail to work as determinedly and self-sacrificingly as we should to bring our fellow-sinners the tidings of salvation. The great commission, ``Go ye into all the world and preach the Gospel to every creature,'' often remains unfulfilled. yet all our church- and school-work has this one object in view, this goal -- to help save souls.

Many and ever more soul-winners -- that is the need of the hour! When our members have caught the soul-saving spirit, and when their hearts are set on fire to save men, greater advancements will be made in our mission-fields here and abroad. There will be a larger outpouring of gifts, a self-sacrificing stewardship. Our members will not be satisfied with giving merely part of their time, spare change, and to make half-hearted efforts, but they will give the best of their time, the best of their money, and put their whole heart and soul into the great work. And there will be a solution to so many discouraging church problems. The Church will grow inwardly and outwardly as never before.

Is it necessary to point out that our young people, the future leaders of the Church, should be in the very front ranks of the Church's missionary crusade? Indeed, young and old should consider it the highest privilege to be colaborers with God in the work of salvation of men through Jesus Christ. But our youth must lead. In this spirit the leaders of the Walther League at the Detroit Convention proposed the so-called systematic mission endeavor. This is merely a united and organized effort of the young people, under the direction of their pastors, to lead others to the Savior and His Church. The plan was most whole-heartedly endorsed by succeeding conventions. The endeavor, which is gaining in favor, has already wrought untold blessings for the Church in deepening the spirituality of the young people.

Realizing that the young people, while taking an active part, can only assist the pastors and congregations, the scope of \emph{While It Is Day!} was widened to be of direct use to all congregations of the entire Church and for all individual members, young and old. It covers all practical phases; and where detailed information is not given, there are enough suggestions to help any judicious leader. \emph{While It Is Day!} is also to serve as a text-book, and it is suggested that the various studies be taken up systematically in regular classes during six-to-eight-week periods. The spirit of a united mission endeavor is well expressed in the title \emph{While It Is Day!} There is no time to be lost. Every hour and every minute men are perishing without the saving Gospel. ``The night cometh when no man can work''; so let us labor ``while it is day.''

P.G. Prokopy

\chapter{Go Ye!}\label{go-ye}

\section*{Matt. 28, 19}\label{matt.-28-19}

\subsection*{The Divine Commission.}\label{the-divine-commission.}

It is God's perogative to have people come to Him, and His invitation is extended to all men in a serious, efficacious call.

``Come unto Me, all ye that labor and are heavy laden, and I will give you rest!'' Matt. 11, 28.

``Look unto Me and be ye saved, all the ends of the earth.'' Is. 45, 22.

``Come, for all things are now ready.'' Luke 14, 17.

``And the Spirit and the bride say, Come! And let him that heareth say, Come! And let him that is athirst come. And whosoever will, let him take the water of life freely.'' Rev.~22, 17.

But to make known this glorious intention and invitation to men \emph{God has commissioned His children}, the believers, throughout the world. They are to be His representatives; they are to be His messengers, His ambassadors; they are to be His agents in making known His call of salvation, in inviting men to the feast of His love and grace.

Even in Old Testament times this was true. To His Zion, to the members of His Church under the Old Dispensation, the Lord calls out: --

``O Zion, that bringest good tidings, get thee up into a high mountain! O Jerusalem, that bringest good tidings, lift up thy voice with strength; lift it up, be not afraid; say unto the cities of Judah, Behold your God!'' Is. 40, 9.

But still more direct, still more unmistakable and powerful, is the Lord's commission to the New Testament Church and to all its members: --

``GO YE THEREFORE AND TEACH ALL NATIONS; BAPTIZING THEM IN THE NAME OF THE FATHER AND OF THE SON AND OF THE HOLY GHOST; TEACHING THEM TO OBSERVE ALL THINGS WHATSOEVER I HAVE COMMANDED YOU.'' Matt. 28, 19. 20.

``Go ye!'' He says.

``Bring My sons from far and My daughters from the ends of the earth!'' Is. 43, 6.

``Ye shall be witnesses unto Me both in Jerusalem, and in all Judea, and in Samaria, and unto the uttermost part of the earth.'' Acts 1, 8.

It is not enough that we build churches and chapels and have free pews for all who desire to come; it is not enough that we erect bulletin-boards and announcement-boards in front of our churches, at street intersections, and in other public places; it is not enough that we publish parish-papers and pulpit programs; it is not enough that we advertise in the newspapers on Saturdays and upon all special occasions. All this is good and laudable; all this means carrying out a part of the work which is ours to do; all this may reach souls that are in need of the message of salvation; all this may bring some into the fold.

``GO!'' means \emph{personal work}, if possible work in person, by direct personal contact; it means seeing that work is done and carrying out the work in person, if possible, or attending to its performance in person, if it must be done through others.

``Go ye also into the vineyard!'' Matt 20, 4. 7. That is the Lord's Commission. It might be done by proxy, of course; but where does a person do his own work, which is entrusted to him, by proxy, unless it be, perhaps, under his own direct, personal supervision?

And there is more to be considered. The Bridegroom, Christ, is sending His friends to win the bride, the believers, whom He wants with Him in the enjoyment of the eternal bliss of heaven. St.~Paul says of his own work in winning souls for Christ: --

``I have espoused you to one Husband that I may present you as a chaste virgin to Christ.'' 2 Cor. 11, 2. And John the Baptist testified: --

``The Friend of the Bridegroom which standeth and heareth Him, rejoiceth greatly because of the Bridegroom's voice. This my joy therefore is fulfilled.'' John 3, 29. John, like Paul, did his work for Christ in person; he gained souls by personal work. It is the way we ought to choose as we have opportunity and in accordance with the method of working set forth by God in His Holy Word.

``Go YE!'' says Christ. The commission is not confined to the apostles. They, indeed, are the teachers of the whole world until the end of time. Through their word others are to believe on Jesus Christ, the Savior of the world. John 17, 20.

``Go YE!'' is not addressed to trained pastors and missionaries only, although to them is committed the task of public preaching. It was a little servant girl, a slave whose name is not even mentioned in the Bible, who called the attention of Naaman's wife to the prophet of Jehovah in Israel. 2 Kings 5, 3. It was the untrained fisherman, Andrew, who told his brother Simon about the Messiah, John 1, 41, and whose example was followed by Philip in speaking to Nathanael. John 1, 45.

``Go ye INTO ALL THE WORLD!'' is the Savior's commission. The world is big and wide, and there are still more than twice as many without the knowledge of Christ as there are, even nominally, within the pale of the Church. Into many of the heathen lands the messengers of the Gospel have gone. Into some of the heathen lands some of our own messengers are gone -- alas! into all too few. There never was a truer answer given than that turned in by a young Christian, the question being: How many missionaries have we in India and China? He said, with wonderful frankness: ``NOT ENOUGH!''

Yes, NOT ENOUGH! We have not reached nearly all lands of the world. We have made but a feeble, an all too feeble, beginning.

Somehow we do not seem to realize the misery of the untold millions that are ``without God in the world.'' Eph 2, 12. Perhaps if they were living across the street from us and we had the picture of their misery and their idolatry and their vileness before our eyes every day, our hearts would be stirred to a greater effort in their behalf, and we might accomplish more in our foreign mission endeavor.

~~We smoothly speak of teeming masses, of millions still in darkest night;\\
\hspace*{0.333em}\hspace*{0.333em}We glibly pray that those in blindness be given spiritual sight;\\
\hspace*{0.333em}\hspace*{0.333em}We prate about our mission duty and of the missionary need:\\
\hspace*{0.333em}\hspace*{0.333em}But what would you do, and what would I do,\\
\hspace*{0.333em}\hspace*{0.333em}IF CHINA WERE ACROSS THE STREET?

~~We say that we are interested when now and then we hear a talk\\
\hspace*{0.333em}\hspace*{0.333em}Of how the heathen hosts are living and in the fiercest horrors walk,\\
\hspace*{0.333em}\hspace*{0.333em}How they are kept the truth from learning, their leaders empty husks them feed:\\
\hspace*{0.333em}\hspace*{0.333em}But what would you do, and what would I do,\\
\hspace*{0.333em}\hspace*{0.333em}IF INDIA WERE ACROSS THE STREET?

~~We feel that we have done our duty when we just sometimes give a mite,\\
\hspace*{0.333em}\hspace*{0.333em}When from the riches of our treasures we now and then deal out a bite;\\
\hspace*{0.333em}\hspace*{0.333em}We spend our billions for vain baubles, for luxuries we do not need:\\
\hspace*{0.333em}\hspace*{0.333em}But what would you do, and what would I do,\\
\hspace*{0.333em}\hspace*{0.333em}WITH AFRICA ACROSS THE STREET?

\begin{center} * * * \end{center}

~~Oh, may the love of Christ constrain us to see our mission duty through\\
\hspace*{0.333em}\hspace*{0.333em}That we be filled with burning fervor, that less we talk and more we do\\
\hspace*{0.333em}\hspace*{0.333em}That we no longer speak of burdens, but lift the misery untold\\
\hspace*{0.333em}\hspace*{0.333em}THAT ALL OUR LIFE BE SPENT IN BRINGING MORE SOULS INTO THE SAVIOR'S FOLD

It is absolutely necessary that we get this better viewpoint, that we begin to think of our missionary duty in terms of direct contact, that we visualize the spiritual needs of those who are still children of wrath without being conscious of that fact.

``INTO ALL THE WORLD!'' Not only the foreign countries, but also the home field, the country in which we live!

~~If you cannot cross the ocean\\
\hspace*{0.333em}\hspace*{0.333em}And the heathen lands explore,\\
\hspace*{0.333em}\hspace*{0.333em}You can find the heathen nearer,\\
\hspace*{0.333em}\hspace*{0.333em}You can help them at your door!

Ah, yes; millions of them, in the very midst of Christianity and civilization!

Do you know whether your nearest neighbors are members of a Christian Church? Have you ever inquired whether the people across the street know anything about the Savior and the way of salvation?

Have you ever considered that thousands of us who are sitting at the full tables of God's riches in Christ Jesus as we have them in our dear Lutheran Church have done little or nothing to bring the Gospel of God's mercy to men and women and children in our own neighborhood?

~~Shall we, whose souls are lighted\\
\hspace*{0.333em}\hspace*{0.333em}With wisdom from on high,\\
\hspace*{0.333em}\hspace*{0.333em}Shall we to men benighted\\
\hspace*{0.333em}\hspace*{0.333em}The lamp of life deny?

Shall we do so by failing to make an earnest effort to reach them by \emph{going} and inviting them to partake in the riches earned also for them by Christ's atoning work?

``PREACH THE GOSPEL!'' That is the means committed to us for the winning of blood-bought souls, the means by which the grace of God is to be brought to the attention of men and to be made alive in their hearts.

Not the so-called Gospel of social service, of which we hear so much in our days; not the message of present-worldliness, with its cry of: Save the people for this world! It is true that the highest forms of social blessings have come to men with Christianity, and that pracitcally every real advance in the world in the last nineteen centuries proceeded from Christianity or is connected with the Christian religion. But that is the effect of the soul-changing power of the Gospel of Christ, a power which is so great that it influences not only those who actually confess Christianity and live in accordance with its high ideals, but that it exerts a purging and a beautifying impulse also on others who come in contact with its monuments.

The Gospel which we are to bring to men is a power of God unto salvation because it is the message of the free grace and mercy of God in Christ Jesus, the one and only Savior of mankind, who in our stead and for our redemption came down from heaven, was incarnate of the Holy Ghost by the Virgin Mary, and gave His life as a ransom for mankind when He died on the cross. It is the message of the forgiveness of sins for the sake of Jesus that is the essence of the gospel. And it is the Gospel which we are bending our efforts to bring to all men everywhere.

``TO EVERY CREATURE,'' ``TO ALL NATIONS!'' To every member of this lost and condemned mankind! To rich and poor, to old and young, to the socially prominent and to the outcast of human society, to the capitalist and to the workingman -- to \emph{all} men this message is to be brought.

``God will have \emph{all} men to be saved and to come unto the knowledge of the truth.'' 1 Tim. 2, 4.

Having made disciples of men wherever we find them, in palace and in hovel, and having brought them to Christ by the Sacrament of Baptism, we are to extend our initial work by teaching them to observe all things whatsoever He has commanded us. There is no end, no limit, to the possibilities of our evangelistic work on this side of the grave. The more we work, the greater are the possibilities and the greater the opportunities for service.

Do you know what place such soul-winning has in the eyes of God?

The whole machinery of redemption was set in motion by Him because of it. Even in the Old Testament He says, time and again, that He is the Savior of His people.

``God, their Savior, which had done great things in Egypt.'' Ps. 106, 21.

``I am the Lord, thy God, the Holy One of Israel, thy Savior.'' Is 43, 4.

``There is no God else beside Me, a just God and a Savior; there is none beside Me.'' Is. 45, 21.

``All flesh shall know that I, the Lord, am thy Savior and thy Redeemer, the Mighty One of Jacob.'' Is. 49, 26.

``Thou shalt know that I, the Lord, am thy Savior and thy Redeemer.'' Is. 60, 16.

``For He said, Surely they are My people, children that will not lie: so He was their Savior.'' Is. 63, 8.

And has not the New Testament fully borne out the promise and the prophecy of the Old? Is not the thought of the salvation of makind on the basis of the love and mercy of God the central theme in every book given to men in the New Dispensation?

``My spirit hath rejoiced in God, my Savior.'' Luke 1, 47.

``Paul, an apostle of Jesus Christ by the commandment of God, our Savior, and Lord Jesus Christ, which is our Hope.'' 1 Tim. 1, 1.

``We trust in the living God, who is the Savior of all men.'' 1 Tim. 4, 10.

``God hath in due times manifested His Word through preaching, which is committed unto me according to the commandment of God, our Savior.'' Titus 1, 3.

``That they may adorn the doctrine of God, our Savior, in all things.'' Titus 2, 10.

``To the only wise God, our Savior, be glory and majesty, dominion and power, both now and forever!'' Jude 25.

Do you want further evidence of the importance of soul-winning as God sees it? Not only does His own name indicate His desire for the salvation of men, but He also states it in words of unmistakable emphasis.

``Have I any pleasure at all that the wicked should die? saith the Lord God, and not that he should return from his ways and live?'' Ezek. 18, 23.

``God will have all men to be saved, and to come unto the knowledge of truth.'' 1 Tim. 2, 4.

``The Lord is not willing that any should perish, but that all should come to repentance.'' 2 Pet. 3, 9.

Do you still need more information to convince you of the interest that God takes in saving men from their sins? Can there be a greater proof than that He sent His only-begotten Son, Jesus Christ, to accomplish the salvation of all mankind?

``We have seen and do testify that the Father sent the Son to be the Savior of the world.'' 1 John 4, 14.

``When the fulness of the time was come, God sent forth His Son, made of a woman, made under the Law, to redeem them that were under the Law.'' Gal. 4, 4.

``God so loved the world that He gave His only-begotten Son, that whosoever believeth in Him should not perish, but have everlasting life.'' John 3, 16.

Is it a wonder, with such facts from the Bible before them, that some of the foremost workers for Christ felt constrained to go and measure up, in some degree, to the expectatino of God?

It was Carey who declared that we must ``expect great things from God,'' and that we must ``attempt great things for God,'' and who in the strength of Is. 54, 2. 3 set out for India.

It was Allen Gardiner who gave up all prospects of becoming wealthy in order to go to the darkest part of South America, Tierra del Fuego, and to lay down his life with the words before his eyes: ``My soul, wait thou only upon God; for my expectation is from Him.'' Ps. 62, 58.

It was David Livingstone who refused to abandon his task after Stanley had found him, but resolutely sent the younger man home with the precious records of work already accomplished, while he turned back to finish alone his great undertaking.

It was Mary Slessor, of Calabar, who left the very coast which served as a station of communication with far-away England and went into the interior to each more of those people, who reverently called her ``Ma,'' the Gospel of salvation.

It was Theodore Fliedner who did not shrink from a discharged female convict, but with this woman as the first inmate of his improvised home began the work which resulted in the revival of the female diaconate.

It was John Geddie who went to Aneityum, laboring there with such success that the native Christians themselves said of him, ``When he landed in 1848, there were no Christians here; when he left, in 1872, there were no heathen.''

It was Brainerd who declared, ``I cared not where or how I lived or what hardships I went through, so that I could but gain souls for Christ.''

It was Gregory who stated, ``Of all the sacrifices there is none in the sight of Almighty God equal to zeal for souls.''

~~``Go ye unto ev'ry nation!''\\
\hspace*{0.333em}\hspace*{0.333em}Is the Savior's great command;\\
\hspace*{0.333em}\hspace*{0.333em}``Preach the Gospel of salvation\\
\hspace*{0.333em}\hspace*{0.333em}To all men in ev'ry land;\\
\hspace*{0.333em}\hspace*{0.333em}Teach them all the glorious message\\
\hspace*{0.333em}\hspace*{0.333em}That I died to end all strife\\
\hspace*{0.333em}\hspace*{0.333em}And that death might be the passage\\
\hspace*{0.333em}\hspace*{0.333em}To the blissful endless life.''

~~`Tis by Jesus' love and merit\\
\hspace*{0.333em}\hspace*{0.333em}All men are at peace with God,\\
\hspace*{0.333em}\hspace*{0.333em}Reassured by His free Spirit,\\
\hspace*{0.333em}\hspace*{0.333em}Saved from all their guilty load.\\
\hspace*{0.333em}\hspace*{0.333em}He who trusts in Christ his Savior,\\
\hspace*{0.333em}\hspace*{0.333em}Who for all men did atone,\\
\hspace*{0.333em}\hspace*{0.333em}Will receive the Father's favor,\\
\hspace*{0.333em}\hspace*{0.333em}Will be saved by grace alone.

~~To the nations most enlightened\\
\hspace*{0.333em}\hspace*{0.333em}With this world's progressive lore,\\
\hspace*{0.333em}\hspace*{0.333em}And to those whose souls are frightened,\\
\hspace*{0.333em}\hspace*{0.333em}Bound by superstitious lore;\\
\hspace*{0.333em}\hspace*{0.333em}Those whose god is this world's mammon\\
\hspace*{0.333em}\hspace*{0.333em}And those deep in poverty,\\
\hspace*{0.333em}\hspace*{0.333em}To the rich and the street gamin,\\
\hspace*{0.333em}\hspace*{0.333em}Comes the call to make them free.

~~Let us shout it full of gladness\\
\hspace*{0.333em}\hspace*{0.333em}Wheresoever men we find;\\
\hspace*{0.333em}\hspace*{0.333em}Let us drive away all sadness,\\
\hspace*{0.333em}\hspace*{0.333em}Grief of heart and care of mine;\\
\hspace*{0.333em}\hspace*{0.333em}Let us tell the wondrous story\\
\hspace*{0.333em}\hspace*{0.333em}Of the marvel of God's love,\\
\hspace*{0.333em}\hspace*{0.333em}Let us magnify His Glory\\
\hspace*{0.333em}\hspace*{0.333em}Till the hardest hearts we move;

~~Till all men of ev'ry station\\
\hspace*{0.333em}\hspace*{0.333em}Rich and poor and young and old;\\
\hspace*{0.333em}\hspace*{0.333em}Till all men of ev'ry nation\\
\hspace*{0.333em}\hspace*{0.333em}May be brought into the fold;\\
\hspace*{0.333em}\hspace*{0.333em}Till the Savior's robe of beauty\\
\hspace*{0.333em}\hspace*{0.333em}Covers ev'ry guilty stain;\\
\hspace*{0.333em}\hspace*{0.333em}Till they know their highest duty\\
\hspace*{0.333em}\hspace*{0.333em}Everlasting life to gain.

Would you know the motive which prompts such a response in the hearts of Christians everywhere? -- You will find more on this point in the next chapters.

\chapter{I Delight to Do Thy Will!}\label{i-delight-to-do-thy-will}

\section*{Is. 40, 8}\label{is.-40-8}

\subsection*{The obligation of Love.}\label{the-obligation-of-love.}

The divine commission is not an arbitrary command; it is not a legal precept issued by God by virtue of His majesty and power. It is, as a matter of fact, addressed to Christians and would have no meaning for anyone else. Only he can understand this commission and properly act upon it in whose heart the Holy Ghost has already wrought a knowledge of the salvation brought by Christ and revealed in His Word. It is a heart of this kind that is actuated by the obligation of love resting upon it.

And how can it be otherwise, since the Christian continually has before his eyes the wonderful picture of Christ and the manner in which He carried out and satisfied the obligation of love resting upon Him by virtue of His own choice?

For what was the guiding principle of His life and work?

``THEN SAID I, LO, I COME; IN THE VOLUME OF THE BOOK IT IS WRITTEN OF ME. I DELIGHT TO DO THY WILL, O MY GOD.'' Ps. 40, 8.

These are words of the Messiah, as the writer to the Hebrews, chap.~10, 5-7, shows. The Son of God had from eternity taken part in the counsel of God pertaining to fallen mankind, and He had declared His willingness to work the redemption, which none but He could accomplish. This attitude is evident throughout our Savior's life.

``Wist ye not that I \emph{must} be about My Father's business?'' was the half-reproachful question which He addressed to His parents when He was taken to the festival of the Passover at the age of twelve years. Luke 2, 49.

``I \emph{must} preach the kingdom of God to other cities also, for therefore am I sent,'' was His declaration to those who sought Him for His miracles only. Luke 4, 43.

``I \emph{must} walk to-day and to-morrow and the day following.'' Luke 13, 33.

``I \emph{must} work the works of Him that sent Me while it is day; the night cometh when no man can work.'' John 9, 4.

``From that time forth began Jesus to show unto His disciples how that He \emph{must} go unto Jerusalem, and suffer many things of the elders and chief priests and scribes, and be killed, and be raised again the third day.'' Matt. 16, 21.

``For I say unto you that this that is written \emph{must} yet be accomplished in Me, And He was reckoned among the transgressors.'' Luke 22, 37.

``Thinkest thou that I cannot now pray to My Father, and He shall presently give Me more than twleve legions of angels? But how then shall Scriptures be fulfilled that thus it \emph{must} be?'' Matt. 26, 53-54.

``Remember how He spake unto you when he was yet in Galilee, saying, The Son of Man \emph{must} be delivered into the hands of sinful men and be crucified and the third day rise again.'' Luke 24, 7.

``\emph{Ought} not Christ to havve suffered these things and to enter into His glory?'' Luke 24, 26.

``And He said unto them, Thus it is written, and thus \emph{it behooved Christ} to suffer and to rise from the dead the third day.'' Luke 24, 46.

Thus we find it all the way through the life of Christ, -- the ``must'' of the divine obligation resting upon Him. He has placed Himself at the disposal of God, and in line with His own eternal will, which is at all times in perfect agreement with that of the Father, John 5, 19, He carried out the plan of redemption.

What the German hymn-writer Paul Gerhardt has the Savior say is true: --

~~~Yea, Father, yea most willingly\\
\hspace*{0.333em}\hspace*{0.333em}\hspace*{0.333em}I'll bear what Thou commandest;\\
\hspace*{0.333em}\hspace*{0.333em}\hspace*{0.333em}My will conforms to Thy decree,\\
\hspace*{0.333em}\hspace*{0.333em}\hspace*{0.333em}I do what Thou demandest.--\\
\hspace*{0.333em}\hspace*{0.333em}\hspace*{0.333em}O wondrous Love, what hast Thou done!\\
\hspace*{0.333em}\hspace*{0.333em}\hspace*{0.333em}The Father offers up his Son,\\
\hspace*{0.333em}\hspace*{0.333em}\hspace*{0.333em}The Son, content, descendeth!\\
\hspace*{0.333em}\hspace*{0.333em}\hspace*{0.333em}O Love, O Love, how strong art Thou!\\
\hspace*{0.333em}\hspace*{0.333em}\hspace*{0.333em}In shroud and grave Thou lay'st Him low\\
\hspace*{0.333em}\hspace*{0.333em}\hspace*{0.333em}Whose word the mountains rendeth!

Where would we and all mankind be if the Savior had wavered in His divine determination, if He had faltered and shrunk at sight of the cross on which His tortured body was to be suspended?! What an immeasurable burden of gratitude is laid upon us by virtue of His unflinching persistence in the obedience prompted by His redemptive love!

Are you looking for still further evidence regarding the position which soul-winning has in the mind of Christ, the one and only Savior of mankind? Consider the place it has in His life and work. Remember that His very name indicates the purpose of His life and work; for Jesus means ``Redeemer, Savior.'' Matt. 1, 21.

It is the name given to our Lord throughout the New Testament; it is used by the inspired writers with an evident feeling of exultation. The very angel of the Lord speaks it with a hushed reverence when he announces the birth of the Lord: --

``Unto you is born this day in the city of David a Savior, which is Christ the Lord.'' Luke 2, 11.

It is found in the joyful testimony of the Samaritans of Sychar: --

``Now we believe, not because of thy saving; for we have heard Him ourselves and know that this is indeed the Christ, the Savior of the world.'' John 4, 42.

And think of the numerous other passages in which the name is blazoned as on a banner to be borne before the eyes of the believers, to make them realize ever more fully the unspeakable gift of God! Read them for yourself: Acts 5, 31; 13, 23; Phil. 3, 20; 2 Tim. 1, 10; Titus 1, 4; 2, 13; 3, 6; 2 Pet. 1, 11; 2, 20; 3, 2.18; 1 John 4, 14.

What the name of Jesus indicates, what the angel's explanation proclaims, that is emphasized in Christ's earthly mission. No one has said it better, no one could express it more definitely than the Lord Himself when He says: --

``The Son of Man is come to save that which was lost.'' Matt. 18, 11. And again: --

``The Son of Man is come to seek and to save that which was lost.'' Luke 19, 10.

This is also the clear statement of that ``Gospel in a nutshell,'' given in Christ's own words: --

``God so loved the world that He gave His only-begotten Son, that whosoever believeth in Him should not perish, but have everlasting life.'' John 3, 16.

``These things I say that ye might be saved.'' John 5, 34.

``I am the Door; by Me, if any man enter in, He shall be saved.'' John 10, 9.

``This is the will of Him that sent Me, that every one which seeth the Son and believeth on Him may have everlasting life, and I will raise him up at the last day.'' John 6, 40.

In the very performance of His miracles our Lord's chief gift was that of the forgiveness of sins with its assurance of salvation. To the man sick of the palsy He gave, first of all, that wonderful certainty: --

``Son, be of good cheer; thy sins be forgiven thee.'' Matt. 9, 2; Luke 5, 20.

And when the great sinner knelt at His feet in the house of the Pharisee, the most outstanding gift of Christ is that which He Himself indicates: --

``Wherefore I say unto thee, Her sins, which are many, are forgiven.'' Luke 7, 47.

That this winning of souls for the kingdom of God was the object of Christ in all His preaching, in all His work, is obvious from the general tone and tendency of all His acts and all of His precepts. He tells the former demoniac to preach the kingdom of God. He summarizes His own invitation in the words: ``Go out quickly into the streets and lanes, highways and hedges, and compel them to come in.'' As Dr.~Pierson says: ``The command is one which is incarnated in His whole life and is suggested or implied in the very idea of discipleship: `Follow Me, and I will make you fishers of men.'\,''

Do we need further evidence to convince us that the obligation of love was the guiding principle of the Savior's life and that the importance of soul-winning in His work is the outstanding feature of the entire Gospel? If nothing else will impress us, we cannot deny the witness of His death upon the cross. He Himself says of it: --

``And I, if I be lifted up from the earth, will draw all men unto Me.'' John 12, 32.

Read the account of the gospels, the description of the Savior's crucifixion and of His death on Calvary. Cp. Luke 23, 32-43.

The matter is most beautifully put by St.~Paul when he writes: --

``The life which I now live in the flesh I live by the faith of the Son of God, who loved me and gave Himself for me.'' Gal. 2, 20.

``Who gave Himself for us that He might redeem us from all iniquity and purify unto Himself a peculiar people.'' Titus 2, 14.

Truly, it is a remarkable topic, and one which should duly impress us with the unbounded glory of the love of Jesus in His vicarious redemption and with the fulness of the love which could cause the great Son of God so to humble Himself for our sakes.

\begin{center}\rule{0.5\linewidth}{0.5pt}\end{center}

But now comes the test for every one of us. As St.~Paul puts it: --

``Let this mind be in you which was also in Christ Jesus, who, being in the form of God, thought it not robbery to be equal with God, but made Himself of no reputation and took upon Him the form of a servant and was made in the likeness of men; and, being found in fashion as a man, He humbled Himself and became obedient unto death, even the death of the cross.'' Phil. 2, 5-8.

The mind of Christ was that according to which He felt the obligation laid upon Him by His Father's love and His own; it was the mind which caused Him to be the great Servant of mankind in order to show them the way of salvation. Jesus Himself calls our attention to this phase of His work: --

``Whosoever will be great among you, let him be your minister; and whosoever will be chief among you, let him be your servant; even as the Son of Man came, not to be ministered unto, but to minister and give His life a ransom for many.'' Matt. 20, 26-28.

The obligation of love which rested upon Jesus has passed on to us, who bear His name and are filled with His spirit. The wonderful union which has been established between Christ and us by virtue of the faith that lives in us has given us some of His power. Since Christ has made his abode in us, together with the Father and the Holy Ghost, we are in a position to bear much fruit of the kind which He inspires and loves. We are now, as St.~Paul writes, His workmanship, created in Christ Jesus unto good works, which God hath before ordained that we should wlak in them. Eph. 2, 10.

In accordance with these facts there is one great motto which Christians love to keep before their eyes at all times, namely: --

``The love of Christ constraineth us.'' 2 Cor. 5, 14.

Obviously this is not the constraint of the Law and of fear; for ``perfect love casteth out fear.'' It is the urgency and the power of the love which we have received in Christ, as an outflow of the divine power in Christ, and it is the zeal which now impels us forward for love of Christ, in appreciation of the boundless mercy which we have received.

Is it necessary to emphasize this point any further? Is the obligation of love brought to our attention to-day and with reference to the situation as we have it before our eyes in the world? Have we a responsibility which we ought to feel with at least a small fraction of the fervor and zeal shown to us by Christ?

Oh, the need of the world for the love which we alone can bring to men by virtue of the Gospel entrusted to us is still immensely, overwhelmingly great. It is not only that men are without Christ, in a kind of a neutral situation, but it is that millions of them are living in open and shameful opposition to Him, children of wrath and heirs of eternal damnation.

Here are some of the facts as they are accessible to us to-day with regard to the WORLD WITHOUT CHRIST!

According to the latest available statistics the population of India is 320,000,000. Now, if we figure all the Protestant societies that are now working in that country of teeming millions (and that includes not a few whose Christianity is of the very liberal kind, not much better than the religion of the heathen themselves), we have far fewer than a million baptized Christians (849,500). Even if we count all those who are members of the Roman Catholic and of the Syrian churches, we have barely five million Christians! Barely one and one half percent of the total population -- and the gains that are being made are so heart-breakingly small! Does our obligation extend to India?

The situation in Southeastern Asia, including Assam, Burma, Siam, and the Malay Peninsula, French Indo-China, that is, all countries east of India and south of China, is as follows. The population, all told, is somewhat over 53,000,000. In this great mass there are fewer than 100,000 Christians, and some sections may be said to be altogether unoccupied as yet. Not even one-fifth of one per cent. won for Christ!

Next comes the immense country of China, with its more than 4,200,000 square miles and its population of 440,000,000. Do you know that here, ALL TOLD, the number of communicant Christians has not yet reached the 400,000 mark, although 174 societies are now at work? The fraction is so infinitesimally small that one hesitates to write it. Entire provinces are still without so much as one messenger of salvation!

Japan's population exceeds 60,000,000, and we have read so much about Christian leaders in the island empire that we have probably overestimated the number of Christians. As a matter of fact, the latest statistics give the number of communicant members of all Protestant missions as not quite 200,000. Again a number which is quite disheartening in its smallness!

As we go over to Korea, which has had intercourse with the Western World for a matter of only a few years, we find a population of 17,000,000 under Japanese rule. Although there are many factors in this country which have been found favorable to mission-work, yet the number of Protestant Christians is below 100,000, or not yet one half of one per cent.

As we next look at the Near East, comprising Egypt, Asia Minor (with Armenia and Kurdistan), Syria, Palestine, Arabia, Mesopotamia, and Persia, the situation is still more depressing. The total population of this section of the world is estimated at almost 55,000,000. We have here the location of the cradle of the human race, the site of the world's greatest ancient empires, the land of the Bible and of the Savior. We still have remnants of the Armenian Church, nominally Christian, in Armenia, there are many sects of the Greek Orthodox Church and one or two of the Roman Catholic Church in this section, not to speak of the Coptic Church in Egypt; but the number of Christians is at best very small, and the number of Protestants is as yet below 20,000.

Next we consider Africa, the ``Dark Continent.'' Its native population is estimated, with some degree of probability, as reaching 150,000,000. In this entire number there are only three million Protestant Christians, and possibly seven million more, who are nominally members of the Abyssinian, Coptic, and Roman Catholic churches. Again the discrepancy is so great that it is appalling.

Latin America includes Central and South America, with a total of 85,000,000. Til now hardly more than a beginning has been made in bringing the Gospel to this mixed population; for, although almost all the countries concerned are nominally Roman Catholic, yet the number of professed Christians amounts to only a very small percentage of the total, since the workers, all told, amount to barely 2,500. A moment's reflection will show the utter inadequacy of the present missionary occupation.

There are a few spots in Oceania, or in the islands of the Pacific, which offer a distinct relief. We are here dealing with Malaysia, Melanesia, Micronesia, and Polynesia, whose combined population is more than 60,000,000. A few islands are entirely Christianized, but over ninety-five per cent. of the territory is still without the Gospel-message, some sections having not even been touched.

And what shall we say of the unoccupied fields in many parts of the world, which stand as a constant challene before the eyes of Christianity? Is it the ``regions beyond'' that offer the most serious problems at the present time, because circumstances have here combined to keep out the name and the Word of Christ. There is the heart of Asia, with Mongolia, Chinese Turkistan, Tibet, Afghanistan, and Baluchistan; there is the interior of Africa, with almost fifty pagan tribes; there is the heart of South America, many parts of which are not even explored.

That is the challenge to Christianity, that is our obligation of love!

Nor have we as yet mentioned the field wich is both a problem and the most emphatic challenge, at our very doors. Even if we count all those who are only nominally members of Christian churches in our country, we have

\begin{center} BETWEEN 60 AND 65 PER CENT. OF OUR TOTAL POPULATION NOT WITHIN THE CHURCH \end{center}

Think of it: some 65,000,000 of our fellow-citizens in this country have not yet accepted the Gospel of Jesus Christ unto their salvation, and, stranger still, many of these have not even heard of their Savior. In the midst of a so-called Christian civilization, people have never been approached with a view of making them acquainted with the great truths which will bring redemption also to them, the justification which is ready for them in the perfect atonement of Jesus Christ.

And the matter is of unusual interest to

\begin{center} US LUTHERANS! \end{center}

Due partly to the need of gathering those who applied to us for spiritual care during the great immigration from Lutheran and semi-Lutheran countries, partly to the unfortunate language question, we have not yet reached out to our fellow-citizens as opportunity offered. And what is more, a conservative estimate tells us that

\begin{center} ABOUT TEN MILLION PEOPLE OF LUTHERAN EXTRACTION IN THIS COUNTRY ARE NOT CONNECTED WITH THE LUTHERAN CHURCH! \end{center}

So many reasons have been advanced for this condition. But, whatever the reason, these souls are a constant challenge to us, they present to us the

\begin{center} OPPORTUNITY AND THE OBLIGATION OF LOVE! \end{center}

It is because personal work has been so largely neglected in our midst that the deficit in souls is so great against our Church. There can be no question concerning the fact that, in addition to the public proclamation of the Gospel, words for christ to the individual are most effective in the winning of souls. A kind, but earnest word to a negligent churchgoer of our own confirmation class, a tactful invitation to a neighbor, a letter confessing Christ in a frank manner -- these are the things that count with the individual and often serve as entering-wedges for the Word of salvation.

``It is the man-to-man work that tells. And because it is this work that is most effective, this is the work that is best to do. Even though it is less attractive work, as we look at it, and seems to others less important to be done, we must admit that the results are worth considering. As John B Gough said of the one loving word of Joel Stratton that won him: `My friend, it may be a small matter for you to speak the one word for Christ that wins a needy soul, -- a \emph{small matter to you}, but it is \emph{everything to him}.' It is forgetting this truth that causes personal work to be neglected.'' (\emph{Trumbull}.)

It was the greatest missionary of all times that said, as he summarized the devotion of a lifetime in one sentence: --

``I AM DEBTOR both to the Greeks and to the barbarians; both to the wise and to the unwise.'' Rom. 1, 14.

If we realize the obligation of love resting upon us,

\begin{center} WE ARE DEBTORS! \end{center}

\chapter{Workers Together with Him!}\label{workers-together-with-him}

\section*{2 Cor. 6, 1}\label{cor.-6-1}

\subsection*{The Biblical Precept and Example.}\label{the-biblical-precept-and-example.}

It is a wonderful name: ``WORKERS TOGETHER WITH GOD!'' -- a name of rich content, a name which bestows a world of honor upon us.

The very expression ``together with us'' is full of significance and power. It reminds us of so many other gifts and blessings of God, especially of those which were so richly imparted to us in Christ Jesus.

We are \emph{heirs together with Christ}, as St.~Paul so beautifully states: --

``The Spirit itself beareth witness with our spirit that we are the children of God; and if children, then heirs; heirs of God and \emph{joint heirs with Christ}.'' Rom. 8, 16. 17.

And it is particularly comforting to us, who are descendants of heathen, that St.~Paul writes: --

``That the gentiles should be fellow-heirs and of the same body and partakers of His promise in Christ by the Gospel.'' Eph. 3, 6.

We are \emph{partakers together of the life in Christ}, which will have its culmination in the enjoyment of the glory of heaven. The apostle states: --

``It is a faithful saying: For if we be dead with Him, we shall also live with Him; if we suffer, we shall also reign with Him.'' 2 Tim. 2, 11. 12.

The honor which has thereby been bestowed on the human race can hardly be estimated highly enough, for it is one point of evidence showing the greatness of God's mercy toward us.

The very inspired writers marvel at some of the facts connected with the history of man's redemption. In the mystery of the incarnation, for instance, one might well wonder why the Lord did not appear in the form of an angel to bring redemption to men. But we are told: --

``Verily He took not on Him the nature of angels, but He took on Him the seed of Abraham.'' Heb. 2, 16.

Well may we sing in the glorious Christmastide:

~~Th' eternal Father's only son\\
\hspace*{0.333em}\hspace*{0.333em}For a manger leaves His throne;\\
\hspace*{0.333em}\hspace*{0.333em}Disguised in our poor flesh and blood\\
\hspace*{0.333em}\hspace*{0.333em}Is now the everlasting Good.

The mystery of the incarnation of our Lord, as the first step in the perfected redemption, is so great that the ``angels desire to look into'' the marvelous facts connected therewith. 1 Pet. 1, 12.

It is true, moreover, that the Lord uses the holy angels as His messengers. Thus we find that the angel Gabriel was at various times sent to Daniel, particularly to strengthen and comfort him on account of the visions which were given to him. The same angel was sent also to Mary and to Zacharias.

But angels are not honored with the name of WORKERS TOGETHER WITH GOD. While an angel brought the news of the birth of the Savior to the shepherds on the fields of Bethlehem, and while it was a chorus of angels that first sang an anthem of praise in glorifying God for this holy birth, it is true, nevertheless, that angels were not entrusted with the divine commission, but this \emph{distinction was given to human beings}.

NOT ANGELS, BUT MEN are chosen by God to preach the Gospel to every creature; upon MEN is placed the obligation of love. Those whose brother the Savior became by His sacred incarnation are to make known to all members of the human family the news of the redemption wrought by their Brother.

With the consciousness of this distinction, of this unequaled honor, we can understand the precepts of the Lord. For it is not only in the divine commission itself that He speaks to us concerning the need of bringing the message of salvation to others, but also in many other passages, whose import and significance should be considered by us with the most assiduous attention.

Even in the Old Testament we find the Lord calling out to us in an excess of jubilation: --

``Say among the heathen that the Lord reigneth; the world also shall be established that it shall not be moved. He shall judge the people righteously.'' Ps. 96, 10.

And we may well consider, in this connection, passages like Ps. 117, 1; Is. 34, 1; Jer. 4, 2.

But it is in the New Testament that this feature of bringing to others the assurance of the redemption gained by Christ is particularly prominent. Who could forget the words addressed by Christ to the healed and grateful demoniac: --

``Return to thine own house and show how great things God hath done unto thee''? Luke 8, 39.

In this case it required no second urging, for we are told that ``he went his way and published throughout the whole city how great things Jesus had done unto him.''

Can we afford to do less with the fulness of God's spiritual blessings resting upon us?

The words of St.~Paul to the Galatians are well known, but they will bear repetition: --

``Let us not be weary in well-doing; for in due season we shall reap, if we faint not. As we have therefore opportunity, let us do good unto all men, especially unto them who are of the household of faith.'' Gal. 6, 9. 10.

Can we pass on greater blessings than the forgiveness of sins, peace with God, the happiness of a good conscience, which are ours in the Gospel?

What more impressive precept than that contained in the words of the great apostle and missionary in Col. 1, 26-29?

``Even the mystery which hath been hid from ages and from generations, but now is made manifest to His saints: to whom God would make known what is the riches of the glory of this mystery among the Gentiles, which is Christ in you, the hope of glory; whom we preach, warning every man and teaching every man in all wisdom, that we may present every man perfect in Christ Jesus; whereunto I also labor, striving according to His working, which worketh in me mightily.''

Words of the greatest humility, surely, that Paul is also trying to do his share. And shall we not strive to follow the precept contained in this declaration with regard to making known to others the riches of the glory of this mystery?

Consider for a moment the place that soul-winning occupied in the apostolic mind. The early Church proved itself in every way a soul-winning organization. We are told that on Pentecost Day those who gladly received the words of Peter were baptized; and the same day there were added unto them about three thousand souls. Acts 2, 41. Again we read that many of them which heard the Word believed; and the number of men was about five thousand. Acts 4, 4. Once more we are told that believers were the more added to the Lord, multitudes both of men and women. Acts 5, 14. And again, that the number of the disciples multiplied in Jerusalem greatly. Acts 6, 7. Compare Acts 12, 24; 19, 18. 20.

Can we ever forget that Saul, just as soon as he himself was brought to the knowledge of the truth, set about winning others, and that the passion for souls never left him? How remarkably well the purpose of his life is set forth in his words to the Romans: --

``Brethren, my heart's desire and prayer to God for Israel is that they might be saved!'' Chap 10, 1.

But there is still more to be considered in trying to realize the full significance of being WORKERS TOGETHER WITH GOD.

The example of those who were, in the special sense of the word, ministers of God, who were in charge of the public proclamation of the Word, stands out on practically every page of Holy Writ. Even in the Old Testament we find Moses pleading with Hobab, his brother-in-law: --

``Come thou with us, and we will do thee good; for the Lord hath spoken good concerning Israel\ldots.And it shall be, if thou go with us, yea, it shall be that what goodness the Lord shall do unto us, the same will he do unto thee.'' Num. 10, 29. 32.

In a much higher degree and to a much greater extent we find this spirit in the spiritual leaders of the New Testament, not only in St.~Paul, but in Peter and the other apostles as well. They all regarded their own lives as being of little account if they might but win souls for Christ. Publicly and from house to house they made known the message of salvation that by all means some might be saved.

But a fact which is still more interesting and significant in the present study is that which concerns

\begin{center} THE WORK OF LAYMEN IN THE APOSTOLIC CHURCH \end{center}

This fact is brought to our attention from a number of angles, both by express statements and by deductions which may rightly be made from the narrative.

Have you ever stopped to consider how many congregations were founded by laymen in the years between 30 and 60 A.D.?

We are familiar with the fact that Stephen was the first Christian martyr. Now it is a remarkable testimony to the manner in which God carries on the work of His kingdom that the persecution which followed the assassination of Stephen was an instrument in the hands of Providence to spread the Gospel.

``And they were scattered abroad throughout the regions of Judea and Samaria, except the apostles\ldots. They that were scattered abroad went everywhere, preaching the Word.'' Acts 8, 1. 4.

Mind you, it is expressly stated that the apostles were not included in this scattering. For the present they stayed in Jerusalem. It was the lay members that spread the Gospel through Western Judea and Samaria. The apostles received information that churches had been established in Samaria, and only then did they send Peter and John to establish these congregations by giving them a regular ministry. Acts 8, 14.

At Damascus likewise Christians were found, and evidently in some numbers; for Saul went there with the avowed intention of bringing men and women bound to Jerusalem. Acts 9, 2. 14.

Western Judea had been so thoroughly evangelized by Christians from Jerusalem, Acts 8, 1, that there were saints, believers, members of the Church at Lydda, at Saron, and at Joppa. Acts 9, 32. 35. 36.

Of special interest in this connection is the account which we find of the founding of churches in Syria, because Antioch later became a great center of Christianity. We are told: --

``Now, they which were scattered abroad upon the persecution that arose about Stephen traveled as far as Phenice and Cyprus and Antioch, preaching the Word to none but unto the Jews only. And some of them were men of Cyprus and Cyrene, which, when they were come to Antioch, spake unto the Grecians, preaching the Lord Jesus.'' Acts 11, 19. 20.

We have here, then, the first account of an open attempt to bring the Gospel to such as were not Jews, to people of an entirely different race and nation. This attempt was made by men from the island of Cyprus and from the region of Cyrene in Egypt. These men, as members of the Jewish Dispersion, were more favorably inclined toward those speaking a different language. It was not a liberal attitude, but a fine zeal for the spread of the Gospel which actuated them.

The reference to Phenicia at this point explains another fact which is brought to our attention somewhat later. In Acts 21, 3. 4 we find that there were disciples at Tyre, and not only men, but also women and children, and in verse 7 we read of brethren at Ptolemais, another Phenician port. These congregations evidently belonged to the group of those that have been founded in the early thirties, when the persecution following the death of Stephen took place.

But the most interesting story is that of the congregation at Rome. We have no historical account of the founding of this church, and it was already in existence for some time when Paul wrote his epistle, early in the year 57. The obvious conclusion is this, that some of the strangers from Rome, Acts 2, 10, returned to the capital city of the empire and testified for the truth of the Gospel. At first the congregation was strongly Jewish in character, but in the course of time Gentiles were added, until they formed the majority. It is quite certain that no other apostle had been in Rome at the time when Paul sent his letter, Rom. 15, 20; for his entire letter is an exposition of the fundamentals of Christian doctrine. The congregation at Rome, founded by laymen at an early date, had maintained itself for more than two decades, being fortunate in having experienced Christians from other parts of the empire visiting them from time to time and strengthening them in their work. So well had the Christians of Rome been established by the early sixties that the decree of Claudius banishing all Jews from Rome had not permanently injured the congregation; for after the death of Claudius the work had rapidly been reestablished and stations founded even in the suburbs, as at Puteoli. Acts 28, 13. 14.

To these interesting and fascinating facts much other information could be added, but we must here add at least one further section, namely, that concerning individual, personal work; for such it is that every Christian is in a position to do.

It is here that we are amazed at the array of facts bringing home to us the BLESSEDNESS OF PERSONAL WORK.

Who will, at this point, not immediately think of the shepherds at Bethlehem, who immediately ``made known abroad the saying which was told them concerning this Child?''? Luke 2, 17. We think also of Anna, the prophetess, who ``spake of Him to all them that looked for redemption in Jerusalem.'' Luke 2, 38.

We cannot omit a reference to the first five disciples of the Lord, two of whom at once became missionaries and informed others of their happiness in having found the Messiah. John 1, 41. 45. Almost pathetic is the case of the Samaritan woman, who ran to the city with her eager announcement: --

``Come, see a man which told me all things which ever I did: is not this the Christ?'' John 4, 29.

These are all cases of testifying for Christ and thus spreading the Gospel of salvation by direct personal contact. But we have records of even more work done by the early Christians, and that by lay members.

A most outstanding example is that of AQUILA and PRISCILLA.

Our first information concerning this consecrated Christian and his equally consecrated wife is given us in connection with the decree of Claudius commanding the Jews to depart from Rome. This was in the year 49 A.D. They came to Corinth and established themselves there. The next year the Apostle Paul, coming over from Athens, worked in the shop which they had established, the craft being that of tent-making. When Paul left the city after about twenty months, these two consecrated Christians accompanied him as far as Ephesus, where they remained when he found it necessary to hurry away to Jerusalem. In the interval of his absence they did fine work in establishing the congregation in Ephesus and in making known to Apollos the full counsel of God, as we read in Acts 18. When Paul came down to Ephesus about two years later, fine progress had been made. Aquila and Priscilla opened their home to the congregation as a meeting place, as Paul notes with grateful appreciation. 1. Cor. 16, 19.

In the year 54 A.D. Claudius Caesar died, and so, the decree banishing the Jews from the capital city was no longer in force. Accordingly, Aquila and Priscilla returned to Rome, evidently to take care of their business interests. But they were just as active in church-work here as they had been in Corinth and in Ephesus; for we find that Paul, writing from Corinth in 57, sends greetings to the church that is in their house. Rom. 16, 5. Some nine or ten years later, however, we find them living once more in Ephesus, for St.~Paul, writing to Timothy in that city, sends greetings to ``Prisca and Aquila.'' 2 Tim. 4, 19.

What a wonderful career is here sketched for us: consecrated lay people serving the great apostle in various ways, as he indicates Rom. 16, 3-5, giving diligence to the instruction of Apollos, harboring the congregation in their house!

Nor were they the only Christians whose labors for the Lord are acknowledged by Paul. If we but read the list of names in Romans 16 and note what he has to say with regard to the several people mentioned there; if we glance at the first verses in Philippians 4; if we consider, for a moment, the names of Luke, of Aristarchus, of Tychicus, and others, of Gaius, concerning whom St.~John writes that his charitable endeavors are very acceptable indeed, 3 John 5-8, -- we are surprised in a manner which causes us to emulate their example.

Truly, we can be

\begin{center} WORKERS TOGETHER WITH HIM, \end{center}

if we but follow the Biblical precept and example.

\chapter{Zealously Affected in a Good Thing!}\label{zealously-affected-in-a-good-thing}

\section*{Gal. 4, 18}\label{gal.-4-18}

\subsection*{Qualifications of the Workers.}\label{qualifications-of-the-workers.}

The Scripture-passage which we have at the head of the present chapter is peculiarly appropriate to our discussion. The Apostle Paul was not in sympathy with an attitude which is always ready to receive, the plea being that faith must be disassociated from works. It is true that saving faith in its essence is the receiving of the grace of God in the Gospel.

But saving faith is, nevertheless, a living faith. It is a light which not only receives fuel, but which also shines. The apostle fittingly calls it ``faith \emph{which worketh by love},'' Gal. 5, 6, that is, a faith which is active in love, which shows itself in works of love.

It is here that we apply the admonition of St.~Paul. He was decidedly impatient with the false teachers among the Galatians, who were trying to lead the believers back into the bondage of the Law, to keep its precepts for the purpose of attaining to a righteousness of works and thus setting aside the righteousness of Christ, the righteousness which comes to men by faith in His vicarious atonement. For that reason the apostle denounced the false teachers in no uncertain terms, telling the Christians that the interlopers were zealously affecting them, but not well; they were trying to stir them up to a frenzy of work-righteousness, but not in a manner which would really redound to the Christians' highest good.

Over against this wrong position the apostle places the correct principle, that of the Christians' growth in sanctification. He writes: --

\begin{center} "IT IS GOOD TO BE ZEALOUSLY AFFECTED ALWAYS IN A GOOD THING!" \end{center}

Paul wanted the Galatian Christians to have the proper attitude of affection toward their Lord and Savior at all times, whether he were present or not. This attitude would result in a zeal for Christ and His kingdom, which would show itself at all times.

There is no finer maxim than the above for the soul-winner, in whom the passion for souls is an every-day matter, who is ever on the lookout for opportunities to be more active in the Lord's service, to fit himself better for the privileges of the church-member, FOR THE WINNING OF SOULS!

It is these qualifications which concern us in this chapter. Not as though they would have to be our full possession before we may begin on our task, but that they are placed before us as ideals after which we may strive all our life.

The fundamental principle, as indicated above, is that of a \emph{faith active in love}. This faith is not a mere reliance on the ability of Christ as an inspiring leader nor a mere appreciation of His greatness as teacher, as so many Modernists sanctimoniously love to picture it, but it is the acceptance of Jesus Christ as one's personal Savior, who by his vicarious redemption has freed us from the guilt and the punishment of sin. By virtue of this faith, men are justified in the sight of God, they are declared to be righteous, the perfect righteousness of Christ being imputed to them.

``By grace are ye saved, through fiath; and that not of yourselves, it is the gift of God; not of works, lest any man should boast.'' Eph. 2, 8. 9.

While this faith excludes all works of man as having any value for the gaining or keeping of salvation, Rom. 3, 28, by the same token this faith is most decidedly active in love. In fact, it does not and cannot exist without works, no more than a good tree can be without fruit. Jas. 2, 17; Matt. 5, 16. Since God has prepared the good works beforehand that we should walk in them, 2 Cor. 9, 8; Eph. 2, 10, it follows that these works will be plentiful in the measure of our own spiritual life.

``Fruitful in every good work.'' Col. 1, 10.

What wonderful opportunities are open to those who are earnestly concerned about the obligation of love resting upon them!

Right here we ought to pause a moment to consider the \emph{power of faith}, that is, of that calm trust in God, connected with saving faith, which relies upon His promises in spite of all difficulties. We are not strong in our own power, but for that reason the power of God and Christ in us are in a position to do all the more through us. The assurance which the Lord gave to His apostle is ours also: --

``My grace is sufficient for thee; for My strength is made perfect in weakness.'' 2 Cor. 12, 9.

Paul's conclusion is: ``When I am weak, then am I strong.'' 2 Cor. 12, 10c.

~~``When I am weak, then am I strong.''\\
\hspace*{0.333em}\hspace*{0.333em}Though hostile armies round me throng\\
\hspace*{0.333em}\hspace*{0.333em}And fill my heart with fear;\\
\hspace*{0.333em}\hspace*{0.333em}Although they jeer on ev'ry side,\\
\hspace*{0.333em}\hspace*{0.333em}My humble faith and trust deride,\\
\hspace*{0.333em}\hspace*{0.333em}And their contempt I bear.

~~``When I am weak, then am I strong'';\\
\hspace*{0.333em}\hspace*{0.333em}When I am conscious of the wrong\\
\hspace*{0.333em}\hspace*{0.333em}That still infests my soul;\\
\hspace*{0.333em}\hspace*{0.333em}When I my utter weakness feel,\\
\hspace*{0.333em}\hspace*{0.333em}No man my bleeding wounds can heal,\\
\hspace*{0.333em}\hspace*{0.333em}No man can make me whole.

~~``When I am weak, then am I strong'';\\
\hspace*{0.333em}\hspace*{0.333em}For Him I count my friends among\\
\hspace*{0.333em}\hspace*{0.333em}Who gave His life for me.\\
\hspace*{0.333em}\hspace*{0.333em}By faith His mighty strength He gives,\\
\hspace*{0.333em}\hspace*{0.333em}By faith my burdened spirit lives,\\
\hspace*{0.333em}\hspace*{0.333em}His love has set me free.

~~``When I am weak, then am I strong,''\\
\hspace*{0.333em}\hspace*{0.333em}And though the path seem often long,\\
\hspace*{0.333em}\hspace*{0.333em}His love points out the goal;\\
\hspace*{0.333em}\hspace*{0.333em}He guides and leads me day by day,\\
\hspace*{0.333em}\hspace*{0.333em}He keeps me in the narrow way,\\
\hspace*{0.333em}\hspace*{0.333em}He welcomes home my soul.

``I CAN DO ALL THINGS THROUGH CHRIST, WHICH STRENGTHENETH ME!'' Phil. 4, 13.

With faith in Christ as the basis everything else may be accomplished. This faith brings about a \emph{genuine belief in the work} in which we are engaged, as one of the supreme things worth while. The attitude of faith is the attitude of devotion, of actually offering up oneself to the Lord.

``I beseech you, therefore, brethren, by the mercies of God, that ye present your bodies a living sacrifice, holy, acceptable unto God, which is your reasonable service.'' Rom. 12, 1.

The choice is not that of a mere intrinsic value and interest, though this factor is important enough in itself, but the attitude plainly says: I am interested in the work of soul-winning because God wants me to be, and because it is so vitally worth while. It is one of the highest forms of service and of devotion to Him.

Such an attitude begets \emph{confidence}, not only in the worthiness of the enterprise, but in the certainty of success under God's guidance and with His help. We have God's command and promise on our side, and therefore our determination must and shall be equal to the definiteness of His Word.

``All things are possible to him that believeth.'' Mark 9, 23.

To this personal consecration must be added positive virtues as they are painted so beautifully in the Bible.

There is the virtue of \emph{faith}, that is, of unwavering trust in God and His promises, as described above.

There is the virtue of \emph{love}, whose obligation we are here considering, set before us in the incomparable ``Psalm of Love'' contained in 1 Cor. 13.

There is the virtue of \emph{hope}, the outgrowth of saving faith, 1 Pet. 1, 3, which looks up to the Lord with the calm certainty that the future will bring the joyful consummation of every expectation as promised in the Word of His mercy.

But there are other qualifications that the soul-winner must strive after and must cultivate with persistent application.

One of the outstanding characteristics of the great Apostle Paul was his \emph{humility}, with an almost pathetic eagerness to efface himself for the sake of others. Like Moses, Num. 12, 3, he was meek in the extreme. And therefore he was in a position to admonish others to show and cultivate true humility in dealing with others.

``With all lowliness and meekness, with long-suffering, forbearing one another in love.'' Eph. 4, 2. Cp. Col. 3, 12; Phil. 2, 3; 1 Pet. 5, 5.

Not as though Paul had permitted any man to abuse him and his work for the Lord, for in such a case he was very emphatic in setting offenders right. Cp. 2 Cor. 10, 12-17.

His position was this, that he who gloried should glory in the Lord alone, not seeking honor for himself, but in meekness placing himself and all his talents in the service of the Lord and his fellow-man.

Such humility, then, does not in any way set aside \emph{firmness and courage}, but rather encourages and supports a definite stand on the basis of Christian liberty. It is true of soul-winners as it is of all Christians: --

``Stand fast, therefore, in the liberty wherewith Christ hath made us free and be not entangled again with the yoke of bondage.'' Gal. 5, 1.

A soul-winner cannot afford to be either foolhardy or cowardly. This is so important that it shall be discussed at greater length in a later chapter.

Humility also does not interfere with the proper \emph{self-respect}. A person who has no respect for himself and for his own soul will hardly have much for the soul of another. The fact that we are ``bought with a price'' must be brought out so strongly in our consciousness that it shows in all our bearing. The attitude of some enthusiasts of the early Church, who thought that one must despise himself and cringe in the face of the world, is as bad as a conceit which is overbearing in its behavior toward others.

A soul-winner needs the virtue of \emph{fidelity} in a high degree. This requires, first of all, a staunch loyalty to the Lord, whom he has promised faithfulness. But it requires also an unwavering devotion to the cause in which we are engaged. The matter of winning others for the Lord is not an easy task. It is usually not done with shouting and by means of mass conversions. It means faithful adherence to the plan adopted as our program, that of making our time and every talent count.

Then there is \emph{veracity}, or \emph{truthfulness}. Under no circumstances can we afford to be two-faced or even to give the impression of being double-tongued. Our cause is the essence of truth, and we have no apology to make for any part of it. We have nothing to hide and nothing to be ashamed of. St.~Paul could truthfully say of himself: --

``We spake all these things to you in truth.'' 2 Cor. 7, 14.

This implies that we ourselves make use of all \emph{sincerity}, that the truth be not only in our tongue, but in our heart. The insincere person may have a message which itself is the very acme of truth, but his duplicity will most likely become apparent sooner or later, and the result is apt to be disastrous, not only to such a person himself, but, above all, to the cause and to the message of the Gospel.

``My little children, let us not love in word, neither in tongue, but in deed and in truth.'' 1 John 3, 18.

With these characteristics as the basic virtues of the soul-winner, he will be ready to enter upon his work of love with all \emph{enthusiasm}, not with hare-brained precipitation, but with a clear-eyed optimism, with a definite reliance upon the fact that it is God's work which we are carrying on.

For that reason, however, \emph{diligence}, is also needed, a cheerful steady application to the duties imposed upon us by the obligation of love. We simply cannot afford to dawdle, to stand around idle, with the specious plea that no man has hired us. Matt. 20, 7. Our Lord has hired us, and we know that we must apply ourselves most assiduously to the task at hand; for ``the night cometh when no man can work.'' For that reason we must also observe the talent of time and be punctual in all our work. Our God is a God of order, and our Savior was punctual in both His incarnation, Gal. 4, 4, and in His death on the cross. It simply means being conscientious in the duties which the Lord has given us to perform, so that both He and our fellow-men can rely upon us at all times. There are few experiences more discouraging than to be obliged to wait at meeting for laggards who think that they have the privilege of coming late at their convenience.

A very important virtue of the soul-winner is \emph{patience}, the ability to apply oneself to a task with unflagging interest and with unremitting toil, in spite of the obstacles and difficulties which tend to make the work tedious. It is not merely that the soul-winner combines patience with faith, love, and meekness in himself, but that he applies this virtue to all the conditions of his endeavor, especially in meeting ignorance and hostility. We are workers together with God, not in the sense that He and we are a team pulling side by side, but in this, that He works through us, that He makes known the message of salvation through our efforts. God gives His Holy Spirit when and where He will.

``I have planted, Apollos watered; but God gave the increase. So, then, neither is he that planteth anything, neither he that watereth, but God, that giveth the increase.'' 1 Cor. 3, 6. 7.

Another very important point in the make-up of a soul-winner is \emph{tact}, that is, the intuitive, quick, and correct appreciation of that which is fit, proper, and right in a given situation. Tact is ever kindly and sympathetic; it never takes advantage of any form of weakness on the part of those with whom we are dealing; it wipes out differences of rank and station without diminishing the feeling of respect on the part of him who receives assistance from us. It succeeds, above all, in removing the feeling of distrust which many people have for such as discuss religious matters with them. It brings the highest topics of spiritual warfare to a self-evident plane without making them trivial. At the same time, there is no lack of true politeness and even of affability in approaching people. We must study the manner in which the Lord Himself approached the various people whom He wanted to win for the Gospel. He deals differently with John than with Peter, with the woman of Samaria than with Matthew, with the great sinner than with Zaccheus, with the thief on the cross than with Saul, the persecutor. Philip deals differently with the Ethiopian eunuch than with the people of Samaria. Paul uses a different approach in the case of Lydia and of the jailer, although both were converted in the same city.

The soul-winner requires \emph{self-denial}; here, again, first for himself, namely, in the sense which Jesus speaks of it: --

``If any man will come after Me, let him deny himself and take up his cross and follow Me.'' Matt. 16, 24

It is necessary that all boasting be excluded, that the idea of ``myself and I and my own righteousness'' be set aside emphatically and definitely.

``Ye are My friends if ye do whatsoever I command you.'' John 15, 14.

This self-denial, then, is in evidence whenever the true soul-winner approaches any one who may be gained for the truth. All thought of self is set aside, and the heart concentrates its efforts upon the one great task in hand, that of convincing the hearer of the truth of the Gospel-message.

~~Lord, help me live from day to day\\
\hspace*{0.333em}\hspace*{0.333em}In such a self-forgetful way\\
\hspace*{0.333em}\hspace*{0.333em}That even when I kneel to pray\\
\hspace*{0.333em}\hspace*{0.333em}My prayer shall be for -- OTHERS.

~~Help me in all the work I do\\
\hspace*{0.333em}\hspace*{0.333em}To be sincere and ever true\\
\hspace*{0.333em}\hspace*{0.333em}And know that all I'd do for you\\
\hspace*{0.333em}\hspace*{0.333em}Must needs be done with -- OTHERS.

~~Let ``self'' be crucified and slain\\
\hspace*{0.333em}\hspace*{0.333em}And buried deep, and all in vain\\
\hspace*{0.333em}\hspace*{0.333em}May efforts be to rise agian\\
\hspace*{0.333em}\hspace*{0.333em}Unless to live for -- OTHERS.

But all these virtues and many others -- faith, love, hope, humility, firmness, courage, fidelity, veracity, sincerity, enthusiasm, diligence, affability, conscientiousness, politeness, dignity, chastity, self-denial, punctuality, cheerfulness, sympathy, tact -- must be based on \emph{knowledge}; they are useless without such thorough understanding as comes from a diligent study of the Bible. To depend on emotions alone, on fleeting impressions, on a temporary interest, or to be stimulated by outward success alone, would be the wrong motivation and would certainly not result in a true soul-winning effort. We must all be willing, eager to study the Word of God and to grow in knowledge of our Savior and the way of sanctification. We must be willing also to attend classes organized for special training in soul-winning. We can always learn from others, especially from such as have had personal experience in the work of soul-winning. It will never do for us to insist upon going our own way when the soul's salvation of so many unnumbered people is concerned. The work is always under the guidance of those whom the Lord has called to be our teachers.

``Woe to him that is alone when he falleth; for he hath not another to help him up\ldots. And if one prevail against him, two shall withstand him; and a threefold cord is not quickly broken.'' Eccl. 4, 10. 12.

``Where two or three are gathered together in My name, there am I in the midst of them.'' Matt. 18, 20.

Let us now summarize the chief qualifications of soul-winners as others have discussed them.

Broughton says that the work side of the soul-winning Church must keep the individual burden before the minds of its members; that no one may shirk the responsibility placed upon him by the Word of God; that the so-called humble talent in the congregation ought to be developed, not merely the rich and prominent people; and that opportunities should be created with the talent at hand.

Kemp says of the qualifications of the soul-winner; 1. He must possess a deep love for the souls of men. 2. He must have an overwhelming passion for the soul's salvation. 3. He must have a deep and heartfelt conviction of the soul's worth. 4. The soul-winner requires tact in his work.

~~Lord, give us love --\\
\hspace*{0.333em}\hspace*{0.333em}The love that flows from Thee,\\
\hspace*{0.333em}\hspace*{0.333em}The love that Thou didst show for us on Calvary\\
\hspace*{0.333em}\hspace*{0.333em}When Thou Thy life laidst down to make us whole.\\
\hspace*{0.333em}\hspace*{0.333em}Let this mind be in us to fill our soul,\\
\hspace*{0.333em}\hspace*{0.333em}That we devote our life to serving Thee\\
\hspace*{0.333em}\hspace*{0.333em}And ``Laying down our life for others'' may our motto be,\\
\hspace*{0.333em}\hspace*{0.333em}That love may rise in us and grow from day to day\\
\hspace*{0.333em}\hspace*{0.333em}And we be ever guided by its gentle sway, --\\
\hspace*{0.333em}\hspace*{0.333em}Give us such love!

\chapter{The Time is Short.}\label{the-time-is-short.}

\section*{1 Cor. 7, 29}\label{cor.-7-29}

\subsection*{Salesmanship for the Lord.}\label{salesmanship-for-the-lord.}

The qualifications which we considered in the last chapter make for a strong Christian personality. The possession of any one of them or of a group of them is a fine asset to the soul-winner. But he must not be satisfied with having gained some measure of ability along one line. It is necessary for him to build up for higher efficiency. Let us group some of the positive qualities which come into consideration in working for the Lord.

Let us list some of the qualifications of the intellect, of the sensibilities, of the will, and of the spirit, or heart, as they must be highly developed and as they must be kept in the highest possible state of efficiency.

\begin{enumerate}
\def\labelenumi{\arabic{enumi}.}
\tightlist
\item
  The qualities which are important for \emph{ability}:

  \begin{itemize}
  \tightlist
  \item
    Observation
  \item
    Concentration
  \item
    Memory
  \item
    Imagination
  \item
    Judgment
  \item
    Reason
  \end{itemize}
\item
  The qualities which ensure \emph{reliability}:

  \begin{itemize}
  \tightlist
  \item
    Honesty
  \item
    Loyalty
  \item
    Sincerity
  \item
    Ambition
  \item
    Enthusiasm
  \item
    Optimism
  \end{itemize}
\item
  The qualities which are essential for \emph{leadership}:

  \begin{itemize}
  \tightlist
  \item
    Decision
  \item
    Punctuality
  \item
    Courage
  \item
    Initiative
  \end{itemize}
\item
  The spiritual qualities which are \emph{required in the soul-winner}:

  \begin{itemize}
  \tightlist
  \item
    Eagerness for growth in knowledge
  \item
    Faith
  \item
    Love
  \item
    Hope
  \end{itemize}
\end{enumerate}

Observation is needed to see and develop opportunities and to read the character of people. Concentration enables one to give the proper attention to a problem. Memory ought to be developed for the sake of remembering facts needed in soul-winning efforts. The imagination must be duly cultivated for the sake of planning campaigns and presenting possibilities. The judgment ought to be so developed that all ordinary arguments may be both employed and answered. Reason must be made the handmaid of all our work in the Church.

Honesty will often do more to convince people than arguments in themselves. Loyalty and faithfulness, together with close application, will create an atmosphere of conviction. Ambition in the interest of the Lord and His work will tend to bring about an attitude of eagerness to build the kingdom of the Lord. This will be further aided by enthusiasm and a proper optimism based on the Lord's promises.

If one cultivates the quality of quick and correct decision, he will be dependable in positions of leadership. Punctuality in all undertakings, in keeping all appointments, enables one to grasp a situation according to its general outlines before others have so much as arrived. Courage enables one to undertake even a difficult task with a determination which is half the battle. Initiative finds an opening and takes hold of problems without hesitation.

These qualities must and should be used in SALESMANSHIP FOR THE LORD.

This is necessary because THE TIME IS SHORT. 1 Cor. 7, 29.

``The end of all things is at hand; be ye therefore sober and watch unto prayer.'' 1 Pet. 4, 7.

``Little children, it is the last time.'' 1 John 2, 18.

``I must work the works of Him that sent Me while it is day; THE NIGHT COMETH WHEN NO MAN CAN WORK.'' John 9, 4.

When a sales manager plans and maps out a campaign and finds that the time for carrying out his objectives is short, he will be all the more careful about instructing his salesmen along the lines of best endeavor. He will try to communicate to every one of them the mental alertness, the eager tension with which he is himself imbued. Every movement must count; he cannot afford one moment of lost motion.

Even so, in these last days of the world, every Christian, a soul-winner by virtue of his profession, will strain every nerve to gain souls for the salvation prepared for them in Christ Jesus.

Since we are here chiefly concerned with personal work, we note, first of all, that the soul-winner must \emph{study the individual} with whom he is dealing. It is a question of observing faces intently, of determining, by the expression of eyes and features, whether a contact has been, or is being, established. A pleasant and cheerful voice is much more likely to attract than one which grates on the prosect's ear. The message which we have is to be brought with clearness, force, and elegance. Clearness appeals to the intellect; force appeals to the emotions; elegance appeals to the taste. The more we know human nature and the better we are able to analyze the emotions, the better we shall be able to influence them in a manner which will pave the way for a willing acceptance of our invitation for Christ.

We are salesmen for Christ; we are engaged in bringing the blessings of salvation to men who are in need of them; we have orders to call the attention of men everywhere to the wonderful call of the Lord:

``Ho, every one that thirsteth, come ye to the waters; and he that hath no money: come ye, buy and eat; yea, come, buy wine and milk without money and without price!'' Is. 55, 1.

With this wonderful obligation and responsibility resting upon us, we ought to be familiar with the \emph{seven mental processes} through which the mind of the missionary prospect must be taken before we can expect him to be interested in the great message of salvation.

First: He must be met; we must somehow get together with him.

Secondly: His attention must be attracted to the message which we have for him, and that in such a manner that a favorable mental impression is immediately created in the prospect's mind.

Thirdly: We must arouse the prospect's interest.

Fourthly: We must convince him that our proposition is to his advantage.

Fifthly: We must cause him, if possible, to have a desire for our message and its contents.

Sixthly: We must, if possible, add to this desire a positive resolve on the part of the prospect to possess the blessing of which we are speaking.

Seventhly: We must bring about favorable action on the part of the prospect.

Each of these processes is separate and distinct, although it may be very closely linked up with the one nearest to it. We must remember that no chain is stronger than its weakest link, and if we fail to prepare for any of these steps, the chances are that we shall lose our prospect. Let us, therefore, study these steps in greater detail and try to grasp their significance for our work of soul-winning.

First: THE PROSPECT MUST BE MET -- WE MUST SOMEHOW GET TOGETHER WITH HIM.

This step is usually designated THE APPROACH. It means that we see the person concerned, if at all possible, in person; for that is the best way for a salesman to see a customer. Even the best publicity work will not be equal to a meeting face to face or, in our case, a heart-to-heart talk. If it is a former member of the church whom you wish to see, possibly even a member of your confirmation class, make an appointment with him. If it is any other person whom you desire to win, call in person, if at all possible. Letters have usually been found to be very poor substitutes for the personal touch, for the direct approach.

When the meeting with the prospect takes place, the soul-winner must be altogether clear in his mind as to what he wants to present. If he is weak in opening the conversation, fidgety and nervous in stating what he has to say, the impression will be very bad and may spoil the whole effort. If he makes a timid start, the result will probably be the same. The first sentences should carry with them, and, if possible, produce as well, a feeling of pleasure. Flippancy is out of place, and familiarity breeds contempt. The worker should be dignified without being sanctimonious. His statements should show confidence and strength in the best sense of the word. First impressions count for very much in the business of soul-winning.

Secondly: THE PROSPECT'S ATTENTION MUST BE ATTRACTED TO THE MESSAGE WHICH WE HAVE FOR HIM.

This step is usually called GETTING THE ATTENTION, or getting over on the prospect's side of the fence. The very first thought which ought to reach the prospect's mind is this, that we have something of benefit to him, something that will interest him; that we can demonstrate to him that the proposition which we have to submit will enable him to live more happily here, with the assurance of a good conscience and of heavenly peace. Sometimes it will make a good impression if, on first meeting the prospect, we can say: ``Mr.~(or Mrs., or Miss) ---, let me assure you from the beginning that I have nothing to sell, but I know of a way of happiness which is bound to appeal to you. All men desire happiness, not the kind that lasts for only a day or two and is then forgotten, but the kind which is connected with security and safety of the most lasting kind.'' The object in which we are trying to interest the prospect is not yet mentioned, but a positive suggestion has been made which places the possibility of a benefit squarely up to the prospect. Notice that at this point we do not in any way put our own persons forward, but rather tend to eliminate ourselves as a factor. Every suggestion must be positive at this point; for a negative statement is apt to put the prospect on his guard, to place him on the defensive.

Thirdly: WE MUST AROUSE THE PROSPECT'S INTEREST.

This step is usually called, for short, AROUSING INTEREST. We keep in mind here that there are various kinds of interest. There is an involuntary interest, such as that which we feel when a bright picture registers, even momentarily, on our mind. What we want is to focus or to rivet the attention, to make the interest voluntary, to present our proposition to the prospect in a way which will cause him to listen with eager pleasure.

At this point we can well make use of curiosity, the feeling which desires to become acquainted with some new project. If we can excite the pleasure of anticipation, the chances are that the interest aroused will be of a nature to accept gratefully whatever we have to offer. The idea is to hold the prospect's attention until it ripens into a fixed interest.

Sometimes a quick shift of the approach will succeed in arousing the interest, as when a person says: ``I'm not interested in your proposition,'' and we can immediately counter with: ``But you will surely be interested if I can show you that this will be of great benefit to your children.''

Fourthly: WE MUST CONVINCE THE PROSPECT THAT OUR PROPOSITION IS TO HIS ADVANTAGE.

This step is often designated DEMONSTRATING TO THE POINT OF CONVICTION. It is not enough that we have the attention of the people to whom we are addressing ourselves in our soul-winning effort. Nor is it enough that they are interested in what we are demonstrating. Our further progress must be such as to carry conviction to their minds and hearts. If our statements up to this point are good, if we have understood the special difficulties of our prospect, if we have brought home the necessity of having our message explained, then the next little point must be gained. The prospect must yield to the extent of thinking or saying, ``Yes, I believe you are right,'' or, ``Your proposition appeals to me.''

There is no need, at this time, of making comparisons, unless people bring them in by way of argument. Our line of talk must still be positive. When challenges come, the better way is that which Philip chose. When Nathanael said to him, ``Can there any good thing come out of Nazareth?'' Philip simply repeated his invitation, ``Come and see!'' John 1, 46. The point of conviction for us is not to convince the prospect, at the first meeting, that the message which we have is the full and whole truth, but only that it is good for him to follow our invitation. Often people will say, ``Undoubtedly I ought to go to some church,'' or, ``Yes, I believe that I ought to send my children to school and Sunday-school.'' Then it is when we can say, ``Come and see for yourself!''

Fifthly: WE MUST CAUSE THE PROSPECT, IF POSSIBLE, TO HAVE A DESIRE FOR OUR MESSAGE AND ITS CONTENTS.

This step is usually called CREATING DESIRE. It is linked up very closely with the preceding step. The intellect having been brought to an understanding, the will must now be engaged. It is here that our reserve talk comes in, where we, in fact, call up all our resources to clinch the matter that we have undertaken. The prospect's curiosity having been maintained to this point, everything that is irrelevant must be most rigidly excluded, lest the entire undertaking be spoiled by a wrong move.

It is at this stage that objections will very likely be encountered. The prospect, being convinced in mind, is not quite ready to yield in action. The objections may be of a personal nature, either in deprecation of the person himself or in attack upon some member of the church. Or the objections may be of a critical nature, concerning the matter that is broached. It will take all our skill to avoid disputes at this point and to offer only clear, positive evidence of the truth which we represent. This matter will be treated more fully in special chapters.

Above all, we must not make the mistake of assuming an apologetic attitude at this stage, for that is bound to make just the opposite impression from that which we desire to convey. There is so much at stake that the slightest digression may throw the matter back to the starting-point and spoil the entire effect.

Sixthly: WE MUST, IF POSSIBLE, ADD TO THIS DESIRE A POSITIVE RESOLVE ON THE PART OF THE PROSPECT TO POSSESS THE BLESSING OF WHICH WE ARE SPEAKING.

This step may be designated briefly as DEVELOPING THE RESOLVE TO ACCEPT THE INVITATION. The desire to come has been created, the first step of the yielding has been performed. But still the prospect hesitates. There is a fear which is holding him back. He will come with the evasions of procrastination: ``I guess I won't do it just now yet,'' or, ``Oh, I don't know; there's so much to be considered,'' or, ``Maybe some day.''

This hesitancy must be overcome by all means; for if it is allowed to prevail, it will be necessary for the soul-winner to go over the whole ground again, and the will of the prospect is weakened by every new delay. One may very well meet this hesitancy in about this way: ``I appreciate your desire to give the matter further thought, Mr.---, but you are really better prepared to make a decision to-day than you will be a week or a month from now. You see, we have gone over every point very carefully. The various points are clearer in your mind than they will be later. Your good judgment tells you that to decide right HERE and NOW is the wise thing to do. It may take a little courage on your part to go ahead. You know it is the easiest thing in the world to put off making a decision. But you will surely agree with me, Mr.~---, that one of the strongest characteristics of prominent and successful men is their ability to decide and act after once making up their mind that it is the wise thing to do; and that is surely what you are going to do. Your intelligence, your judgment, tells you that this is the best thing you ought to do; your feelings and your best interests make you realize it. It is only necessary for your will to act, and that can be done by your simply acting according to your own best judgment at once.''

Seventhly: WE MUST BRING ABOUT FAVORABLE ACTION ON THE PART OF THE PROSPECT.

This step has fitly been called GETTING ACTION. It is the consummation , the climax, of the whole procedure. Without this step the whole effort is practically wasted. And, as stated above, the danger is that each new failure in the case of any one prospect diminishes the chances of winning him. It may be necessary to dispense some final information, which will help the will over the last hurdle. Above all, only affirmative suggestions are in order at this time. As soon as one says, ``I suppose you would want a little more time to think it over,'' or words to that effect, the prospect will grasp the opportunity with relief, if not with delight.

AVOID ALL NEGATIVE SUGGESTIONS; NEVER MAKE ONE DURING THE LAST STEPS OF MISSION ENDEAVOR!

Now, if we keep in mind exactly how much depends upon our work, upon the most careful application of the best principles and rules of salesmanship, we shall certainly carry out the suggestions which have proved their value through centuries of selling.

THE TIME IS SHORT! We cannot afford to waste any time in foolish experimenting.

There is one more point that may be added here for the sake of completeness. One of the features of our present missionary endeavors is the distribution of tracts and Bibles. This work is properly carried on only in connection with the personal solicitation and appeal. And it requires a number of suggestions which ought to be studied carefully and heeded without fail. The most important of these rules are the following: --

\begin{enumerate}
\def\labelenumi{\arabic{enumi}.}
\item
  Read and master the tract before distributing it to any person. It may just occur to the prospect to ask questions concerning the contents of the tract, and it would hardly make a favorable impression upon him if you should be obliged to hesitate about entering into a conversation about the topic treated in the tract.
\item
  Be sure to hand the tract to the right person. We do not distribute tracts promiscuously and indiscriminately. There are few things that so disgust a person as being given information on a point on which he is already sold or in which he has not the slightest interest.
\item
  Stamp every tract with the name and address of your church or of the pastor of the church. Unless the prospect knows where to turn when he is ready to act, the whole effort is wasted.
\item
  Carry tracts with you whenever you know you will have an opportunity to reach some one. This may well be on every trip in or out of the city. Many tract organizations are having excellent results, due to their custom of distributing tracts in street-cars.
\item
  Give tracts at the proper time, at the psychological moment, when the people, if possible, are in the right state of mind to receive just that particular information.
\item
  Never give without a suggestion to read. Sometimes it will be advisable to read a portion to the prospect, especially if he is not at all inclined to read anything of this nature.
\item
  Distribute Bibles, Testaments, etc., whenever this is possible and will have any show of success. A hundred may be given away without results, but the next copy may lead a person to Christ. The cost is small in comparison with the wonderful possibilities for good.
\end{enumerate}

And don't forget: THE TIME IS SHORT!

\chapter{Because of His Importunity.}\label{because-of-his-importunity.}

\section*{Luke 11, 8}\label{luke-11-8}

\subsection*{The Need and Power of Prayer.}\label{the-need-and-power-of-prayer.}

One of the most significant, illuminating, and stimulating facts about Jesus is that which tells us that the Savior made \emph{prayer a habit}. This is all the more remarkable if we consider that this habit on the part of the Lord is reported in such a matter-of-fact way, without the slightest indication of a false enthusiasm in the incidents or in their recital, that the impression of the account is thereby heightened. We simply find a wonderful intimacy and fellowship existing between Jesus and His heavenly Father, which found its expression in the act of prayer, not merely as a devotional exercise, but as a form of communication by and through which He derived the support and the strength which He needed for His work.

In the very first months of His public ministry in Galilee, shortly after He had made Capernaum His headquarters, Jesus,

``Rising up a great while before day, went out and departed into a solitary place and THERE PRAYED.'' Mark 1, 35.

And again we are told concerning the Savior that

``He withdrew Himself into the wilderness and PRAYED.'' Luke 5, 16.

When the Lord withdrew to the Mount of Transfiguration, it is expressly stated that He PRAYED. Luke 9, 29.

When Jesus had fed the five thousand men out in the wilderness on the northeastern shore of the Sea of Tiberias, His first act, after dismissing the multitude, is recorded by three of the four evangelists telling the story: --

``He went up into a mountain apart to PRAY.'' Matt. 14, 23; Mark 6, 46; Luke 6, 12.

Of particular interest in this connection are the accounts of the last evening of the Lord's life. Not only do we find Him speaking about prayer and instructing His disciples concerning its form and proper use, but we have at this time the incomparable high-priestly prayer with its wealth of comfort for all Christians. John 17.

And who would not think at this point of the narrative of the Lord's suffering in Gethsemane? We are told that

``He went a little farther and fell on His face and PRAYED.'' Matt. 26, 39.

A few verses farther on: --

``He went away again the second time and PRAYED.'' V. 42.

``And He left them and went away again and PRAYED the third time.'' V. 44.

Cp. Mark 14, 35. 39; Luke 22, 41.

Let us consider for a moment what this means as told of our Lord. As Broughton has it: ``It has always been very striking to me when studying the life of Jesus that, however busy He was, He was NEVER TOO BUSY TO PRAY. However closely He was beset by problems and difficulties, however great might be the press about Him, He never let the opportunity slip by to teach by example and precept the importance of prayer. I think He often left the great crowd and prayed by Himself because He wanted to teach that prayer was so very important.''

Now, Jesus was fully aware of the restriction which concerns prayer in its relation to earthly, temporal things. When His soul desired relief from the excruciating torture of a sorrow which brought Him face to face with external death, He yet bowed under the will of His heavenly Father in saying: --

``Nevertheless, not as I will, but as Thou wilt.'' Matt. 26, 39. Cp. v. 42.

But in the same measure as Jesus yielded to the will of His heavenly Father in the matter of temporal relief from the cup whose drinking could not be spared Him who had come to be the Savior of mankind. He demanded the fulfilment of His requests at the hand of His God and Father.

What matchless strength is there in His words:

``Father, \emph{I will} that they also whom Thou hast given Me be with Me where I am, that they may behold My glory which Thou hast given Me; for Thou lovedst Me before the foundation of the world.'' John 17, 24.

What boldness of speech have we here! WHAT IMPORTUNITY!

And it is precisely this IMPORTUNITY which the Lord requires of us in His work that He expects us to employ in desiring favors and assistance of Him.

``I say unto you, Though he will not rise and give him because he is his friend, yet BECAUSE OF HIS IMPORTUNITY he will rise and give him as many as he needeth.'' Luke 11, 8.

This parable illustrates the meaning of the Lord in a most effective manner, especially since it is connected with one of the occasions when the Lord taught His disciples to pray. There should be an insistence connected with our prayer which knows no failure, which simply clings to the Lord with the importunate declaration: --

``I will not let Thee go except Thou bless me.'' Gen.~32, 26.

The same fact is brought out by the Lord in His parable of the widow who was opporessed by her adversary. If even the unjust judge was constrained to say, ``I will avenge her lest by her continual coming she weary me,'' Luke 18, 5, how much more will our loving Father in heaven be ready to lend us the full measure of His divine assistance in carrying out the work which He Himself has given us to perform!

This spirit of trust in the efficacy of prayer must be ours if we desire to do effective work in carrying out the DIVINE COMMISSION. It is in agreement with the Lord's own admonitions regarding prayer. The Savior Himself bids us: --

``Ask and it shall be given you; seek and ye shall find; knock, and it shall be opened unto you. For every one that asketh, receiveth; and he that asketh, findeth; and to him that knocketh it shall be opened.'' Luke 11, 9. 10.

His promise is wide and sweeping as long as our requests are voiced in agreement with His holy will. He says: --

``Again I say unto you, That if two of you shall agree on earth as touching anything that they shall ask, it shall be done for them of My Father which is in heaven.'' Matt. 18, 19.

Gathered together in the name of the Lord, upheld and stimulated by His divine promises, there is nothing that can daunt us.

AND THIS IS PARTICULARLY TRUE OF SOUL-WINNING.

If the two component parts of soul-winning, according to Hogben, are: \emph{Going to God for sinners}, and: \emph{Going to sinners for God}, and both of them have the definite promise of God, being in full aggreement with His will, then surely there can be no flinching on our part as we go forward to perform the task assigned to us.

Again the Savior tells us: --

``If ye abide in Me and My words abide in you, \emph{ye shall ask what ye will}, and it shall be done unto you.'' John 15, 7.

Just as all-embracing, within the limits set above, is the other admonition of the Savior: --

``Hitherto have ye asked nothing in My name. Ask, and ye shall receive, that your joy may be full.'' John 16, 24.

Particularly impressive is also the statement made by the Lord in connection with the lesson connected with the drying up of the fig-tree: --

``If ye have faith and doubt not, ye shall not only do this which is done to the fig-tree, but also, if ye shall say unto this mountain, Be thou removed and be thou cast into the sea, it shall be done. And all things, whatsoever ye shall ask in prayer, believing, ye shall receive.'' Matt. 21, 21. 22.

Still more emphatic is the declaration made by the Apostle James: --

``Ye have not because ye ask not. Ye ask, and receive not because ye ask amiss.'' Jas. 4, 2. 3.

That, indeed is true: If our prayer is made in matters which do not have the approval of the Lord, with which His promise is not connected, then we can expect no results. But what is more definitely in agreement with the will of God and of the Savior than the labor of SOUL-WINNING?

In this connection we must also remember the example of the Lord's saints as told in the Bible.

What wonderful prayers are those of David, the sweet singer of Israel, both those contained in the Psalter and those found in the historical books of the Old Testament! What a beautiful example is that of Daniel, with his faithfulness in making known his needs to the one true God! Dan. 6, 10-13. How much may be learned from other believers of the Old Testament, such as Hannah, the mother of Samuel, Elisa, Hezekiah, Jonah, Nehemiah, and others!

Nor is the New Testament less emphatic in setting forth the need and the advantages of earnest prayer in matters which concern the work of the Church. The apostles time and again set forth the direct assistance in their work. When Peter and John had been imprisoned for the sake of the message which they proclaimed, the entire congregation prayed to the Lord with one accord, in a wonderful declaration of faith and trust. Acts 4, 24-30. When Peter had been imprisoned by Herod Agrippa I, the congregation of Jerusalem gathered for a meeting of prayer at the house of John Mark's mother. Acts 12, 12.

Shall we believe that the effectual fervent prayer of the righteous does not avail as much to-day as it did in those days?

Soul-winning is clearly under God's command, agreeing with His direction. How, then, could He withdraw His promise at a time when His assistance is needed as badly as ever in the history of the world?!

Of course, the work is difficult, and the responsibility is great. But it is just at this point that we have the assurance of His abiding presence of the gift of the Holy Ghost.

``If ye, then, being evil, know how to give good gifts unto your children, how much more shall your heavenly Father give the Holy Spirit to them THAT ASK HIM!'' Luke 11, 13.

``PRAYING ALWAYS with all prayer and supplication in the Spirit and watching thereunto with all perseverance and supplication for all saints.'' Eph. 6, 18.

Now, the question has been asked by many workers in the Church, just as it is being asked to-day, partly in a spirit of anxiety, partly in a spirit of doubt: --

WHY IS THE PRAYER OF SOUL-WINNERS NOT MORE EFFECTIVE?

Various answers may be given according to the Bible, and it behooves us to look well to our own hearts and the motives which appear in our work.

Perhaps, when we pray, we have not yet wholly given ourselves to the guidance of the Lord. Perhaps there is still \emph{some selfishness}, some desire for honor, \emph{some self-righteousness}, in our attitude. As long as there is some iniquity of this kind in our hearts, the Lord will not hear. Ps. 66, 18; Job 27, 9. Such a condition on our part stands in the way of our prayer; it cannot reach the ears of the Lord because of the obstacle which we place in its way. We are personally responsible for the lack of effectiveness in our petitions in such a case.

There is, in the second place, the ineffectual prayer due to LACK OF FAITH. Faith, in this connection, means trust in the promise of the Lord, and it is very definitely connected with saving faith, the trust in the forgiveness of sins and in the love of God for Christ's sake. Faith means that we are ready to undertake a thing for the Lord even if we cannot see and cannot figure out the end and outcome of the undertaking before we take the first step. God is looking for people, for workers together with Him, who are ready to take Him at His word and promise, so that He may reward their trust accordingly.

It may be, in the third place, that there is a \emph{lack}, on our part, \emph{of free surrender} to the will of God. God wants us to plan, of course. But the manner of carrying out the plan must be in agreement with His Word and promise, not with any false ambition on our part. Our projects are often tinged with human weakness and with false estimates of men. If we should insist upon doing our work \emph{only} in the way according to which \emph{we} have planned it instead of watching for the Lord's guidance and freely following every hint thrown out to us, it would be all wrong, and we should deserve to fail in our undertaking.

If these conditions obtain, our prayers are bound to be ineffective, and it is entirely our own fault. The way was open, and we did not take it. Prayer has its promises, which, when observed, achieve its purpose.

\begin{enumerate}
\def\labelenumi{\arabic{enumi}.}
\item
  Definiteness of aim. It was the singleness of purpose of Elias in his prayer which caused it to be so powerful. Jas. 5, 17. 18.
\item
  Spirit-taught desire. It is necessary that we believe in the certainty of our aim and be guided by the Spirit according to the best interests of the kingdom of God. Mark 11, 24; Rom. 8, 26. 27.
\item
  Inward purity, or the absence of self-righteous endeavor. That is what the Lord teaches in Ps. 66, 18; 1 John 3, 19-22.
\item
  Unwavering faith. ``He that wavereth is like a wave of the sea driven with the wind and tossed.'' Jas. 1, 6. 7.
\item
  Appeal in the name and for the sake of Jesus, the one Redeemer of mankind. This is a condition which is essential, as the Lord clearly says. John 14, 13. 14; 15, 16; 16, 23-26.
\end{enumerate}

WE NEED PRAYER

\begin{enumerate}
\def\labelenumi{\alph{enumi}.}
\tightlist
\item
  When we plan our soul-winning campaigns. It is true that we should and must use our intelligence and good common sense in making such plans and in preparing to carry out the endeavors connected with them. If we intend to concentrate on those who have drifted away, we shall naturally use a different approach than if we have in mind the needs of the unchurched. There is also a vast difference between the various ways of dealing with people belonging to the same class of prospects. Christ deals altogether differently with the woman of samaria than with the Syrophenician woman, with Peter than with Zacchaeus. But for that very reason we need prayer to guide our intelligent endeavors.
\end{enumerate}

``If any of you lack wisdom, let him ask of God, that giveth to all men liberally and upbraideth not; and it shall be given him.'' Jas. 1, 5.

It is strange, but true, that the ways of God often differ widely from the ways of men. In this connection we may well be reminded of the words of the Lord: --

``My thoughts are not your thoughts, neither are your ways My ways, saith the Lord. For as the heavens are higher than the earth, so are My ways higher than your ways and My thoughts than your thoughts.'' Is. 55, 8. 9. Cp. vv. 10. 11.

WE NEED PRAYER

\begin{enumerate}
\def\labelenumi{\alph{enumi}.}
\setcounter{enumi}{1}
\tightlist
\item
  When we are considering individuals. It is possible, indeed, to divide the prospects into classes, and we observe this distinction in a general way in planning our campaigns. But soul-winning, in its last analysis, is working with the individual, and it is for the individual soul that we must pray to the Lord.
\end{enumerate}

The necessity of this becomes evident in all cases where former church-members have left the fold or are at the point of doing so. It will be of assistance to us, of course, if we are acquainted, at least in a general way, with the peculiarities of the individual, so taht we may take these into account in our work. Very often the personal equation is the most important factor in the winning of a soul. Where the increasing lack of interest is due merely to negligence, we shall handle the situation differently than where some form of hostility to the Church is in any manner apparent. But the main point is that which is indicated in the Gospel of St.~Luke with regard to the incident of the denial of Peter.

``And the Lord said, Simon, Simon, behold, Satan hath desired to have you that he may sift you as wheat! But \emph{I have prayed for thee} that thy faith fail not; and when thou art converted, strengthen thy brethren.'' Luke 22, 31. 32.

Prayer of the right kind will put us into the proper mood and attitude of mind in approaching the individual, no matter what the peculiar condition which must be taken into account.

WE NEED PRAYER

\begin{enumerate}
\def\labelenumi{\alph{enumi}.}
\setcounter{enumi}{2}
\tightlist
\item
  When actually going out on our quest for souls. This is also true, no matter whether we are considering a general campaign or have some particular person in mind. A mere physical or intellectual courage alone is not sufficient; in fact, this kind of courage alone may work great harm if a person should rely upon it. But the matter of spiritual courage, of a reliance upon the promises of the Lord, is a different matter entirely. This presupposes a passion for souls such as only the power of God can give. One may expect almost anything when engaged in a systematic mission endeavor: coldness, indifference, hostility, scorn, derision, mockery, in fact, the whole scale of human weaknesses and wickednesses. In spite of such expectatinos we must go forth with unwavering determination to WIN SOULS FOR THE LORD.
\end{enumerate}

WE NEED PRAYER

\begin{enumerate}
\def\labelenumi{\alph{enumi}.}
\setcounter{enumi}{3}
\tightlist
\item
  When approaching the individual. Let us suppose that all the preliminary steps have been taken, that every move has been most carefully mapped out. We have begun our work. But here comes the facing of one whom we wish to gain for the message of salvation. It just may be that the actual opening of the conversation will relieve the strain and that an opening will be given which will enable us to bring our invitation and to have it accepted with some degree of willingness. It may be, on the other hand, that the reception which is accorded us is anything but encouraging, that, in fact, the person addressed will fly into a rage. It behooves us, under such circumstances, neither to provoke the adversary and cause a quarrel nor to beat an undignified retreat, unless the latter course is positively the only thing left to do. It may often be possible to leave at least some good literature. Many a person likes to bluster when first approached. It may be possible to find a point of contact which will establish friendly relations, if not at once, then at a later date. All the while, however, the soul-winner will be like Moses.
\end{enumerate}

``And the Lord said unto Moses, \emph{Wherefore criest thou unto me}? Speak unto the children of Israel that they go forward.'' Ex. 14, 15.

WE NEED PRAYER

\begin{enumerate}
\def\labelenumi{\alph{enumi}.}
\setcounter{enumi}{4}
\tightlist
\item
  In asking for the Holy Spirit and in commending the manner to the Lord. This was the special comfort with which the Lord sent out His disciples:
\end{enumerate}

``It shall be given you in that same hour what ye shall speak. For it is not ye that speak, but the Spirit of your Father, which speaketh in you.'' Matt. 10, 19. 20.

We are not indued with apostolic authority and gifts, which were then granted by the direct call of the Lord. but we have the means of grace to strengthen us, and we have, above all, the Word, which gives us full information concerning the way of life and concerning all other information needed for a life of faith and sanctification. With this knowledge to sustain us, we can cheerfully rely upon the help of the Holy Ghost in directing our thoughts and minds in speaking to prospects, in trying to make them conscious of the gifts of God's grace in Christ Jesus. Prayer will prove a source of power under such circumstances, and -- we are trying to

\begin{center} GAIN SOULS FOR THE LORD \end{center}

\chapter{I Am Persuaded.}\label{i-am-persuaded.}

\section*{Rom. 8, 38}\label{rom.-8-38}

\subsection*{Having the Courage of One's Convictions.}\label{having-the-courage-of-ones-convictions.}

The discussion of the present chapter is essential for the purpose of this study. It speaks of a most important part of the personal equipment of the worker for Christ. Without the qualification which it imples much of the testifying for Chirist is of an indifferent, mechanical kind, without the force that, in itself, carries the certainty of conviction.

Luther was wont to refer to a man who was not at all times ready to stand up for his convictions as a ``soft-stepper.'' He did not mean to question the sincerity of any one, but he felt that some of the fundamental principles of the truth were occasionally sacrificed on the altar of what men would like to describe as tact, but which often has its roots in a timidity not at all in keeping with the high ideals held out by the Word of God.

``I AM PERSUADED! Such was the cry of the Apostle Paul. The words well represent the fundamental thought which actuated the great missionary and witness for Christ in his life-work. From the hour of his conversion Paul stands out from the pages of Holy Writ as a man who knew no uncertainty, no vacillation.

No sooner had Paul received Baptism than we read of him: --

``Straightway he preached Christ in the synagogs that He is the Son of God.'' Acts 9, 20.

``Saul increased the more in strength and confounded the Jews which dwelt at Damascus, proving that this is very Christ.'' V. 22.

In the very first letter which has been preserved from the hands of the great missionary, he writes from the hands of the great missionary, he writes with the directness which ever characterized him:

``Our Gospel came not unto you in word only, but also in power and in the Holy Ghost and in much assurance.'' 1 Thess. 1, 5.

A few years later he wrote to the congregation at Corinth, which had been founded about the time when Paul wrote his letters to the Thessalonians: --

``My speech and my preaching was not with enticing words of man's wisdom, but in demonstration of the Spirit and of power, that your faith should not stand in the wisdom of men, but \emph{in the power of God}.'' 1 Cor. 2, 4. 5.

This from a man who confesses that, for his own person, he was with the Corinthians in weakness and in fear and in much trembling. Paul may not have been a courageous man by nature, and he seems to have been affected with some malady which hindered him from being at his best at all times. 2 Cor. 12, 7.

And yet this man, filled with the Spirit of God as he was, could speak of the ``great boldness of speech'' which he employed in writing to the Corinthians, 2 Cor. 7, 4; he could ask the Ephesians: --

``Pray for me that utterance may be given unto me, that I may \emph{open my mouth boldly} to make known the mystery of the Gospel.'' Eph. 6, 19.

It was the same man who could voice his earnest expectation and hope: --

``That in nothing I shall be ashamed, but that \emph{with all boldness}, as always, so now also, Christ shall be magnified in my body, whether it be by life or by death.'' Phil. 1, 20.

No wonder that his friend, the beloved physician Luke, could write of him: --

``Preaching the kingdom of God, and teaching those things which concern the Lord Jesus Christ, \emph{with all confidence}, no man forbidding him.'' Acts 28, 31.

What was the basis of this remarkable boldness of the great apostle? Chiefly, as he himself indicates time and again, the knowledge of his salvation grounded in the Word of Truth. All his letters breathe this certainty: --

``I know whom I have believed and \emph{am persuaded} that He is able to keep that which I have committed unto Him against that Day.'' 2 Tim. 1, 12.

This certainty is expressed also in one of the loftiest passages in all the writings of the great apostle: --

``I AM PERSUADED that neither death nor life, nor angels nor principalities nor powers, nor things present nor things to come, nor height nor depth, nor any other creature, shall be able to separate us from the love of God, which is in Christ Jesus, our Lord.'' Rom. 8, 38. 39.

What a glorious persuasion! How these words ``I AM PERSUADED!'' ought to stand out before us in appeal and admonition to impart to us some of the spirit of the great missionary!

Is it any wonder that Paul was able to face the howling mob in the court of the Temple upon the occasion of his last visit in Jerusalem? Acts 22. Does not this explain the calmness with which he answered the charges of the Jews before the tribunal of Felix? Acts 24, 10. Can we note as much as a tremor in him when he uttered his memorable cry, ``To Caesar I appeal!'' Acts 25, 11; or when he faced the pomp of Agrippa and Bernice in the same court a short time afterward, Acts 26?

And let us consider for a moment just how this conviction affected the great witness for Christ in his testimony. Not one iota of the truth was he willing to give up, no matter what the object or the provocation. Not even for one hour would he be subject to those who desired to bring him and his fellow-workers into bondage, and the entire letter to the Galatians is aglow with a holy zeal for the upholding of a conviction which was rooted in the heart of the Gospel.

Far from designating the attitude of Paul as stubbornness, we must, on the contrary, acknowledge it as an expression of a conviction which challenges the admiration, and invites the emulation, of every true worker of Christ. That blessed assurance ``I AM PERSUADED!'' made Paul the great missionary of the Apostolic Age and one of the greatest preachers of all times.

The greatest teacher of the Church since the time of the apostles was Martin Luther, the man who rescued the doctrine of justification from the oblivion of centuries, giving it once more that position in the teaching of the Church which has focused the attention of the world on Jesus Christ as the one and only Savior of mankind.

What was it that made Luther the great teacher of the Church? What was it that gave him such a power over the hearts of men? What causes us to feel the profound influence of his personality to this day?

The answer is, briefly, this: Because Luther had the courage of his convictions. Like Paul he had grown up in the midst of a religion of self-righteousness. In a greater measure than Paul he had fought the battle of acceptance in the sight of God on the basis of his own merits. From the powerful presentation of the great apostle he had learned to know the rock foundation of Christian faith: the righteousness of Christ revealed in the Gospel.

This knowledge, this understanding, gave Luther the remarkable boldness, the awe-inspiring determination, which is the outstanding feature of his character and work. The great apostle's mighty ``I AM PERSUADED!'' lived in this lowly Augustinian, making him one of the greatest heroes the world has ever seen.

Can any one imagine Luther making use of subterfuges or proposing compromises? The world has paid to Luther the tribute of its admiration on account of his fortitude in that supreme test of courage and conviction when he was cited to appear before the temporal and spiritual princes of the world in Worms. The very schoolchildren are able to repeat the words which he uttered on the 18th day of April, 1521, when he was asked to recant. Year after year thousands of hearts thrill to the ringing defiance of the great Reformer: --

``My conscience is bound in God's Word. I can or will recant nothing, since it is neither safe nor advisable to do aught against conscience. Here I stand; I cannot do otherwise. God help me! Amen.''

But some one may say, It is easy enough to play the hero when one has the great mass of the people of any country behind him and knows that popular sentiment is on his side. To this we answer that these factors were not very strong at the beginning of the sixteenth century, when princes, kings, and emperors ruled with absolute power over their subjects. But even if we should be willing to make this concession, we have other incidents from the life of the great Reformer which show that the courage of his convictions controlled his actions to the exclusion of every selfish consideration.

If ever, during the entire period when the reformatory movement swept through Europe, it would have been a matter of expediency to have the Protestants offer a united front to the forces of the Church which was at just that time shutting itself out from the truth and going down to the level of a stubborn sectarianism, it was when Philip of Hesse arranged the Colloquy of Marburg, the first days of October, 1529. No one would have been more gratified at a union of all Protestant bodies than Luther himself. Because he knew that

\begin{center} THERE CAN BE NO REAL UNION WITHOUT A UNITY ON THE WORD OF GOD! \end{center}

In fourteen points an agreement was reached. But in the fifteenth point the ``different spirit'' of the Reformed theologians became evident, a spirit which would sacrifice the Word of God for the sake of man's reason. To have yielded at this point would have meant to set aside the conviction based upon a clear understanding of the Word of God. Therefore Luther fortified himself by writing the very word of the Bible (IS) concerning which the controversy was being waged on the table before him. And then he held out for the truth, not with conceited stubbornness, but as a champion of the unadulterated oracles of God.

The mighty ``I AM PERSUADED!'' would not permit him to act differently.

Shall not we learn from Luther?

~~We speak of the \emph{faith} that lived in his heart,\\
\hspace*{0.333em}\hspace*{0.333em}Which caused him from popery's night to depart,\\
\hspace*{0.333em}\hspace*{0.333em}Which showed him how futile the precepts of men,\\
\hspace*{0.333em}\hspace*{0.333em}Which opened the way to God's mercy again;\\
\hspace*{0.333em}\hspace*{0.333em}We praise him for preaching this saving faith,\\
\hspace*{0.333em}\hspace*{0.333em}Confessing his Savior with his last breath. --\\
\hspace*{0.333em}\hspace*{0.333em}But where would the Church of Luther be\\
\hspace*{0.333em}\hspace*{0.333em}If Luther had been like you and me?

~~We tell of the \emph{firmness} that Luther showed,\\
\hspace*{0.333em}\hspace*{0.333em}Though dark was the outlook and stormy the road;\\
\hspace*{0.333em}\hspace*{0.333em}On God's own Word his conviction was based,\\
\hspace*{0.333em}\hspace*{0.333em}And though he earth's mightiest ruler faced,\\
\hspace*{0.333em}\hspace*{0.333em}His heart refused from the truth to be led:\\
\hspace*{0.333em}\hspace*{0.333em}``I cannot do otherwise,'' boldly he said. --\\
\hspace*{0.333em}\hspace*{0.333em}But where would the Church of Luther be\\
\hspace*{0.333em}\hspace*{0.333em}If Luther had been like you and me?

~~In awe we whisper of Luther's \emph{work},\\
\hspace*{0.333em}\hspace*{0.333em}For never he knew what it meant to shirk;\\
\hspace*{0.333em}\hspace*{0.333em}His life was spent in serving the Lord,\\
\hspace*{0.333em}\hspace*{0.333em}And daily he delved in the life-giving Word;\\
\hspace*{0.333em}\hspace*{0.333em}No task was too great and no study too hard,\\
\hspace*{0.333em}\hspace*{0.333em}He cheerfully toiled, with no thought of reward. --\\
\hspace*{0.333em}\hspace*{0.333em}But where would the Church of Luther be\\
\hspace*{0.333em}\hspace*{0.333em}If Luther had been like you and me?

Yes, indeed, where would it be? And where is it going to be in the future, unless we take up this business of SOUL-WINNING with the same force of the courage of our convictions that we admire in Luther?

The Lord asks us to have this courage of our convictions and to carry forward His work with all boldness. He does not want us to be

``Children, tossed to and fro and carried about with every wind of doctrine, by the sleight of men and cunning craftiness wherewith they lie in wait to deceive.'' Eph. 4, 14.

The Apostle Paul censures the Corinthians severly on account of their lack of knowledge and conviction in the matter of their faith: --

``And I, brethren, could not speak unto you as unto spiritual, but as unto carnal, even as unto babes in Christ. I have fed you with milk and not with meat; for hitherto ye were not able to bear it, neither yet now are ye able.'' 1 Cor. 3, 1. 2.

And the writer of the Epistle to the Hebrews speaks just as emphatically in reproving his readers for their deficiency in this respect: --

``When for the time ye ought to be teachers, ye have need that one teach you again which be the first principles of the oracles of God; and are become such as have need of milk and not of strong meat\ldots. Strong meat belongeth to them that are of full age, even those who by reason of use have their senses exercised to discern both good and evil.'' Heb. 5, 12-14.

There can be no question regarding the Lord's will. He wants people as soul-winners who have the knowledge of salvation from His Word and are willing to stand up for their convictions. Like a trumpet-call the admonition of Peter rings out: --

``Be ready always to give an answer to every man that asketh you a reason of the hope that is in you with meekness and fear.'' 1 Pet. 3, 15.

This does not mean that we are at any time to brazen it out when we are attacked, but that we be informed and that we stand on this information. Jude also writes: --

``Ye should earnestly contend for the faith which was once delivered unto the saints.'' V. 3.

This is no small matter, but it is a sacred charge, a commission which no believer can evade, as he values the Word of his heavenly Father and of his Savior Jesus Christ.

\begin{center} EARNESTLY CONTEND!  BE READY!  STAND UP FOR JESUS! \end{center}

~~Stand up, stand up for Jesus!\\
\hspace*{0.333em}\hspace*{0.333em}The trumpet call obey;\\
\hspace*{0.333em}\hspace*{0.333em}Forth to the mighty conflict\\
\hspace*{0.333em}\hspace*{0.333em}In this His glorious day.\\
\hspace*{0.333em}\hspace*{0.333em}Ye that are men now serve Him\\
\hspace*{0.333em}\hspace*{0.333em}Against unnumbered foes;\\
\hspace*{0.333em}\hspace*{0.333em}Let courage rise with danger\\
\hspace*{0.333em}\hspace*{0.333em}And strength to strength oppose.

Give the Lord the best that is in you! Go forth in the strength of His might!

This does not mean that we are going to face the world in a challenging manner and with a chip on our shoulder. It means, rather, that we take up our business of soul-winning with a calm assurance growing out of a certain knowledge of salvation as revealed in God's Word, an assurance which in itself is bound to carry conviction.

Lutheran soul-winners must carry with them this conviction, the definite and invincible assurance, that the Word of God is the truth and that the salvation which it teaches is the one and only road to heaven.

Lutheran soul-winners must carry with them the conviction, the calm certainty, that the Lutheran Church, if really worthy of the name, teaches the doctrines of the Bible in their full truth and administers the Sacraments in agreement with Christ's institution. If this conviction does not live in us, we cannot be true soul-winners.

We do not mean to say at this point that our Lutheran Church in its outward form is without a flaw. There was once a sect which demanded that there be not the slightest foible or weakness in the outward appearance and life of the Church and of all its members. But this demand was rightly discountenanced by the teachers of the Church as being out of harmony with the manner in which the Bible speaks of the outward organization of what we call the visible Church.

But while we must and do admit that there are many things in the outward form of the Lutheran Church which are not in full agreement with the highest demands of sanctification as found in the Bible, that the Church as such, as well as its various members, fall short in their realization of the highest ideals as found in the Bible, we do maintain that, by the grace of God, the doctrine of the Evangelical Lutheran Church is in complete harmony with the revealed will of God, with His Word.

From this it follows that no Lutheran soul-winner will ever be found ashamed of his Church; for that would mean being ashamed of Christ, by whose mercy we have the truth in all its purity in our Confessions and in our teaching. As workers together with God and with His holy apostle we say:

``I am not ashamed of the Gospel of Christ; for it is the power of God unto salvation to every one that believeth.'' Rom. 1, 16.

~~Ashamed of Jesus! Just as soon\\
\hspace*{0.333em}\hspace*{0.333em}Let midnight be ashamed of noon.\\
\hspace*{0.333em}\hspace*{0.333em}'Tis midnight with my soul till He,\\
\hspace*{0.333em}\hspace*{0.333em}Bright Morning Star, bids darkness flee.

~~Ashamed of Jesus! that dear Friend\\
\hspace*{0.333em}\hspace*{0.333em}On whom my hopes of heaven depend!\\
\hspace*{0.333em}\hspace*{0.333em}No; when I blush, be this my shame,\\
\hspace*{0.333em}\hspace*{0.333em}That I no more revere His name.

The soul-winner who is thus equipped will, as a matter of conviction, be \emph{opposed to unionism}. We are fully aware of the fact that the Lord has those that are His own whenever His Word is still taught, even if this be not in full purity. There are true children of God in sectarian congregations as long as the Bible is still used in their midst.

But from this fact it does not follow that we can fellowship with sectarian churches, or that we can gloze over the differences in doctrine which separate the various Christian church-bodies. The words of our Savior are too clear and too direct to be set aside.

``Beware of false prophets, which come to you in sheep's clothing, but inwardly they are ravening wolves.'' Matt. 7, 15.

``Be ye not unequally yoked together with unbelievers. For what fellowship hath righteousness with unrighteousness? And what communion hath light with darkness?\ldots{} Wherefore come out from among them and be ye separate, saith the Lord, and touch not the unclean thing.'' 2 Cor. 6, 14. 17.

We may very well, for our own person, have the conviction that a certain member of another church is a true Christian, and his confession may, in every respect, agree thereto. But as long as such a person is still a member of a false church-body, that is, of one which openly holds a doctrine or doctrines not in agreement with Scripture, then we cannot fellowship with such a person in public worship as long as unity of doctrine has not been openly established.

The word is too plain: ``TOUCH NOT THE UNCLEAN THING!''

Let us also not be deceived by specious arguments for unionism. It has been said, All the Christian church-bodies want to get to the same heaven; therefore let us forget all differences and live in harmony and peace. But there is a big difference between desiring to get somewhere and getting there. One may want to get to Chicago from St.~Louis, but if he starts out in a southeasterly instead of a northeasterly direction, all his good intentions will avail him nothing.

``Can two walk together except they be agreed?'' Amos 3, 3.

We must watch particularly with regard to our own church connection. While one may get to heaven with a minimum of the Christian truth, such as may be offered in many Christian denominations, the fact that the Lutheran Church holds the maximum of truth makes it imperative that we Lutherans be satisfied with nothing but the full and complete truth.

If a person has all his life drunk nothing but water with some solution of poison in it, his body may have become immune to the poison or may be able to throw off its evil effects. But one who knows pure water cannot deliberately choose such as contains some poison. If a person has two glasses of water before him, one strictly pure, the other with just enough pioson in it likely to kill him, he would be committing suicide if he drank the latter.

Thus it is with us. We have the truth, the full truth. We cannot afford to be satisfied with less than the full truth. Moreover, the conviction of our glorious possession must live in us and emanate from us, not in a spirit of intolerance and fanaticism, but of a certainty resting in the Word of God. If we go out as soul-winners with this attitude to guide us, then the conviction living in our hearts will in itself carry conviction to others; for they will not easily withstand the impression of the holy persuasion with which we go about

\begin{center} OUR FATHER'S BUSINESS! \end{center}

\chapter{By All Means Save Some!}\label{by-all-means-save-some}

\section*{1 Cor. 9, 22.}\label{cor.-9-22.}

\subsection*{Meeting the Unchurched.}\label{meeting-the-unchurched.}

BETWEEN SIXTY AND SIXTY-FIVE PER CENT. OF OUR TOTAL POPULATION HAS NO CHURCH AFFILIATION!

Read that sentence again, for we shall consider it once more in connection with the object of the present chapter. It is a fact which, somehow, must sink into our consciousness by degrees. It means that an average of six or seven out of every ten persons whom we see on the street, whom we meet on our travels, with whom we deal in a business way, ARE NOT EVEN NOMINALLY CHRISTIANS!

This consciousness is bound to work in us the attitude of St.~Paul when he wrote, by the inspiration of the Holy Ghost, the significant words: --

``I am made all things to all men that I might BY ALL MEANS SAVE SOME!'' 1 Cor. 9, 22.

The connection in which we find these words is characteristic of the apostle. We find him, in the first place, asserting with all vehemence his freedom. No apostle has emphasized this factor more strongly, as when he calls out to the Galatians: --

``Stand fast, therefore, in the liberty wherewith Christ hath made us free and be not entangled again with the yoke of bondage!'' Gal. 5, 1.

If any one was foolish enough to believe his good works were sufficient to earn anything for him in the sight of God, Paul was invariably the first one to disillusion him.

And yet, by a strange apparent paradox, Paul writes: --

``Though I be free from all men, yet have I made myself servant unto all THAT I MIGHT GAIN THE MORE!'' 1 Cor. 9, 19.

It is the wonderful truth which Luther also emphasized when he said a Christian is ``free from all and no man's servant, but that he is, at the same time, subject to all and every man's servant.''

In the field of justification there is only one fact that stands out, namely, the unmerited grace of God in Christ Jesus, our Savior. In the domain of sanctification the highest ideal is that set forth by Christ: ``Whosoever will be great among you, let him be your minister; and whosoever will be chief among you, let him be your servant.'' Matt. 20, 26. 27.

This was the guiding principle of Paul's life, as he briefly shows in describing his manner of work. To the Jews he became a Jew in order to win the Jews. Without denying or setting aside one word of the eternal Truth, he accommodated his methods to the circumstances in which he found himself, always with the intention of winning souls for Christ. He did this in the case of Timothy, Acts 16, 3, in that of the Jews at Jerusalem, since some of the weaker Christians in that city were becoming suspicious of his work, Acts 21, 23ff., and in other instances. Wherever Paul found that he could yield to weakness and to present lack of information without sacrificing the eternal Truth, he was very ready so to do.

To those under the Law, whether they belonged to the Jewish nation or to the Gentiles (mainly circumcised Gentiles), he became as one under the Law in order to gain those under the Law. He was willing to conform to the customs, modes of life, and methods of instruction in vogue among them as long as these matters were really things indifferent. At the same time he never denied the truth or acted in a two-faced manner.

To those without the Law, that is, to the heathen in the strict sense of the word, he became as without the Law, although, as he says, for his own person he was bound under the Law of Christ and was ever eager to do His will in order to gain those without the Law. When in a heathen community, Paul did not practice the Jewish customs in which he had been trained, for this would merely have antagonized the Gentiles; he omitted all reference to regulations of the Old Testament which were strictly Jewish in character.

To the weak the apostle became weak in order to gain the weak. His loving insight and gentle tact enabled him to understand the scruples and weaknesses of those who had not as yet made much headway in Christian knowledge. If the fundamental principles of redemption were not subverted, Paul was ready to exercise patience and kindness to the highest degree. And all this he did because the love of Christ was the motive for all his actions. He was possessed of a life of the spirit implanted in his Savior and anxious to demonstrate itself in the service of his fellow-men everywhere.

We, as WORKERS TOGETHER WITH GOD, must learn from Paul, the great missionary, not to despise any one nor to permit disgust over foolish weaknesses and over lack of spiritual information and knowledge to enter our hearts. It is a wonderful summary which we find here: --

``TO ALL MEN I HAVE BECOME ALL THINGS IN ORDER BY ALL MEANS TO SAVE SOME!''

This is the spirit that appears in the history of missions, especially in that of the early Church. It explains the consecration with which men and women bore witness to the Gospel of Christ. Their attitude is well exemplified in the case of Peter and John and their challenging statement: --

``Whether it be right in the sight of God to hearken unto you more than unto God, judge ye. For WE CANNOT BUT SPEAK THE THINGS WHICH WE HAVE SEEN AND HEARD.'' Acts 4, 19. 20.

This spirit explains also the humility which we find in the early workers, with St.~Paul at their head, who exclaimed in a passionate outburst: --

``Unto me, who AM LESS THAN THE LEAST OF ALL SAINTS, is this grace given that I should preach among the Gentiles the unsearchable riches of Christ.'' Eph. 3, 8.

Now let us return to the first statement of the present chapter. Let us try to realize the fact that

\begin{center} BETWEEN SIXTY AND SIXTY-FIVE PER CENT. OF OUR ENTIRE POPULATION \end{center}

is not even nominally Christian. This means, according to the Religious Census of the United States, that so many people in our country do not profess to be members of any so-called Christian church-body, neither Protestant (in the widest sense of the term) nor Catholic. It means that they have no connection with any Church of any type, except, perhaps, that their children occasionally attend some Sunday-school.

It seems almost unbelievable that this should be so, that the many thousands and even millions of our fellow-men and fellow-citizens should declare that they are not personally interested in the Christian religion in even its broadest aspect; but the figures are available for the entire United States, also in the annual figures furnished by Dr.~Carroll. It has been stated that between fifteen and twenty million children in our country are without any instruction in the truths of Christianity.

What is most surprising in the situation, perhaps, is this, that figures coming from rural and semirural districts are not materially different in their totals from those of the larger cities. Many a small city, even many a town, shows the same proportion of unchurched as the large cities. It seems strange that hundreds and thousands of people can daily hear church-bells, see church-spires, pass by churches which extend their invitations in various ways, and yet never make a move to investigate the claims of the Church as being an institution for soul-winning.

Now, it has been said by some church-members that people in this country who are born, grow up, live, and die almost within the shadow of a church-steeple have only themselves to blame if they are lost. This is true enough from God's standpoint. He expects all men, even on the basis of the slight remnant of the natural knowledge of God (Rom. 1, 18ff.) to ``seek the Lord, if haply they might feel after Him and find Him.'' Acts 17, 27.

But this is God's privilege, God's perogative. It does not excuse us from doing the work laid upon us by the DIVINE COMMISSION when the Lord tells us: ``GO YE!''

The UNCHURCHED PEOPLE of America, of any country, are a CONSTANT CHALLENGE TO ALL TRUE BELIEVERS! We simply MUST go out and meet them!

We must meet the OUT-AND-OUT HOSTILE. We must remember, in all our work for the Lord, that ``the carnal mind is enmity against God,'' Rom. 8, 7, that is, all men by nature are in a state of enmity, or hostility, against the Gospel. No man can by his own reason and strength be interested in the Gospel as the message of salvation. But to this natural aversion against the doctrine of full and free redemption is often added a positive hostility, which aims at the direct destruction of the Gospel news and of its witnesses.

We meet people of this type even without special effort on our part, and we shall certainly meet many more if we engage more generally in specific mission endeavors. Have you ever tried to sound people with whom you are associated in business, with whom you are working together, whom you meet time and again in a social way? It is so easy, with the modern form of church publicity, to establish a point of contact, either for positive or for negative results. One has but to exhibit a pulpit program, a Lenten folder, or some other printed material referring to church activity, with the question, ``This will probably interest you?'' in order to get some sort of reaction.

In the case of men and women who have grown up without church connection the result of even a casual remark may often yield decided results in establishing a point of contact. The opening may lead to a discussion of church-work and of church activities, which may result in WINNING a soul for Christ. The thing to do is to overcome the diffidence, the lack of courage, which is our besetting weakness.

The most difficult people belonging to this group are such as have fallen away from church; for these are often filled with a bitterness and a hatred for everything connected with church and church activities which amounts to an obsession. The task is made more difficult in this case by the fact that personal matters often enter into the complication of circumstances, and the people concerned are not in a condition to discuss church-matters with any probability of blessing to themselves.

What we try to do, if this is in any way possible, is to place the suggestion of church in the minds of those with whom we come in contact here. If we can get enemies of the Gospel merely to listen to an invitation to come to church or to accept a written program of church services, something, at least, is gained. It is necessary to be careful, of course, lest we cast the pearls of the Gospel before spiritual swine. If a number of people are present and all of them are in a mood for mockery, we should, indeed, confess our faith, but it would be foolish to provoke a string of mockery which will not serve the cause of the Gospel. In such cases, however, a word of warning, such as we find in the Bible, may well be left in the hearts of the enemies, such as the statement of Stephen: ``Ye do always resist the Holy Ghost,'' or the reference of St.~Paul to the stubbonness of the human heart: ``Neither murmur ye as some of them also murmured,'' 1 Cor. 10, 10, or that impressive warning: --

``Harden not your hearts, as in the provocation, in the day of temptation in the wilderness.'' Heb. 3, 8.

We must meet the INDIFFERENT. This is a difficult class to deal with. It is hart to mold a statue out of treacle. When people simply assume the position that they are not interested and do not care to become interested, it is hart to make an impression. It is here that salesmanship for the Lord counts for very much. In fact, if there is such a thing as concentrated ingenuity or its equivalent, it is needed at this point. Two points may be remembered in this connection.

The first is that our invitation must be \emph{attractive}. Just as we try to give a certain appeal to our houses of worship and to the outward form of our worship, in line with the highest treasures of our liturgical and hymnological heritage, so we, in the thoroughly dignified manner and yet with the idea of arousing interest, present our printed invitations. The proclamation of the Word of God is the center, the climax, of our church services. But if an invitation carefully worded and tactfully presented, will arouse some unchurched person to such a state of interest as to bring him to church and cause him to hear the Gospel, the effort is not only justifiable, but it is to be commended most highly.

The second point is that of \emph{persistence}. A person with whom we are dealing may promise five, ten, twenty times to come and always forget or let his indifference guide him. But it just may be, under the Spirit's gracious sway, that he will come the sixth or eleventh or the twenty-first time. When Monica, the pious mother of Augustine, was ready to despair because she had, for so many years, tried to gain her son for Christ and had failed, she was very properly consoled with the words: ``It is impossible that a son regarding whom so many prayers have arisen be lost!'' Her persistnce won the day.

We must meet the ``BLUE-DOMERS.'' These are the people who are influenced falsely by the present movement ``back to nature.'' The movement has much to commend it; for it tends to bring people out into the open, to God's great out-of-doors, into the fresh air, the uncontaminated sunshine, where they may observe some of the miracles of God's fatherly hand at first hand and get some appreciation of the goodness of Him who ``maketh His sun to rise on the evil and on the good and sendeth rain on the just and on the unjust.'' Matt. 4, 45.

But there is a great danger connected with the movement, and that danger is becoming more apparent with the increasing number of motorists. Even church-members and former church-members are being infected with the automobile fever over the week end; how much more such as have never been interested in church! Since many of these have adopted the phrase about ``worshipping the Creator under the \emph{blue dome} of heaven,'' they have been designated ``blue-domers.''

Now we are altogether ready to concede that it is possible to worship God out in the open. Many of the mission-festivals which are celebrated out in the open have brought rich returns in blessings. But the fact is that the ``blue-domers'' are not thinking of services with the preaching of the Word out in the open, their interest being entirely in the ``open,'' and not at all in the ``worshipping,'' although they may occasionally exclaim over the beauty of some landscape.

In speaking to people of this type, it is necessary to emphasize the means of grace which the Lord has given to men, and to mention the advantage of using these means under the most orderly circumstances. This does not imply an immense cathedral with all the appointments which the love and the wealth of the members can afford; for one may serve the Lord just as well in a small chapel, in a store-room, or in a dug-out. But the Lord's will with regard to assemblies of Christians for the sake of hearing His Word is very plain: --

``Where two or three are gathered together in My name, there am I in the midst of them.'' Matt. 18, 20.

``Not forsaking the assembling of ourselves together, as the manner of some is.'' Heb. 10, 25.

``He that is of God heareth God's words.'' John 8, 47.

``Blessed are they that hear the Word of God and keep it.'' Luke 11, 28.

We must meet the spiritual FREE-LANCES, that is, the people who flit from one church to another, without ever becoming members anywhere. In the parable of the fourfold soil the Lord speaks of seed that falls upon stony places, where there was little earth.

``And forthwith they {[}the seeds{]} sprang up because they had no deepness of earth; and when the sun was up, they were scorched; and because they had no root, they withered away.'' Matt. 13, 5. 6.

The Lord here speaks of those who believe for a short time. We may well apply His words also to such as never take root in Church. Many of these unchurched people are like the Athenians of Paul's day, who were ever on the alert to hear some new thing. Acts 17, 21. They are guided by the desire to hear some alleged pulpit orator, or to be found in some beautiful modern temple of worship, or to be associated socially with the more prominent people in the community. They consider the churches like so many spiritual restaurants, but they never commit themselves definitely to any one of these houses.

It is usually not difficult to interest such people in church matters or church activities as such. The difficulty is to have them come out of one way or the other with a definite platform of belief and faith. Their behavior is not in agreement with the Word of God, which clearly speaks of congregations or parishes, which expects us to identify ourselves with a church of the pure Word and Sacraments. His command is clear: --

``Obey them that have the rule over you and submit yourselves; for they watch for your souls as they that must give account.'' Heb. 13, 17.

Sometimes it is possible to reach prospects of this kind with an argument concerning the breaking down of character on account of such practices. It is not the person who changes opinions and views like his clothes who will command respect, but the man with definite ideas.

``It is a good thing that the heart be established with grace.'' Heb. 13, 9.

The most difficult group, in a certain way, is that of the HESITANT. There are some people who always just about make up their minds to be church-members, but draw back just at the last moments, usually with some sort of flimsy excuse. Sometimes this is due to vacillation, pure and simple, and then it may sometimes be necessary ``to make up their minds for them.'' In other words, we must watch for the opportune time and then push the idea of joining to its consummation. It is just at this point that the last step in salesmanship is important.

It may be advisable, for the sake of just such timid souls, to refer prospective members, not to an ``adult catechumen class'' or even to a ``church-membership class,'' but to a ``class for the discussion of Bible truths,'' or to the ``Bible searchers,'' or even to the ``Bible class.'' Since it is a fact that some unchurched people shy at the name ``catechumen class,'' why should we not meet them halfway by simply choosing a name similar to some of those just suggested?

Sometimes the hesitation is due to the fact that a person is secretly addicted to some sin which he finds himself unable to combat successfully. Very often just such a person needs the advice and the assistance of a consecrated Christian, possibly of a pastor, who can impart such strength from Scriptures to the unfortunate prospect as to cause him to divide definitely for church-membership and all its privileges.

Above all, the SOUL-WINNER will, at this point, study ever more willingly and diligently the great examples of our Savior in dealing with men and women with whom He came in contact. Of the frankly unchurched with whom He dealt we have the woman of Samaria, the Syrophenician woman, and the demoniac of the Gadarene region. Our main line of attack will always be the invitation, issues in thousands of forms and with patient repetition: --

``COME AND SEE!''

And let us never forget to put the matter into the hands of God, who alone is able, through His Holy Ghost, to change the unwilling enemies to willing disciples.

\chapter{Patient Toward All Men.}\label{patient-toward-all-men.}

\section*{1 Thess. 5, 14.}\label{thess.-5-14.}

\subsection*{Meeting Objections of the Wrongly Informed.}\label{meeting-objections-of-the-wrongly-informed.}

As WORKERS TOGETHER WITH GOD much of our work will naturally concern the unchurched; in fact, this is the only part of our work in which we can be aggressive, in which we can and should take the initiative.

Lutheran soul-winners are not proselyters. In all our work we follow the admonition of the apostle:

``LET NONE of you suffer\ldots as a BUSY-BODY in other men's matters.'' 1 Pet. 4, 15.

We find that the Apostolic Church was ever most careful in avoiding the offense, according to which men break into the herd of another and steal his sheep. The transgression is commonly known as ``sheep-stealing'' and is rightly detested by all such as uphold the dignity of the Church and the rights of the Lord's servants. The demand of the Lord is simply this, that we should not take the initiative in getting people to join our church who are members of any Christian denomination, not only of another congregation of our own church, but of any church of which we have evidence that it is still rightly designated as a Christian church. This means, generally speaking, acceptance of the Bible as the Word of God, belief in the Triune God of Scriptures and in Jesus Christ as the only Redeemer of mankind, and the use of the means of grace in public worship. In this sense we regard all the Reformed bodies (Presbyterians, Eposcopalians, Baptists, Methodists, Christian Reformed, etc), with the exception of the Campbellites, as well as the Catholics, as Christian bodies. But we do not so regard the Unitarians, the Universalists, the Mormons, the Christian Scientists, the Spiritualists, and many other bodies of the same kind, for all of these deny the Trinity of the Bible.

With regard to members of non-Christian bodies the same rules are observed as apply to the unchurched. With regard to such as are members of a body which causes us to regard it as Christian we follow the rule of Holy Writ which states that we are not to interfere, not to take the initiative in getting their members to join our Church. It is most fortunate, as one Church Father puts it, that ``the ears of the hearers in such denominations are often purer than the lips of their teachers''; that is, by the grace of God and the enlightenment of the Holy Spirit, many false statements made by their teachers are stripped of their falseness in being transmitted to the hearers, or many of the latter rely upon the Word which they study at home and promptly forget the incorrect teaching which was brought to them in their own church.

At the same time we welcome such people if they, of their own initiative, come to our services or approach us in their search for the full truth. Since, by the grace of God, we are in possession of the full truth in Christian doctrine, we are UNDER THE OBLIGATION OF LOVE to impart this blessing to all who seek it. That is the object of all our advertising, to call the attention of men to the saving truth which we possess and which should by us be made known to others. Even if such people are, at first, only visitors at our services, we bid them a cordial welcome, asking them to judge for themselves whether we teach the full truth of the Bible or not. Of the people of Berea it is stated with approval: --

``These were more noble than those in Thessalonica, in that they received the Word with all readiness of mind and searched the Scriptures daily \emph{whether those things were so}.'' Acts 17, 11.

We take the same attitude with regard to those who approach us privately or who broach the subject of religion and of Christian doctrine in a conversation anywhere. This gives us an opportunity to testify for the truth, and our testimony may prove the entering-wedge for the Gospel-message in all its purity.

The Apostle Paul indicates to us in what spirit the work with the wrongly informed should be carried on.

``Now we exhort you, brethren, warn them that are unruly, comfort the feeble-minded {[}faint-hearted{]}, support the weak, BE PATIENT TOWARD ALL MEN!'' 1 Thess. 5, 14.

The spirit of Paul was the spirit of Christ. Jesus also hated sin, but loved the sinner, not with a weak sentimentality, but with an unspeakable yearning for the salvation of his soul. And therefore we, too, must let the same mind dwell in us. While we are impatient for the glory of the Lord and jealous for the full truth of the Bible, we are PATIENT WITH THE WEAKNESS AND THE DULNESS of men in spiritual matters.

We have the background of hundreds of years of sound Gospel-preaching; we have the background of a thorough indoctrination, preferably in the Christian day-school; we have the background of church customs which are hallowed by centuries of use and have a significance which has endeared them to us.

But now people come to us who are wrongly informed concerning many of these facts. Others begin to argue with us on the basis of a false prejudice. And very contradictory statements indeed are those that come to our attention. One says: ``Why, you Lutherans are just like the Catholics; you have altars in your churches, and you use the cross in your services.'' Others say: ``You Lutherans are just like the Presbyterians, or like the Methodists -- you are so strict with regard to worldly amusements.'' The number of objections which have been voiced might be continued almost indefinitely.

A Lutheran SOUL-WINNER ought to be in a position to meet such objections; for it is not at all difficult so to do, especially if one uses a little common sense.

If the objection concerns our Christian DAY-SCHOOLS, we can readily point to the fact that hundreds of the foremost educators of our country have declared that an education without religion, without the Word of God, is deficient in an essential point. We are merely following the injunction of the Word of God in indoctrinating our children. We may, in this connection, make good use of various tracts published by Concordia Publishing House and by the American Lutheran Publicity Bureau.

If people find it strange that we have altars in our churches, we ought to point out to them that the altar in our churches is nothing but a table for the celebration of the Eucharist. The pictures and statues which we have on our altars are not used for idolatrous purposes, and everything that is in itself objectionable in such pictures has been eliminated. We ought to be acquainted, in this connection, with the labors of Luther to purify the services of the Church. This information is available in some of our own publications.

If people object to the use of the cross on our churches and as a token of remembrance, reminding us of the bitter suffering and the vicarious death of our Savior, we ask them whether they are ashamed of this token and its significance. We ascribe no magical power in the sign of the cross, but use it merely as a remembrance of the miracle of our redemption. Fortunately the senseless opposition to the use of the cross is disappearing, and we find it in use on many Reformed church-buildings.

If people object to the use of gowns by our pastors, we can readily point out to them that our pulpit gowns have been retained with a very good liturgical reason. The pastor, in his official position, especially as preacher and celebrant of the Holy Communion in the church, is acting not for his own person, but by virtue of his call and as the representative of Christ. Therefore even his ordinary clothes are covered and his own person is eliminated as much as possible. No individual performance is permissible in a Lutheran church, and therefore even the messenger of the Lord indicates by his gown that he declares his message to be: ``Thus saith the Lord.''

If people object to our unequivocal stand with regard to worldliness and carnal amusements, we are in a position to tell them that our Church does not think of condemning anything that is permitted in Scriptures. In the prohibition discussion and many other topics which have agitated the minds of church people in this country and elsewhere we were able to maintain a calm aloofness simply because we adopted the standpoint of the Bible.

There may be many another point which we will be injected into a discussion by the wrongly informed with regard to outward forms and ceremonies or with regard to the so-called position of the Lutheran Church on questions agitating the public mind, but in practically every case the well-informed Lutheran soul-winner will be able to meet the objection or at least parry the attack until he has gained further information for himself.

HENCE THE NEED OF CLASSES FOR THE INFORMATION AND INSTRUCTION OF SOUL-WINNERS. \emph{Let everything be done under the auspices of the congregation -- if possible, under the personal direction of the pastor, who possesses all or much of the information needed or knows where he may get additional information on short notice!}

But there is another class of people to be considered here, namely, those who are WRONGLY INFORMED with regard to the DOCTRINES OF THE BIBLE. Some of these are seeking the truth and come to us with open minds. Others adopt a more or less challenging attitude, which demands \emph{``to be shown.''} WE OUGHT TO BE IN A POSITION TO SHOW THEM. Every Lutheran soul-winner ought to be sufficiently well informed regarding the fundamental doctrines of the Bible to be able, if need be, to point out the falseness of the sectarian position.

With regard to members of \emph{non-Christian organizations}, who usually adorn themselves with the Christian name, our position is clear. They are to be regarded as being entirely outside the pale of Christianity, our approach, in thieir case, being almost entirely the same as in the case of pagans. It may not always be wise to state in so many words that the prospect cannot be considered a Christian, for he may regard himself as such, not having the slightest notion wherein the essence of the Christian faith consists. But our manner of dealing with such people will be in the nature of patient instruction concerning the fundamentals of saving faith, especially the doctrine of justification.

When members of the \emph{Roman Catholic Church} come to us, we often find that they are already convinced of the falseness of many claims made by the Pope. Our task is not so much an emphasizing of that which is wrong in the Roman Church as in bringing out the beauty of the glory and the comfort of the Bible doctrine as taught in the Lutheran Church. But it is well to have on hand some of the chief Scripture-passages directed against the outstanding abuses in Romish doctrine and life. With regard to the doctrine of works and self-righteousness as officially held in the Pope's Church we remember: --

``We conclude that a man is justified by faith, without the deeds of the Law.'' Rom. 3, 28.

``By grace are ye saved, through faith; and that not of yourselves, it is the gift of God; not of works, lest any man should boast.'' Eph. 2, 8. 9.

With regard to the worship of Many as practiced in the Catholic Church, we keep in mind Mary's own words expressing her need of a Redeemer: --

``My spirit hath rejoiced in God, my Savior.'' Luke 1, 47.

And with regard to the other saint-worship of the Catholic Church, we ought to remember the passage from the story of Cornelius and Peter: --

``And as Peter was coming in, Cornelius met him and fell down at his feet and worshiped him. But Peter took him up, saying, Stand up; I myself am also a man.'' Acts 10, 25. 26.

If Catholics accuse us of not acknowledging good works, we may safely refer them to the statements of our Confessions on this point, for they agree exactly with the demand of the Bible: --

``In Jesus Christ neither circumcision availeth anything nor uncircumcision, but FAITH, WHICH WORKETH BY LOVE.'' Gal. 5, 6. Cp. vv. 1-5.

With regard to the REFORMED church-bodies in general, we ought to remember that the ``different spirit'' of which Luther complained at Marburg, in 1529, consists chiefly in this, that they have permitted reason to be the judge of Scriptures. It was this point which proved the decisive factor and which has definitely kept the Lutheran and the Reformed churches apart; for the so-called United-Evangelical Church of Germany and other countries is not a homogeneous mass, but a conglomeration.

God does not want us to deny or set aside human reason in dealing with the eternal truths of His Word, but, as St.~Paul has it: --

``Bringing into captivity every thought to the obedience of Christ.'' 2 Cor. 10, 5.

That is to say: The greater wisdom is God's, as reason itself must acknowledge. What we are to do is to accept the divine truths without question, knowing that they are essentially above and beyond our reason.

``We preach Christ crucified, unto the Jews a stumbling-block and unto the Greeks foolishness, but unto them which are called, both Jews and Greeks, Christ the Power of God and the Wisdom of God. Because the foolishness of God is wiser than men, and the weakness of God is stronger than men.'' 1 Cor. 1, 23-25.

As far as the CHRISTIANS, or CAMPBELLITES, are concerned, it is rather difficult to fix their status, since they disclaim any confessions. Naturally we turn to their publications, only to find that there are indications pointing to a denial of the Trinity of the Bible. But the situation, on the whole, may be said to be not quite that bad. That is, individual congregations and individual members of Campbellite congregations may be sincere enough in their confession of Christianity. If they are, we must deal with them partly on the basis of what has been said regarding Reformed bodies in general, partly on the basis of what must be kept in mind concerning the Baptists.

The necessity of meeting the BAPTISTS with clear arguments from Scripture is apparent almost every day. Their viewpoint, in general, is that of the Reformed bodies, but their very particular characteristic is that connected with Baptism, especially their rejection of child baptism and their insistence upon immersion as the only correct form of baptism.

In rejecting the baptism of children, the Baptist bodies are pretty well a unit in declaring that the command to baptize does not concern children and that children can have no faith. They like to quote with approval the text Matt. 28, 20: ``Go ye therefore and teach all nations, baptizing them in the name of the Father and of the Son and of the Holy Ghost.'' The argument is that, according to the text, the teaching must come first and then the baptizing. But, aside from the fact that even the English translation, with its participle, indicates that the baptizing should at least go hand in hand with the teaching, we know that the Greek text reads: --

``Make disciples of all nations BY BAPTIZING THEM.''

As for a command of the Lord to baptize children, it is included in the very words of the divine commission, which expressly speaks of all ``nations.'' Neither men nor women nor children are separately or distinctly mentioned, for all three are included in the word ``nations.''

We must also remember what Peter told his audience in his great sermon on Pentecost Day: --

``Repent and be baptized, every one of you\ldots. For the promise is unto you \emph{and to your children}.'' Acts 2, 39.

If the argument is advanced that children, especially infants, cannot have the saving faith in their hearts, we point to such passages as the following: --

``Whoso shall offend one of these little ones which believe in Me.'' Matt. 18, 6.

``And that from a child {[}from infancy{]} thou hast known the Holy Scriptures, which are able to make thee wise unto salvation.'' 2 Tim. 3, 15.

The argument as to the \emph{form} of baptism is often perplexing, since the contention is made that the word \emph{baptizein}, in the classical language, means only ``to immerse,'' and since Luther expressed himself as preferring this form of applying the water. -- We keep in mind here that the usage of the word in the Bible determines its meaning, and we note that the word \emph{baptizein} is used as a synonym of \emph{niptein}, which clearly means ``to wash.''

``The Pharisees and all the Jews, except they wash (\emph{niptein}) their hands oft, eat not\ldots. And when they come from the market, except they wash (\emph{baptizein}), they eat not. And many other things there be which they have received to hold, as the washing (\emph{baptismous}) of cups, and pots, and brazen vessels, and of tables.'' Mark 7, 3. 4.

``And when the Pharisee saw it, he marveled that He and not first washed (\emph{baptizein}) before dinner.'' Luke 11, 38.

The reference is to the washing of the hands as it was practiced by the Jews before sitting down to a meal, and yet the word \emph{baptizein} is used.

The point that we insist upon is this, that our liberty in the choice of methods of applying the water be not interfered with. The essential thing is that water be applied; beyond this nothing is commanded by God.

With regard to the PRESBYTERIANS and all Reformed bodies of a strongly Calvinistic trend, it is to be noted that they want to confine the counsel of God regarding the salvation of men to the elect only, not for all men. Over against this terrible doctrine we hold the clear words of Scripture: --

``Got so loved the \emph{world} that He gave His only-begotten Son.'' John 3, 16.

``Christ died \emph{for all}.'' 2 Cor. 5, 15.

``God will have \emph{all men} to be saved and to come unto the knowledge of the truth.'' 1 Tim. 2, 4.

The Bible knows nothing of an election to eternal damnation, and the reason why the great majority of men are not actually saved is not to be sought in God, but in their own perversity, according to which they reject the counsel of God's love.

As for the METHODISTS, many of them are perfectionists, that is, they hold that men who have come to faith are altogether without sin. Now, it is true that the believers, by virtue of their baptism, and by virtue of their faith which receives the full atonement of Christ, are pure and clean in the sight of God. John 15, 3. According to the new man, the new nature which is ours by virtue of our conversion, we are holy in the sight of God. But according to the evil nature which we still bear around with us, which we combat all our lives, we are sinful and require daily contrition and repentance.

``If we say we have no sin, we deceive ourselves and the truth is not in us. If we confess our sins, He is faithful and just to forgive us our sins and to cleanse us from all unrighteousness.'' 1 John 1, 8. 9. Cp. Rom. 7, 14-24.

So far as the EPISCOPALIANS are concerned, very much depends upon their particular interests, whether they are High Church or Low Church people. Their confessions mean little, and each case must be judged on its own merits or demerits.

Many of the Protestant churches of our day are under the influence of Modernism and deny the fundamental facts of Holy Writ. Whenever this is the case, we govern ourselves accordingly, treating the people concerned as unchurched. At the same time, we do not wish to be identified with the Fundamentalists, much as we appreciate their stand on many of the doctrines defended by them. The thing for these Fundamentalists to do is to follow the injunction of the Bible: ``Come out from among them and be ye separate.'' 2 Cor. 6, 17. If the sifting has been done, we are in a position to meet them.

Meanwhile let us continue to MEET THE OBJECTIONS of the wrongly informed with PATIENCE TOWARD ALL MEN!

\chapter{I Will Seek That which was Lost.}\label{i-will-seek-that-which-was-lost.}

\section*{Ezek. 34, 16.}\label{ezek.-34-16.}

\subsection*{Canvassing.}\label{canvassing.}

We have now come to the point where the practical execution of the plan is the primary consideration. It is understood, of course, that the Lutheran soul-winner does not confine his efforts to any one day in the year, that he is not satisfied with one particular occasion for doing the greatest good. Our aim is to do good to all men, to try to interest them in their soul's salvation at all times, to keep the possibilities of the message of redemption in view whenever occasion offers

At the same time, experience has shown that great, united, systematic mission endeavors are productive of much good. It is self-evident, in the case of Lutherans, that the emotional element must not become too prominent. Information concerning mission-work, concerning the will of God pertaining to our sanctification, a thorough knowledge of the needs of men and of the way to help them in their spiritual need is essential to our work. Emotionalism alone is like a straw-fire, which quickly burns out and therefore is without lasting effects. The fire which we aim to kindle by our missionary endeavors is intended to set fire to heart and conscience, to mind and soul, for we want men to accept the message of the recemption of their souls through the atonement wrought by Christ.

``I will make My words in thy mouth fire.'' Jer. 5, 14.

``Thy Word was unto me the joy and rejoicing of mine heart.'' Jer. 15, 16.

``O earth, earth, earth, hear the Word of the Lord!'' Jer. 22, 29.

``Behold, the days come, saith the Lord God, that I will send a famine in the land, not a famine of bread or thirst for water, but of hearing the words of the Lord.'' Amos 8, 11.

``The Word of God is quick and powerful and sharper than any two-edged sword, piercing even to the dividing asunder of soul and spirit and of the joints and marrow, and is a discerner of the thoughts and intents of the heart.'' Heb. 4, 12.

The fundamental idea, the thought underlying the whole project and giving it the proper motivation, is that indicated by the love of the Messiah.

``I WILL SEEK THAT WHICH WAS LOST!'' Ezek. 34, 16.

That was the definite intention, the consecrated determination, of the promised Messiah, of the Christ of God, hundreds of years before He appeared in the flesh. It was not due to any fault on His part that men were lost. Every member of mankind had but himself to blame for losing his way and being in danger of damnation. All we like sheep have gone astray. -- But the Savior declared that He would SEEK that which was lost. Without any merit or worthiness on the part of those who were concerned in His counsel of love He yet decided to make the sacrifice, to offer up Himself for the salvation of the human race.

LET THIS MIND BE ALSO IN US! With this love of Christ stirring our hearts into a flame, with the full knowledge of the great issues involved, we must, in our canvassing, SEEK THAT WHICH WAS LOST. Let that object stand out before our eyes at all times, and we shall not waver in our purpose.

The suggestion is to set our great endeavor in motion with a concerted effort and with the consciousness that hundreds and thousands of our fellow-Christians throughout the Church are similarly engaged.

LET OUR MUTUAL ZEAL PROVOKE VERY MANY!

If a whole State or district or a fairly large division of a State is engaged in a systematic enterprise at the same time, the leadership may well be vested in a committee or board with an executive secretary. It would be particularly advantageous if the forces of the young people of the Church and of any missionary organizations within the Church could operate under the officers of the Church. In some sections of our Church this spirit is even now in evidence, the officers of the various bodies recognizing the possibilities of utilizing the energy of the young people and the latter offering their services to be directed according to the best interests of the Church.

Much labor may be probably saved if such a plan is executed by checking off various mission-fields in advance. In one of the Districts of our Church the following question sheet has been prepared in making ready for a systemic canvass.

\begin{verbatim}
Name of city, town, or community: ....................
County: ............  Population in 1920: ............
 1.  Are you acquainted in this city (town -- community)?
 2.  Is it a prosperous place, and is the population increasing or
   at a standstill?
 3.  How old is the place?  
 4.  Of what nationality, descent, or extraction are most of its
   people or the people in the neighborhood?
 5.  How many churches are there represented?
 6.  What Denominations?
 7.  Which branch of the Lutheran Church, if any, is represented
   there?
 8.  Do you know of any Missouri Synod Lutherans living in this
   place or in the vicinity?
 9.  Are they affiliated with any church?
10.  Are you in a position to give their names, please?
11.  Are there many unchurched people?
12.  If the population is rural, how is it served spiritually?
13.  Has our Synod ever started a mission or a congregation in
   this place?
14.  Has any other synod affiliated with the Synodical 
   Conference?
15.  If so, when (refers to Questions 13 and 14)?
16.  If work was discontinued, why was this done?
17.  If we have never done work there, was there a special
   reason?
18.  Do you think a mission could be started there now?
19.  If so, in which language should the preaching and 
   teaching be done?
20.  Other remarks.
\end{verbatim}

If this information, or a large part of it, is available in smaller towns or in rural communities, it will serve very well to prepare the ground for a systematic canvass, for it affords centers from which our work can radiate in all directions.

But whether such specific preparation went ahead or not, the work in every congregation or in every circle of congregations uniting and cooperating in this work ought to be organized.

The first step is to GAIN AND ORGANIZE the workers. We are assuming, as a matter of course, that all the members of every congregation will take an interest, and, if possible, an ACTIVE INTEREST, in every form of missionary endeavor. But there is always a large number which cannot be counted upon for actual participation. Many of the older people no longer have the physical strength demanded by a strenuous campaign. Others are often absent from the city or have work of a kind which makes it impossible for them to take part. It is to be hoped, and that most devoutly, that the number of actual shirkers in an enterprise of this importance will be very small.

The cooperation of all church-members may be solicited in enlisting them with the young people whose ministry is here particularly presupposed. The pastor may address the organizations concerned with a direct appeal to cooperate. Even a special sermon setting forth the need of personal work would be highly acceptable.

\emph{Arrange for a meeting} of all those who are interested in the systematic mission endeavor, even for those who cannot take an active part, but are willing to lend their moral and, possibly, their financial support. Open the meeting with a devotional service, which emphasizes the need of consecration and of personal work. Explain the object of the meeting as clearly as possible, with reference to the Lord's commission, the obligation of love, the world's need of the Gospel, the possibility of soul-winning, the importance of trained workers, and all the other points which are needed to bring home to every one present the paramount importance of the enterprise. Let the undertaking be discussed from the floor of the meeting. If you have not arranged for the appointment of an executive committee beforehand (which is sometimes advisable), let it be done in this meeting. Have the assembly perfect a permanent organization, with a minimum of rules and by-laws. Try to get as many pledges of WORKERS as possible.

Emphasize the need of training. \emph{Arrange meetings for information and drill}. In these meetings the informational side must be stressed most strongly; for it is knowledge on the part of the canvassers that will count, even if they are never expressly called upon to make a statement of their belief. In the pamphlet \emph{Send Me!} there are eight outlines of study, which may form the basis for systematic training. But the class may also take up the first chapters of this present book and go over them very carefully, especially with a view of becoming acquainted with the Scriptural background. The Word of God is full of divine energy and power, and an intensive study of any series of passages will invariably bring blessings which enable us to understand more fully the surpassing riches of God's grace and mercy.

At the same time the \emph{devotional element} must not be lacking in these lessons. Unless we always remain conscious of the immense responsibility laid on us by the OBLIGATION OF LOVE, our work for Christ tends to become mechanical and lifeless. The fire of our own soul will most easily kindle fire in the souls of others, while a mechanical repetition of Bible-passages may cause little response. Let every one attending the meetings realize the seriousness of the situation and the responsibility resting upon every one who takes part in this work.

The \emph{practical side of canvassing} must be both STRESSED and DRILLED. A great many people are diffident when meeting strangers; they appear shy and self-conscious. We must try to overcome this feeling for the sake of successful canvassing. If we present ourselves at the door of a strange house with an air which is either too apologetic or too bold, we may spoil our approach at once. For that reason it is well that the lessons stress points of such interest to the workers. Let those who have had successful experience in canvassing explain the manner in which they made their calls and just how they avoided being drawn into arguments without missing the chance to testify for the truth. The last meetings especially should be devoted to practical demonstrations, with the imaginary situation approaching as nearly to real life as circumstances will permit. At this time the teams of two will have been selected, and the committee in charge, under the direction of the pastor, can easily arrange a setting which will be like that actually found when the workers make their rounds.

It is self-evident that \emph{literature which is used in the systematic mission endeavor} must be understood by all those who take part in the work. Samples of canvassers' cards should be in the hands of all members as their use is explained. Let these cards be used in the practical demonstrations, so that the teams will know exactly what to do when they come to a door. Emphasize the fact that \emph{every space} should be utilized if possible. The canvassers must also understand the advertising literature of their own congregation, such as cards stating the location of the church, time of worship, pulpit programs, a letter of personal invitation from the pastor stating the outstanding features of Lutheranism, etc. These cards or letters may be left at every house of unchurched and of people coming under this heading as a matter of routine. \emph{Literature is not distributed to such as profess membership in any Christian denomination, unless such people ask for this information}.

Meanwhile the pastor (or pastors) and the committee (or committees) will prepare for the actual canvass by \emph{laying out} the field according to its geographical boundaries, so that each group working in the endeavor may have a definite district. This is essential especially where parish boundaries are no longer observed. Under no circumstances may we become busybodies in other men's matters in carrying out our systematic mission endeavor, nor may we interfere with the work of any Christian congregation. If we find people who are already members of one of our own congregations or of any other Christian congregation, it is self-evident that we do not include these in our endeavor; we simply note down the facts as given us (name, address, church connection), thank the people for the information given, and then go about our business. It has been found that an actual map of the districts canvassed, drawn as nearly as possible to scale, will greatly stimulate the interest of the workers and of other members of the congregation. If a member of the committee who has some training in statistical work will indicate, in the course of the campaign, the homes of all people visited, using tacks of a different color for such as are members of the local church, such as are members of other churches, such as are unchurched and seem to be definite prospects, and such as are unchurched and show no interest, at present, in the message of the Gospel, the map will serve for visual instruction and very likely stimulate to further efforts.

It may prove of great value to \emph{obtain} at this time, through a committee especially elected or appointed for this purpose, \emph{a list of all those} who, for any reason whatever, \emph{have drifted away from the Lutheran Church}. Let this committee be known as the Tracing Committee. In order to accomplish something for the success of the campaign, this committee will consult all available records (under the direction of the pastor), comparing those of years ago with the present church records. Records of baptisms, confirmations, marriages, even of deaths, ought to be consulted, also the lists of those who have removed from the parish, probably leaving behind them some relatives or friends. If the former address of such people is available, inquiry, if necessary at the offices of moving companies, may bring information concerning the new address. All the names gained by this search are placed in the list of the districts as outlined according to geographical boundaries in order that special efforts may be concentrated upon gaining these people once more. If they have moved to remote parts of the city or to another State, the pastor living nearest their present location ought to be notified. If they are still in the district served by the congregation, they will be given special attention in the canvass and later in the follow-up work.

Let us not forget, throughout, that our interest is not confined to former Lutherans or to people of German or Scandinavian descent, of whom we may often assume that they may feel some interest in the Lutheran Church on account of former associations. Our systematic mission endeavor has in mind the winning of as many of the unchurched as possible, regardless of race, color, and nationality. We shall, therefore, pay just as much attention to this phase of our endeavors as to any other.

After all preliminary and preparatory work has been done in a proper manner, so that all concerned in the endeavor have a clear conception of the extent and character of their participation and \emph{know exactly what they are to do}, the missionary CANVASS itself may be undertaken.

BE SURE TO HAVE ENOUGH SUPPLIES ON HAND! These include, in addition to the local advertising matter, \emph{canvassers' cards, tracts} covering topics which will probably be broached (for there is no time for arguments during a canvass), and cards for tabulating results. Envelopes showing the name and address of the church are almost indispensable. The working room of the Executive Committee ought to have cards of a regulation size, preferably 5x3, in four colors, for tabulating the results of the canvass. The last-named work should be in charge of a Typing Committee, which can immediately transfer the records from the canvassers' cards to the working records for follow-up work.

The canvass itself may, in preparatory fashion, be conducted \emph{by mail}, this work being in the hands of a Mailing Committee. The object of this undertaking is to address all those prospects who, according to the information gathered, have fallen away, in an inspiring letter, inviting them to the special service at the church, where a sermon suited to the occasion will be preached and other features may be introduced which will make this service stand out from others. If this service is followed by a social meeting in the parish hall, which is conducted in a cheerful spirit and yet in thorough harmony with the serious purpose in view, it may be productive of much good in bringing in such as are still strangers or such as have fallen away from the Church and its ministrations.

The organization for the canvass will, at this time, have about the following form: --

\begin{center}
\textsc{Congregation and Pastors}
\end{center}

\begin{center}
Executive Committee\\
and\\
Regular Officers of the Endeavor\\
\end{center}

\begin{center}
\begin{tabular}{c c c c}
Statistician & Tracing & Mailing & Typing\\
or & Committee & Committee & Committee\\
Committee on\\
Statistics
\end{tabular}
\end{center}

\[
 \overbrace{
  \textit{Canvassers in Teams}
  \hspace{2cm}
  \text{Others Lending Moral and Financial Support} 
 }
\]

Let us now take up the PERSONAL CANVASS itself, especially for supplying such information as may not as yet have been given. As stated above, much attention is given, in the last training-lessons, to practical drill in the manner of approaching people and getting results. When everybody is letter-perfect and the enthusiasm is of the right kind, the canvass proper may be held, either on a stated Sunday afternoon or within a given time, which must be exactly fixed beforehand. Before going out on their routes as assigned to them by the Executive Committee, the workerswill assemble for some ten or fifteen minutes to join in a short devotional service by the pastor. Equipped with all the supplies needed, the canvassers will then go out on their rounds.

One member of each team of two takes care of the cards, while the other does the talking and takes the lead in any short conversation at the door. The address of the place will be noted down before the door-bell is rung. As soon as some one appears at the door, the spokesman will say: --

``Good afternoon {[}or whatever time of day it may be{]}. Pardon me, we are no salesmen {[}or saleswomen{]}, but we are taking a religious census in the neighborhood. You surely will be ready to give us a little information?'' Then follow the questions for the name, church-membership or preference, if any; children from two years up, any Sunday-school affiliation, language, or whatever else may be desirable. If people declare that they are members of some church, we stop right there, for in that case the other information does not concern us. It must be definitely understood that people belonging to any other Lutheran congregation or to any other Christian denomination besides our own will not be urged to come to our churches, only the unchurched and those outside the pale of Christianity being invited. An exception holds good only in case of such people as come to us of their own free will, being anxious to find out what the Lutheran Church teaches. Be sure to express your thanks for the information before leaving the door.

Among the practical suggestions to canvassers the following from Gallmann, \emph{A Manual for Welfare Workers}, are valuable: --

~~~Treat fundamentals only\\
\hspace*{0.333em}\hspace*{0.333em}\hspace*{0.333em}Be Brief.\\
\hspace*{0.333em}\hspace*{0.333em}\hspace*{0.333em}Do not argue. He who argues is lost.\\
\hspace*{0.333em}\hspace*{0.333em}\hspace*{0.333em}Discuss only one point at a time.\\
\hspace*{0.333em}\hspace*{0.333em}\hspace*{0.333em}Be natural, sincere, courteous, not easily discouraged.\\
\hspace*{0.333em}\hspace*{0.333em}\hspace*{0.333em}Avoid the better-than-thou spirit.\\
\hspace*{0.333em}\hspace*{0.333em}\hspace*{0.333em}Avoid obnoxious features about person or appearance.\\
\hspace*{0.333em}\hspace*{0.333em}\hspace*{0.333em}Provide Christian association and maintain personal contact.\\
\hspace*{0.333em}\hspace*{0.333em}\hspace*{0.333em}Always inquire if understood\\
\hspace*{0.333em}\hspace*{0.333em}\hspace*{0.333em}Prepare subject-matter before making visit.\\
\hspace*{0.333em}\hspace*{0.333em}\hspace*{0.333em}Pray at door-step.

The \emph{cards} filled out by the canvassers will be turned in immediately to the Executive Committee and the pastor in order that they may be interpreted and the classification made. There is a great inspirational and educational value in having the canvassers supplement their written reports with an oral account of their work, and this may be done in a social meeting on an evening after all cards have been turned in.

\emph{In tabulating the result} of the canvas, it is best not to make too many subdivisions in the names submitted. Three, or possibly four, groups of prospects or missionary material may be considered. Group A will contain ``definite prospects,'' that is, all who are really interested in the Church, but have merely been battling with various adverse circumstances. Group B contains the ``uncertain,'' those who have fallen away from the Church or, while still unchurched, show only a faint interest. Group C contains the names of such as are ``very doubtful,'' such as show no interest whatever, though they are not outright hostile. A fourth group, D, will contain the names of such as are ``apparently hopeless,'' people who are outspoken in their opposition to the Church, but who are listed nevertheless, since the Lord is always able to change circumstances very quickly and cause men's hearts to turn to Him for help. God's ways in dealing with people are often strange, and it is a part of Christian wisdom to follow His guiding hand.

In \emph{making the transfer of information} from the canvassers' cards to those filed for future use, it is advisable, as indicated above, to have each group appear under a different color. It is particularly encouraging if, due to follow-up work, some cards from a lower group may be taken out, to be replaced by another card in a higher group. The Typing Committee may also prepare one or more \emph{mailing-lists} under the discretion of the pastor in order that the addresses gathered may be immedidately available for further missionary purposes.

Thus the SOUL-WINNER is actually at work carrying out the OBLIGATION OF LOVE!

\chapter{Let Us Not Be Weary.}\label{let-us-not-be-weary.}

\section*{Gal. 6, 9}\label{gal.-6-9}

\subsection*{Follow-Up Work.}\label{follow-up-work.}

There is a reason for selecting the heading of this chapter in just that form: --

LET US NOT BE WEARY!

The work of the personal mission-canvass is not easy. It often requires all the believer's physical and spiritual stamina. The great majority of people in the world are not interested in the message of salvation. The number of cheerful greetings on the part of those visited will be comparatively few. It does happen, of course, that people are frankly delighted to have us call and show a personal interest in them. That is when the canvasser feels the rich blessing which attends the soul-winner.

On the whole, it may be said that the systematic canvass has certain elements that make it attractive. There is the combination of talents and energy; there is the stimulation of the contact with kindred minds bent upon the accomplishment of a great object; there is the exhiliration of the work itself; for there is always the possibility of finding souls that hunger and thirst after righteousness.

It is different, on the whole, with FOLLOW-UP WORK. There the preliminary report has been made, the possiblity of new and encouraging discoveries is fairly remote. Follow-up work means to use endless kindness and tact and patience in keeping alive the spark of interest and in kindling it into a warm and bright flame. It means, in some cases, losing people who seemed to be fairly good prospects because they, after all, prefer the ways of the world to the ways of the Church. It means finding oneself in the spiritual condition of Elijah when he cried out in the bitterness of his heart: --

``It is enough; now, O Lord, take away my life, for I am not better than my fathers\ldots{} The children of Israel have forsaken Thy covenant, thrown down Thine altars, and slain Thy prophets with the sword; and I, even I only, am left.'' 1 Kings 19, 4. 10.

For this reason the words of the holy apostle are so wonderful for the soul-winner at this stage: --

``LET US NOT BE WEARY IN WELL-DOING; for in due season we shall reap, IF WE FAINT NOT. \emph{As we have therefore opportunity}, let us do good unto all men, especially unto them who are of the household of faith.'' Gal. 6, 9. 10.

In this we ought to be strengthened particularly by a consideration of the mercy of God in His follow-up work for men. There are the countless evidences of His love and grace in dealing with entire nations. With what unspeakably wonderful kindness and patience did He deal with His chosen people in the Old Testament! As often as they provoked Him to anger with their murmuring and with their idolatry, He nevertheless turned to them in mercy whenever they repented and cried to Him for help.

``The Lord testified against Israel and against Judah by all the prophets and by all the seers, saying, Turn ye from your evil ways and keep My commandments and My statutes, according to all the Law which I commanded your fathers, and which I sent to you by My servants, the prophets.'' 2 Kings 17, 13.

Even when the Lord was finally provoked to such depths of anger as to have Israel removed from the Land of Promise, He caused the more serious people in its midst to cast their lot with Judah; and even when Judah was led astray into captivity, He promised deliverance and return to those who would seek His face. And let us not forget that from the remnant of the covenant people who returned to the land of their fathers the Lord chose a maiden to be mother of the Redeemer, and Christ's first personal followers, the nucleus of the New Testament Church, were people who were descendants of Abraham according to the flesh.

Of such a nature is the Lord's FOLLOW-UP WORK!

And he shows the same patient love in dealing with individuals. Where would David have been if the Lord had not sought him again and again when he sinned? What would have become of Peter if there had been no room for repentance after his denial of his Savior?

We have but to examine our own lives to realize, in a measure, the patience of God in dealing with us poor sinners. As often as we have provoked Him, not only with small transgressions, but with great and deliberate sins, and that time and again, He has been ready to forgive and to forget and to pour out on us once more the fulness of His mercy in Christ Jesus.

That is true FOLLOW-UP WORK, such as we are to learn from the Lord!

In order to do real follow-up work, then, it is necessary for the worker to intensify whatever previous impressions of soul-winning he has gained. He must dwell, time and again, on the \emph{example of Scripture}, which must cause us to bring the invitation of Philip with patient repetition: --

``COME AND SEE'' John 1, 46.

No matter what objections are brought by those whom we are trying to win for Christ, and no matter how often we have confessed our belief in the Bible and all its doctrines, the final call which we must issue to all men everywhere is to COME AND SEE. Let them but examine our Confessions, let them but hear our sermons, let them but search the Scriptures, -- that is what we ask, knowing that the Holy Spirit, working through the Word, will kindle faith where and when He will.

We have the opportunity, in follow-up work, to \emph{concentrate in prayer}. We are no longer dealing with the problem as a whole, we are not regarding the indefinite in persons, but we have certain people to deal with. Of these people we possess some information, not much, perhaps, but enough to say to the Lord: ``Be hold, he whom Thou lovest is sick.'' We can present the particular difficulties of each case to the Lord, reminding Him, at the same time, that He has taught to wrestle with Him for the souls of men and that He has given us the promise: --

``My Word shall not return unto Me void!'' Is. 55, 11.

Follow-up work means that we are going to be much more interested in \emph{personal work in the congregation}. There are so many ways in which we can be of use to the Lord in the work which is already established, in teaching Sunday-school, in serving on committees, in doing clerical work, in being present at meetings where the weal and woe of the church is discussed. The greater our interest in this phase of church-work, the more we shall be prepared to serve the Lord in follow-up work dealing with prospective members.

Follow-up work means a \emph{life in the Word of God}. The Gospel, the Word of God, is the only source of spiritual life and energy which we have, the Holy Eucharist being the visible form of this Word. The more deeply we penetrate into the riches of the knowledge of God, especially as it is revealed to us in Christ, the more we shall be prepared to be soul-winners in follow up work.

Now as for the APPROACH ITSELF. This depends largely upon the contact in the canvass and in the campaign by mail. Where the prospect is practically gained, it remains to clinch his resolve. If he is favorably inclined toward accepting the invitation which has been extended to him, a tactful reminder will probably suffice. It is at this point that the words which St.~Paul writes to the Corinthians ought to be kept in mind: --

``I seek not yours, but you.'' 2 Cor. 12, 14.

Many people are extremely sensitive with regard to joining a church, for they have the strange notion that the church seeks them only on account of their contributions. Our task is to convey to all prospects the assurance that we are offering them in the Gospel far greater spiritual gifts than they can ever hope to pay for with all the wealth of the world's gold-fields.

In the case of those who are less favorably inclined or apparently hopeless we are inclined to give up too soon, particularly after the experience of a very stinging rebuff. Let us not forget that there is an importunity of faith which does not know the word failure until the Lord Himself has indicated that further work is useless.

NEVER GIVE UP UNTIL THE WORK IS DEFINITELY IN VAIN!

The number of calls which we make may be reduced, but the name of any prospect should not be taken from our mailing-lists until a conclusive demonstration of such a hardening of heart has been brought as to convince us that we should, by further efforts, be casting our pearls before swine.

Let us now summarize what may be said under the heading of follow-up work.

It stands to reason that this task should be undertaken \emph{as soon as possible after the canvass itself}. The enthusiasm of the campaign must not be spent, otherwise the reaction may find us in an apathetic mood. If the various teams have been working together with the proper zeal, they will probably be ready to go out for personal follow-up work in the district covered by them. They will look up all the people listed under Group A and invite them personally to come to services. If a special church service can be arranged and a definite, printed invitation be extended, much will be gained. But the emphasis on this feature must not be too strong, otherwise the special stimulus will always be expected, and the reaction is apt to be that following sectarian revivals. Our aim is indoctrination, and to that end we must have regular church attendance on the part of the prospects.

Above all, a positive and definite approach will rarely fail of success. If parents have promised to send their children to Sunday-school, arrangements should be made according to which some one will call for the children at least on the first Sunday, if not for an indefinite period, until they have become accustomed to the idea of coming themselves. If older people seem diffident, arrange to have some church-member living in the neighborhood call for them and accompany them to church. If the church-member ordinarily comes to church in his auto, the ride may be a secondary inducement for them to go along, warding off excuses pertaining to the weather. Care should be exercised that all people not only come to church, but also meet the pastor, if this may at all be arranged.

Be at church yourself, with a friendly greeting for such prospects as you have personally met. See that other members of the church are also given an opportunity to say a word of welcome. Be sure to avoid even the appearance of snobbery, the respect of persons, which is so severely condemned in the Bible. \emph{Read Jas. 2, 1-9}.

The organization of the committees and of the teams ought to be kept intact for at least a year at a time. This is necessary because the pastor may not be able to visit all prospects in a short time, and it is he who should by all means get in touch with such as have become interested, or may become interested, in the work of the Church. It will be advisable also that the workers, in assisting the various committes, keep in touch with all those listed, not only in Group A, but also in Group B and C, by personal visits if possible, and at least by the sending of letters of invitation and other literature, such as announcements of special services, parish-papers, pulpit programs, and other forms of publicity. If Lenten services are held in a city or community, the announcements of such services should certainly reach every person on the congregation's mailing-list.

As stated above, if there are children concerned all efforts must be bent toward gaining them at least for the Sunday-school, but, if possible, also for the day-school. If there are children past the junior age and adults who are not yet confirmed, they should be tactfully induced to attend the special classes arranged for such as desire to become acquainted with the Bible truth as taught in our Church. In every case talk over the matter with \emph{your pastor first}. This suggestion applies especially in the case of older people whom one would like to interest in church-membership and in the joining of church societies. While all prospects may be invited most urgently to attend meetings, particularly the social meetings of the various organizations in the church, the eligibility to membership must be discussed with the pastor and the church authorities first. Of course, if there is no valid reason which would prevent their joining, and especially if they have been admitted to the Lord's Supper by the pastor, then, all other things being equal, they may be admitted to the various societies. If the prospects have any ability or accomplishment along any lines whatsoever, -- the faculty to teach, to sing, or some other talent, -- they may become useful members of the various organizations concerned.

But let ALL THINGS BE DONE DECENTLY AND IN ORDER and with the object of winning souls for the Lord. For that is the spirit in which our work is done, a humble offering of our talents and services in building the Lord's kingdom.

~~~Here am I, Thou great Creator;\\
\hspace*{0.333em}\hspace*{0.333em}\hspace*{0.333em}Here am I, O Lord, send me!\\
\hspace*{0.333em}\hspace*{0.333em}\hspace*{0.333em}Here am I to do Thy bidding:\\
\hspace*{0.333em}\hspace*{0.333em}\hspace*{0.333em}What I have I owe to Thee.\\
\hspace*{0.333em}\hspace*{0.333em}\hspace*{0.333em}Soul and body, Thou hast given\\
\hspace*{0.333em}\hspace*{0.333em}\hspace*{0.333em}All of mercy, full and free\\
\hspace*{0.333em}\hspace*{0.333em}\hspace*{0.333em}That for Thee I may employ them --\\
\hspace*{0.333em}\hspace*{0.333em}\hspace*{0.333em}Here am I, O Lord, send me!

~~~Here am I, my only Savior;\\
\hspace*{0.333em}\hspace*{0.333em}\hspace*{0.333em}Here am I, O Lord, send me!\\
\hspace*{0.333em}\hspace*{0.333em}\hspace*{0.333em}Thou hast wrought my full redemption\\
\hspace*{0.333em}\hspace*{0.333em}\hspace*{0.333em}By Thy death on Calvary.\\
\hspace*{0.333em}\hspace*{0.333em}\hspace*{0.333em}For my life Thy life was given\\
\hspace*{0.333em}\hspace*{0.333em}\hspace*{0.333em}On that barren, cursed tree\\
\hspace*{0.333em}\hspace*{0.333em}\hspace*{0.333em}That my life be spent in service --\\
\hspace*{0.333em}\hspace*{0.333em}\hspace*{0.333em}Here am I, O Lord, send me!

~~~Here am I, Thou Source of power,\\
\hspace*{0.333em}\hspace*{0.333em}\hspace*{0.333em}Here am I, O Lord, send me!\\
\hspace*{0.333em}\hspace*{0.333em}\hspace*{0.333em}Thou hast wrought, O Holy Spirit,\\
\hspace*{0.333em}\hspace*{0.333em}\hspace*{0.333em}Faith my Savior for to see\\
\hspace*{0.333em}\hspace*{0.333em}\hspace*{0.333em}By the comfort of Thy presence\\
\hspace*{0.333em}\hspace*{0.333em}\hspace*{0.333em}I have strength to live for Thee\\
\hspace*{0.333em}\hspace*{0.333em}\hspace*{0.333em}In a life of love and service --\\
\hspace*{0.333em}\hspace*{0.333em}\hspace*{0.333em}Here am I, O Lord, send me!

~~~Here am I, Thou Fount of Mercy;\\
\hspace*{0.333em}\hspace*{0.333em}\hspace*{0.333em}Here am I, O Lord, send me!\\
\hspace*{0.333em}\hspace*{0.333em}\hspace*{0.333em}Wheresoe'er Thy hand directs me\\
\hspace*{0.333em}\hspace*{0.333em}\hspace*{0.333em}I shall follow willingly.\\
\hspace*{0.333em}\hspace*{0.333em}\hspace*{0.333em}If but in the slightest measure\\
\hspace*{0.333em}\hspace*{0.333em}\hspace*{0.333em}I repay my debt to Thee,\\
\hspace*{0.333em}\hspace*{0.333em}\hspace*{0.333em}Then my life has not been wasted --\\
\hspace*{0.333em}\hspace*{0.333em}\hspace*{0.333em}Here am I, O Lord, send me!

\chapter{Feed My Lambs!}\label{feed-my-lambs}

\section*{John 21, 15}\label{john-21-15}

\subsection*{Founding and Conducting a Sunday-School.}\label{founding-and-conducting-a-sunday-school.}

The subject of this chapter is very closely connected with that of all personal endeavor in mission work. It links up with the historical fact that the laymen of the early Church were actively engaged in spreading the Gospel, as we learned in Chapter 3, and that the Lord expects all Christians, whether pastors or laymen, to take a direct, personal, active interest in the spread of His kingdom.

The matter is brought home to us even by a consideration of some historical facts in church history. The Methodist Church did not come into existence until a century after the establishment of the first Lutheran congregation in America. And yet, this denomination has more than twice as many members in our country as all Lutheran bodies put together. The Baptists began work in our country in 1636, or about the same time that Lutheran preaching was established on the Delaware. Yet the Baptists, too, are much stronger than the combined Lutheran bodies of our country.

What is the explanation? We cannot well speak of laxity in doctrine and life, for both bodies, at the time of their most rapid growth, were most conservative. Moreover, some of their tenets might be considered unusually objectionable to the average person; for the Methodists had church rules which went beyond the Word of God in strictness, and the Baptists insisted upon immersion in baptism, a form which does not particularly commend itself to the average person. -- On the other hand, we know that the errors of these denominations were not the cause of their rapid growth, because the Holy Ghost does not operate through specific false teaching.

The explanation is to be found, at least in part, in the fact that the part of the truth which was stil held by these two church-bodies was spread with great zeal by laymen connected with these denominations. And one of their chief agencies in establishing new churches and in doing Home Mission work was the Sunday-school, introduced to this country immediately after the War of the Revolution. Wherever a Methodist or a Baptist settled, he tried to interest his neighbors in the establishment of a Sunday-school, and such a Sunday-school quickly became the nucleus of a congregation, for preaching was invariably introduced at the earliest opportunity.

This fact was brought home to the present writer in a very interesting and emphatic manner some years ago. I was pastor in a Western city and also had charge of a small mission-station located in one of the suburbs. This staition was visited every Sunday afternoon, and the only way to reach my destination was to take the street-car. Now, I noticed that a number of young people invariably took the same car. They were a cheerful group, but always carried on with a certain dignity. They had Bibles and other religious literature with them, and they rode beyond my destination. After some time I ventured to inquire where they were going and who they were. I received the information very quickly. They were members of one of the largest and richest churches in the city, and they went out every Sunday afternoon to conduct a Sunday-school in one of the small mining-towns to the northwest.

I found out a few more facts, in the course of time. One was this, that these young people, although, for the most part, sons and daughters of wealthy people, did not take their automobiles on these Sunday afternoon trips, lest the poor children whom they intended to serve be overawed by the splendor and have no confidence in them. Besides, this group of young people were only one of nine sets of young people from the same church, all of whom conducted mission Sunday-schools in various suburbs of the city.

These facts gave me some food for thought, and I am only too glad to pass this information on to others.

ONE CHURCH WITH NINE MISSION SUNDAY-SCHOOLS, all of them conducted by young people, whose pastor was himself a mission-worker of note and managed to inspire his young people with the desire to do some real personal work for the Lord!

This is a possibility for church expansion which has been tried in our Church in only a few isolated instances. But what a vista opens up before him who uses a little consecrated imagination together with a measure of common sense -- with 60 per cent. of our population unchurched, with approximately twenty million children growing up without any formal instruction in the truth of the Bible!

Here is work which consecrated young people (and old people) can undertake. A small amount of training is now available for Sunday-school teachers, no matter where they are found. Moreover, our pastors will be glad to conduct classes for the training of church-workers, as suggested in these pages. And the literature for Sunday-school work is being published; it can easily be turned out in much greater quantities!

What can consecrated laymen, what can SOUL-WINNERS do in furthering the cause fo the Gospel and of the kingdom of God along these lines?

Let us remember, first of all, that \emph{nothing should be undertaken where the work is already established}, where there is a congregation and a pastor within reasonable reach of any section of city or country. In this case any unchurched people, young or old, will be reached through the systematic mission endeavor described in Chapters X and XI.

Let us remember, in the second place, that \emph{work of this kind ought to be undertaken only under the auspices of the Church}. Sometimes only one pastor is concerned, as when work is to be done in a suburb adjacent to a city where we have a church. Sometimes a number of pastors and congregations will be interested, and in that case the local pastor will be glad to arrange for the work. Sometimes a layman may become interested in mission-work in some city or town at a great distance from any of our congregations. In that case it will be advisable, at least, to confer with the mission board of the District or have the local pastor take care of this formality.

The main thing is \emph{that the work be done!} And to this end a number of suggestions are here offered.

When a field such as is described above is to be opened to the Church, an \emph{exploration} is in order. If this is done by a careful canvass, it will yield the best survey of the field and give a picture of the possibilities, which can be shown in a graph.

The contact having been established by means of the canvass, the next step is to \emph{enlist the aid of a sufficient number of volunteer workers} to act as teachers and officers of the school, allowing a generous estimate of the people thus needed. These people should be pledged, in fact, they ought to pledge themselves, to work for the success of the new missionary venture with all energy; for unless workers can be depended upon, all missionary efforts lack stability.

After the teaching staff has been provided for, the \emph{territory itelf must be prepared}. If the canvassers have been successful in their first endeavor, a list of prospective pupils may be made. If there is a possibility of interesting at least some of the older people from the beginning, provisions for a Bible class should be made.

Next comes the \emph{publicity work}. A few days before the opening of the projected Sunday-school, the entire territory to be served should be covered with hand-bills, dodgers, and other forms of advertising. If the local newspaper office does the printing, there may be a possibility of getting some space in the issue preceeding the Sunday of the opening. Of course, personal invitations go to all who have signified their willingness to join the Sunday-school classes.

On the opening day the superintendent and all the teachers ought to be on hand in plenty of time in order that no delay may interfere with the organization of the school. The children and all other attendands should be warmly, but not effusively, welcomed as they arrive, and, several registration desks having been provided, their names and addresses, their age, information concerning baptism (if available), brothers and sisters, possibly also church connection or confession of parents, should immediately be entered. This must be done, of course, with proper kindness and tact. Let the teachers and officers who are not engaged in the welcoming of the children and in their registration immediately engage the attention of those already enrolled by entering into a cheerful conversation with the children or showing them some Biblical pictures provided for the walls or for individual display. Much information can be gained in this manner concerning the knowledge which the children possess in religious matters. At the same time a tentative division into classes may be made, so that all the children will be in the care of their respective teachers when opening time comes or shortly after.

Having started with a familiar, cheerful hymn, the leader should address the assembled children, bidding them welcome once more and explaining the purpose of the school in a few simple sentences. The regular lesson may then be taken up at once, for the children expect this and must not be disappointed. Since, however, the first lesson will be largely in the nature of get-acquainted hour, the lesson will be brief, and much time may be given to singing.

Be sure to rehearse this opening so carefully, and to drill all officers so thoroughly, that everything will go forward without a hitch. Very much depends on first impressions. Remember that a Sunday-school is chiefly a mission-school, and while order and discipline must prevail, it must be guided with great tact and kindliness. For this reason the devotional part of the service will also be planned with great care in order that all children may be impressed with the sacredness of the Word of God and with the privilege of taking part in lessons where it is taught.

For the organization of the Sunday-school itself and the manner in which it may best be conducted, see the pamphlet \emph{The Lutheran Sunday-school} published by Concordia Publishing House.

If the work thus begun is tactfully carried forward, so that at least some adults of the neighborhood become interested, it may not be long before regular services with preaching may be inaugurated.

Our goal is the establishment of a congregation with a Christial day-school and with all other institutions which make for the most effective church-work.

Do you realize in what spirit this work must be undertaken?

~~~I can do all things through Jesus, my Savior,\\
\hspace*{0.333em}\hspace*{0.333em}\hspace*{0.333em}Wherever He calls me to labor for Him;\\
\hspace*{0.333em}\hspace*{0.333em}\hspace*{0.333em}I can do all things, though all my endeavors\\
\hspace*{0.333em}\hspace*{0.333em}\hspace*{0.333em}Seem to lack energy, purpose, and vim.\\
\hspace*{0.333em}\hspace*{0.333em}\hspace*{0.333em}If He will call me, ``Come, work in My vineyard!''\\
\hspace*{0.333em}\hspace*{0.333em}\hspace*{0.333em}If He assigns me the work I should do,\\
\hspace*{0.333em}\hspace*{0.333em}\hspace*{0.333em}Then I shall go, for whatever befalls me,\\
\hspace*{0.333em}\hspace*{0.333em}\hspace*{0.333em}I in my faith must be steadfast and true.

~~~I can do all things through Jesus, my Savior,\\
\hspace*{0.333em}\hspace*{0.333em}\hspace*{0.333em}For in His blood my redemption is found;\\
\hspace*{0.333em}\hspace*{0.333em}\hspace*{0.333em}Often as sin and as weakness assail me\\
\hspace*{0.333em}\hspace*{0.333em}\hspace*{0.333em}He to His promise of mercy is bound;\\
\hspace*{0.333em}\hspace*{0.333em}\hspace*{0.333em}Knowing that nothing but wrath I do merit,\\
\hspace*{0.333em}\hspace*{0.333em}\hspace*{0.333em}Still for forgiveness I trustfully plead,\\
\hspace*{0.333em}\hspace*{0.333em}\hspace*{0.333em}Since in His righteousness lies my salvation\\
\hspace*{0.333em}\hspace*{0.333em}\hspace*{0.333em}If but His message of pardon I heed.

~~~I can do all things through Jesus, my Savior,\\
\hspace*{0.333em}\hspace*{0.333em}\hspace*{0.333em}Strong in the might from His power which flows,\\
\hspace*{0.333em}\hspace*{0.333em}\hspace*{0.333em}Clinging to Him as my Champion and Hero,\\
\hspace*{0.333em}\hspace*{0.333em}\hspace*{0.333em}Following on as my pathway He shows.\\
\hspace*{0.333em}\hspace*{0.333em}\hspace*{0.333em}Nothing can daunt or my confidence sever\\
\hspace*{0.333em}\hspace*{0.333em}\hspace*{0.333em}Nothing can part me from Jesus, my Lord;\\
\hspace*{0.333em}\hspace*{0.333em}\hspace*{0.333em}He will not leave me nor ever forsake me --\\
\hspace*{0.333em}\hspace*{0.333em}\hspace*{0.333em}That He has promised to me in His Word.

~~~I can do all things through Jesus, my Savior,\\
\hspace*{0.333em}\hspace*{0.333em}\hspace*{0.333em}Proud both and humble to follow His call;\\
\hspace*{0.333em}\hspace*{0.333em}\hspace*{0.333em}'Twas little children He took to His bosom\\
\hspace*{0.333em}\hspace*{0.333em}\hspace*{0.333em}'Twas little children He praised above all.\\
\hspace*{0.333em}\hspace*{0.333em}\hspace*{0.333em}If I can lead little children to Jesus,\\
\hspace*{0.333em}\hspace*{0.333em}\hspace*{0.333em}Teach them that in Him salvation they gain,\\
\hspace*{0.333em}\hspace*{0.333em}\hspace*{0.333em}Then I am happy, though humble my station,\\
\hspace*{0.333em}\hspace*{0.333em}\hspace*{0.333em}Then I am sure I have not lived in vain.

Thus we heed His injunction to

\begin{center} "FEED MY LAMBS" \end{center}

\end{document}
